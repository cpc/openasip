\documentclass[a4paper,twoside]{tce}

\usepackage{pslatex}

\begin{document}
\author{Lasse Laasonen}
\title{TTA Processor Generator}
\ver{0.50}
\lastday{13.02.2006}
% id number in S- sequence
\docnum{16}
% draft/complete/committed
\state{draft}
\maketitle

% highlighted style for parameters/variables or otherwise non-fixed parts
% of a command syntax or of lines of text displayed by the application
\newcommand{\parm}[1]{\textsl{#1\/}}



\chapter*{Version History}

\begin{HistoryTable}

 0.1    & 23.08.2004 & L. Laasonen   &
 First incomplete draft.\\

 0.2    & 02.09.2004 & L. Laasonen   &
 Added ``Product Features'' chapter; a lot of small changes.\\

 0.3    & 17.09.2004 & L. Laasonen   &
 Updates in requirements.\\

 0.4    & 24.09.2004 & L. Laasonen   &
 Added format declarations in IDF and metadata files, removed unnecessary
 chapters.\\

 0.5    & 29.09.2004 & A. Cilio &
 Changes to metadata file format. Output files in a separate
 directory. Revision of text.\\

 0.5.1  & 30.09.2004 & A. Cilio &
 Added bibliography (it was forgotten).\\

 0.6    & 21.10.2004 & L. Laasonen &
 Changed metadata file and IDF formats. Added search paths for HBL's and
 plug-ins. Changed usage. Added support for presynthesized library.\\

 0.7    & 28.10.2004 & A. Cilio &
 Complete review. Changed metadata file format. \\

 0.8    & 02.11.2004 & L. Laasonen &
 Changed order of HBL and plug-in search paths. Now `-' instead of `\_' in
 file and directory names in HBL. Changes in `library-info' files.  Removed
 naming rules of source files in HBL. File name `ttacore' changed to
 `toplevel'.\\

 0.9 & 08.11.2004 & L. Laasonen &
 Reverted order of HBL and plug-in search paths. Replaced use of entity word
 with definition block. In HBL, specified names of load enable ports.\\

 0.10 & 10.11.2004 & L. Laasonen &
 Added BEM file format specifications.\\

 0.11 & 11.11.2004 & L. Laasonen &
 Added format of library-info file. Changed IDF format. Changes in User
 Interface section.\\

 0.12 & 15.11.2004 & L. Laasonen &
 Changes in `block-info' file. Added an command line option to specify the
 output directory.\\

 0.13 & 16.11.2004 & L. Laasonen &
 Changes in `block-info' for FU specifications. Removed immediate units from
 HBL.\\

 0.14 & 25.11.2004 & L. Laasonen &
 Added \emph{external-port} to FU `block-info'.  Added \emph{pkg-dependency}
 to `library-info'.  Added `pkg' folder to HBL. Combined GCU and IC
 generators into one plug-in.\\

 0.14.1 & 29.11.2004 & L. Laasonen &
 Added a note on opcodes.\\

 0.15 & 01.12.2004 & L. Laasonen &
 Removed external port definitions from `block-info' and added new
 `port-info' file to leaf directories of HBL.\\

 0.16 & 10.12.2004 & L. Laasonen &
 Renamed `port-info' file, now `implementation-info'. Added name of hardware
 definition block to meta data. Moved operation code declarations to
 implementation-specific meta data.\\

 0.17 & 14.12.2004 & L. Laasonen &
 Added latency information to meta data of register files.\\

 0.18 & 13.01.2005 & L. Laasonen &
 Lots of changes in BEM file format.\\

 0.19 & 14.01.2005 & A. Cilio    &
 Complete revision of BEM file format.\\

 0.20 & 17.01.2005 & A. Cilio    &
 Changes to BEM: removed control port code, added entry for merged
 port-operation code. Minor changes.\\

 0.21 & 19.01.2005 & L. Laasonen &

 Changes to HBL metadata file formats. Ports now in
 `implementation-info'. Added chapter on HBLmanager.\\

 0.22 & 21.01.2005 & L. Laasonen &
 Added bridge address declaration to BEM.\\

 0.23 & 25.01.2005 & L. Laasonen &
 Revised bridge address declaration in BEM.\\

 0.24 & 03.02.2005 & L. Laasonen &
 Removed \emph{cu-ra-port-code} element from BEM. \\

 0.25 & 04.02.2005 & L. Laasonen &
 Revised structure of document. Added ADF as input argument to ProGe command
 line. Revised naming of HDL files in HBL. Changed plug-in specification in
 IDF. Corrections suggested by Pekka J��skel�inen. \\

 0.26 & 10.02.2005 & A. Cilio &
 Thorough revision of Chapters 1--4. Still pending issues.\\

 0.27 & 10.02.2005 & A. Cilio &
 Completed revision. Still pending issues.\\

 0.27.1 & 10.02.2005 & L. Laasonen &
 Fixed a couple of typing errors.\\

 0.28 & 10.02.2005 & A. Cilio &
 Major revision. Removed multiple HDL's in single HBL. Changed use of HBL's
 (no repertoire). Changed search paths. Changed block parameters, no default
 values.\\

 0.29 & 18.02.2005 & L. Laasonen &
 Added option to select HDL of top level. Added support for mixed-HDL
 processors. Removed renaming of hardware definition blocks. Added
 \emph{target-language} element to `implementation-info' file. Format of
 'library-info' file modified. Added `header' folder to HBL.\\

 0.30 & 04.03.2005 & L. Laasonen &
 Added options to list plug-ins and inspect their properties to ProGe. Added
 support for plug-in parameters.\\

 0.31 & 04.03.2005 & L. Laasonen &
 HBL can contain pre-synthesised blocks. Now `library-info' can give the
 name of the file containing the synthesised block.\\

 0.32 & 08.03.2005 & L. Laasonen &
 Removed option `use-sources' (unnecessary).\\

 0.33 & 09.03.2005 & L. Laasonen &

 Removed direction of FU ports from `block-info'. Structured output
 directory into subdirectories. Extension idea: ADF to block and
 implementation information converter.\\

 0.33.1 & 11.03.2005 & L. Laasonen &
 Extension idea: smart block selector.\\

 0.34 & 11.03.2005 & L. Laasonen &
 VHDL definitions of FU's and RF's are used from \emph{functionunits}
 and \emph{registerfiles} packages, respectively. HBLmanager updates
 the package files automatically.\\ 

 0.35 & 15.03.2005 & L. Laasonen &
 Major change to BEM. Removed operation code table and bit width definitions
 (widths now computed automatically). Added extra-bits definitions.\\

 0.35.1 & 17.03.2005 & A. Cilio &
 Added details to smart block selector. Minor changes.\\

 0.35.2 & 18.03.2005 & L. Laasonen &
 Minor changes.\\

 0.36   & 08.04.2005 & A. Cilio &
 Re-introduced renaming of HDL files.  Added ProGe use cases.\\

 0.37   & 15.04.2005 & A. Cilio &
 Reformulated ProGe use cases.\\

 0.38   & 21.04.2005 & L. Laasonen &
 Changes to BEM: FU port code definitions, in-line immediate control tag.
 Other minor changes.\\

 0.38.1 & 25.04.2005 & A. Cilio &
 Added error variant to ProGe use cases. \\

 0.39   & 26.04.2005 & L. Laasonen &
 Reverted change to in-line immediate control tag.\\

 0.40   & 29.04.2005 & L. Laasonen &
 Added \emph{no-operation} element in BEM specification.\\

 0.41   & 16.05.2005 & A. Cilio &
 Reformulated ProGe use cases according to J.~Takala's comments.\\

 0.42   & 03.06.2005 & L. Laasonen &
 Added \emph{always-true-guard-code} and \emph{always-false-guard-code}
 elements to the BEM file specification.\\

 0.43   & 20.06.2005 & L. Laasonen &
 Added immediate slot declaration to BEM.\\

 0.44   & 14.07.2005 & L. Laasonen &

 Changes to BEM: the \emph{extra-bits} element of RF and IU port code
 declarations must be omitted when \emph{encoding} element is missing.\\

 0.44.1 & 17.08.2005 & A. Cilio &
 Minor changes, spell check. Added missing citations.\\

 0.44.2 & 18.8.2005 & L. Laasonen &
 Fixed an error: \emph{destination} element of \emph{slot} element
 contained \emph{no-operation} element too, by mistake.\\ 

 0.45 & 13.09.2005 & L. Laasonen &
 HBL replaced with HDB. ``IC/GCU plugin'' renamed ``IC/Decoder plugin''.
 Commented out the HBLManager chapter (to rewrite).\\

 0.45.1 & 23.09.2005 & L. Laasonen &
 Removed the \emph{adf} attribute from the root element of IDF file.\\
 
 0.46 & 12.10.2005 & P. J��skel�inen &
 Added \emph{bus} and \emph{socket} to IDF format.\\
 
 0.47 & 02.11.2005 & P. J��skel�inen &
 Updated IDF specification. \\
 
 0.47.1 & 02.11.2005 & A. Cilio &
 Minor corrections. Removed chapter ``HBLManager''. \\

 0.48 & 08.12.2005 & L. Laasonen &
 Added \emph{immediate-register-field} element to BEM file
 specification. \\

 0.49 & 08.02.2006 & L. Laasonen &
 Added \emph{width} attribute to \emph{immediate} element and
 \emph{no-operation} element to destination field declaration in BEM file
 specification.\\

 0.50 & 13.02.2006 & L. Laasonen &
 Added \emph{id-pos} element to source and destination field
 declarations in BEM file. Removed \emph{pos} attribute from
 \emph{socket} element in BEM file. Position elements define relative
 positions instead of absolute.\\

\end{HistoryTable}


\tableofcontents



\chapter{INTRODUCTION}

\section{Purpose}

This document describes the TTA Processor Generator (ProGe), an application
of the TCE toolset. The purpose of the document is to give an overview of
ProGe and define the functional specifications needed to design it.

\section{Product Idea}

% (PJ: product)

%  Describe the project motivation in few words and the context in which
%  this product should have a reason to exist - see TCE Project Plan for an
%  example

The goal of the TCE toolset is to provide semi-automatic design platform for
TTA processors. To be able to realize the processors designed with the TCE
toolset, it is necessary to generate synthesizable descriptions of target
processors in a hardware description language (HDL).

\section{Product Overview}

% (PJ: product + overview)

% This section describes as briefly as possible the system that implements
% the functionality wanted.

ProGe is a command line application that generates HDL files from the given
processor definition file. The main input data file read by ProGe is the
Implementation Definition File (IDF), whose format is described later in
this document.

The first version of ProGe will support VHDL as sole hardware
description language, but ProGe is designed to be
language-independent. To add support for other target languages, such
as Verilog, should be straightforward.

The main building blocks of a target processor, such as Function Units
(FU's), Register Files (RF's) and Immediate Units (IU's), are too
complex to be generated automatically by ProGe. Instead, they are
generated off-line by the user and stored in Hardware Database
(HDB). The IDF file given to ProGe tells what blocks ProGe should
select from the database. TCE provides an independent application for
maintaining the HDB.

A fundamental requirement of ProGe is support for custom, user-defined
implementations of interconnection network (IC) and global control unit
(GCU). A few predefined implementation alternatives of interconnection
network and a simple control unit are provided.

ProGe generates an output directory and stores several files in
it. The top-level of the processor is described in one file. This file
wires together all the blocks taken from HDB with IC and GCU. The
number and type of files generated depend on the chosen IC/Decoder
generator. For example, an IC/Decoder generator could create one file
for the interconnection network and another for the decoder of the
control unit. The hardware descriptions of building blocks taken from
HDB are in separate files as well, since they are copied from files
stored in the file system.

ProGe works also with synthesized hardware libraries generated by
third-party tools, such as Synopsys high-level synthesis toolset, for
example.

\section{Definitions}

\begin{description}
\item[opcode]%
  Operation code. A bit pattern used to select an operation to be executed in
  a function unit.
\end{description}

\section{Acronyms and Abbreviations}

\begin{center}
\begin{tabular}{p{0.15\textwidth}p{0.85\textwidth}}
ADF   & Architecture Definition File \\
BEM   & Binary Encoding Map \\
CostDB& Cost Database \\
FU    & Function Unit \\
GCU   & Global Control Unit \\
GUI   & Graphical User Interface\\
HBL   & Hardware Block Library \\
HDL   & Hardware Description Language \\
IC    & Interconnection network\\
IDF   & Implementation Description File \\
IU    & Immediate Unit \\
PCF   & Processor Configuration File \\
PIG   & Program Image Generator \\
ProGe & TTA Processor Generator \\
RF    & Register File \\
TCE   & TTA Codesign Environment \\
TTA   & Transport Triggered Architectures \\
VHDL  & Very High Speed Integrated Circuit Hardware Description Language \\
\end{tabular}
\end{center}

\section{User Profile}

%  What's the user like? Student? Researcher? Professional? Expertise in
%  hardware or software expert? How much skilled?

Users of ProGe fall into three groups:
\begin{enumerate}
\item%
  Users with little or no expertise on hardware design, who work only with
  predesigned HDB's and HDL generator modules.
\item%
  Hardware experts that design new building blocks for TTA processors and
  are able to update the HDB.
\item%
  Hardware experts that design new plug-in software modules for generating
  hardware descriptions of interconnection network and processor control
  blocks.
\end{enumerate}


ProGe offers a very simple and easy-to-learn interface for users that work
only with existing building blocks. Even users without a strong background
on hardware or software development should be able to generate TTA target
processors with minimal effort.

If the target TTA processor under development contains custom building
blocks that are not found in the default hardware databases, then a
stronger background on hardware development is required. Before a custom
building block can be utilised by ProGe, the user has to add and register
hardware descriptions of it into a hardware database.

The design of custom blocks is not actually part of the work flow of ProGe,
but is a necessary preliminary step. Of course, designing custom blocks in a
hardware description language requires expertise of the HDL.

Creating a new IC/Decoder generator requires both hardware expertise
and some basic knowledge about software development, since the
plug-ins are implemented in C++.


\chapter{PRODUCT OVERVIEW}

\section{Product Perspective}

% (external connections)

%  High-level description of the product with focus on its communication
%  with external products/modules. What external modules does it depend
%  from?  What is provided in the product and what is required and assumed
%  available as external service?

ProGe is a client application and is part of the TCE hardware
subsystem. No other module of TCE depends on ProGe.

The user interface of ProGe (in its simplest variant, an interface based on
command line) is easily replaceable with other user interfaces, like a
graphical user interface (GUI), without affecting or requiring adaptations
to the core functionality of the application.

ProGe requires a complete specification of the target processor
implementation, which is specified by means of an IDF file. In
addition, ProGe depends on the binary encoding map, a data base which
specifies the binary encodings of any programmable part of the target
processor, such as sockets, guards, bridges and operation codes.

HDL definitions of certain classes of building blocks of the processor
(currently, function units and register files) are stored in the file
system of the OS and are referred to in HDB. HDB contains also
essential information of the building blocks to be taken into use in
the HDL implementation of the processor.

ProGe does not restrict the hardware description language. The building
block descriptions can be expressed in any language. As long as the language
is supported by third-party synthesis and simulation applications, it can be
used to define processor building blocks.  An HDB and a target processor
generated by ProGe may contain building blocks expressed in different HDL's.

The same building block may be implemented with different hardware
architectures (``high-level implementations'', in our parlance). Thus, HDB
can contain several equivalent\footnote{
%
  Two blocks are ``equivalent'' when replacing one with the other does not
  change the behaviour of any program running on the target processor.}
%
blocks with different HDL specifications.

\section{Product Position Statement}

To provide the capability to generate a synthesisable hardware description
of TTA target processors designed with TCE toolset. This hardware
description can be simulated, synthesised and implemented in silicon using
third-party tools.

\section{Use Cases}

\subsection{Generation of a Fully-Specified Processor from Perfectly
  Matching Library Blocks}
\label{sec:uc-fully-specified}

Purpose: To generate a synthesisable hardware description of a processor
given a description of target architecture and a reference, for each
predesigned processor block, of which descriptions of block implementations
should be included in the output description.

\begin{flushleft}
\begin{tabular*}{\textwidth}{@{\extracolsep{\fill}}lp{0.88\textwidth}}
Inputs:  &
architecture description, reference to block implementations to be used for
generation, set of hardware descriptions of block implementations\\
Outputs: &
synthesisable HDL that describes the processor\\
\end{tabular*}
\end{flushleft}

\begin{enumerate}
\item\label{itm:input-data}%
  The user specifies an input architecture and references to implementations
  of processor building blocks to be used during the generation process.

\item\label{itm:input-hdb}%
  The user specifies the set of predefined descriptions of processor
  building blocks from which to select the appropriate hardware
  descriptions.

\item\label{itm:input-plugin}%
  The user selects one of the several hardware architectures that
  implement the interconnection network and the instruction decoder of
  the global control unit.

  \emph{Error variant}: If the hardware architecture specified is not known,
  the generator aborts with error message.

  \emph{Error variant}: If the hardware architecture specified is not
  compatible with the transport pipeline delay required by the architecture
  description, the generator aborts with error message.

\item\label{itm:arch-lookup}%
  For each processor building block that is predefined (not generated from
  scratch by the processor generator), the generator looks for a block
  implementation that matches the reference given by the user.

\item\label{itm:plugin-run}%
  The generator completes the processor description by generating the
  interconnection network and the processor control path, including fetch
  and decode logic, according to the hardware architecture requested by the
  user.

\item\label{itm:output-generation}
  The generator creates output data.
\end{enumerate}

\emph{Variant:} Replace (\ref{itm:input-plugin}) as follows: If the user
does not specify any hardware architecture for IC/Decoder, then the generator
selects the default architecture.

\subsection{Obtaining a List of Available Plug-ins}

Purpose: To obtain a list of available hardware architectures
available for implementing interconnection network and the decoder of
the global control unit.

\emph{Inputs}: None.
\emph{Outputs}: None.

\begin{enumerate}
\item %
  The user runs ProGe with an option for requesting a list of available
  hardware architectures of IC and decoder.
\item %
  ProGe outputs a list.  Each element of the list gives the name of a
  hardware architecture and a brief description of it.
\end{enumerate}

\subsection{Retrieving Plug-in Information}

Purpose: To obtain a description of one of the available hardware
architectures for the interconnection network and global control unit.

\emph{Inputs}: None.
\emph{Outputs}: None.

\begin{enumerate}
\item %
  The user runs ProGe with an option for requesting descriptions of
  IC/Decoder hardware architectures and the name of the requested
  architecture.
\item %
  ProGe outputs a description of the hardware architecture requested,
  including: properties of the hardware architecture, name and purpose of
  accepted parameters to tune it.

  \emph{Error Variant:} If no hardware architecture with given name is
  registered as valid IC/Decoder hardware architecture, ProGe outputs
  an error message and quits.
\end{enumerate}

\section{Summary of Capabilities}

% (advantages)

\begin{center}
\newcommand{\tw}{\textwidth}
\begin{longtable}{@{}p{0.42\tw}@{\hspace{0.03\tw}}p{0.55\tw}}
  User Benefit & Supporting Feature \\[1.5ex]

Easily customizable IC and GCU &
HDL generator modules linked at run time (plug-ins). \\[1ex]

Support for alternative implementations of building blocks &
HDB supports multiple hardware architectures of a block; IDF is used to
select the desired versions. \\[1ex]

Support for synthesised building blocks &
HDB contains necessary information to make pre-synthesised version of
the hardware block usable in the generated processor definition.\\[1ex]

Ease of use &
Default IC/Decoder generator provided. HDB contains implementations of common
building blocks. If missing, BEM is generated automatically from
ADF. Default implementation of hardware blocks when implementation
definition is missing.\\[1ex]

Multi-HDL support &
ProGe is HDL-aware and can select building blocks defined in different HDL's
for the same target processor.\\[1ex]

Possibility to access building blocks &
Standalone helper application: hardware database manager.\\[1ex]

\end{longtable}
\end{center}

\section{Assumptions and Dependencies}

% (portability)
% (PJ: design constraints)

%  Assumed environment, third-party applications or library assumed to be
%  present. External services (like OS services) that are required.

% PEKKA:
% For example "the GUI should use wxWindows GUI library to ease portability
% to other operating systems and environments" or "XML should be used for
% all clear text persistent data storage".

%  In case of functional specifications of most modules, it's sufficient to
%  specify what is required in addition to assumption/dependencies given in
%  main functional specifications document of TCE Project, and refer to it
%  for the rest.

In addition to the general dependencies specified in~\cite{ProjectPlan},
ProGe requires operating system support for a programming interface to
dynamic linking loader. This interface is standard in recent versions of
Linux/GNU operating systems, and should be compatible with the interface on
Sun Solaris.



\chapter{PRODUCT FEATURES}

% This chapter describes the functionality of the product in detail.

% In case of an application, all command line arguments and user interface
% screens (GUI) are described in detail.  Also, all tasks the module or tool
% must be able to accomplish are described here.

% Organise in sections.  Dedicate one section for each main feature.

\chapter{DATABASES}

This chapter describes the data that is processed and used by ProGe and its
helper applications from a user point of view. The data formats are defined
in a complete way.

For information about the implementation of the data bases described here,
the reader is referred to the Processor Generator Design
Document~\cite{ProgeDesign}.

\section{Hardware Database}

Hardware Database stores the implementation specific data of processor
building blocks needed to make them usable in the generated processor
definition. For more detailed information, see \cite{HDBDesign}.

\section{Intensity of Use}

ProGe is used rarely. Other applications do not use the modules of ProGe.
For these reasons, performance requirements of ProGe are not high.

\section{Files and Configuration}

\subsection{Processor Configuration File}
\label{ssec:proge-conf}

The Processor Configuration File (PCF) is the input file usually given to
ProGe. This file just wraps up the files that describe different aspects
(architecture, implementation, instruction encoding) of the processor. Those
files are ADF, IDF and BEM (defined in Section~\ref{ssec:bem-specs}). The
format of PCF is defined in~\cite{TCE-specs}.

\subsection{Architecture Definition File}

The Architecture Definition File~\cite{ADF-specs} defines the architecture
of the target processor to be generated. It defines the FU's
and RF's and the interconnection of buses, sockets and
ports of the processor.

\subsection{Binary Encoding Map File}
\label{ssec:bem-specs}

The Binary Encoding Map (BEM) defines how TTA instructions are encoded into
bit vectors. The instruction words are a concatenation of bit patterns that
identify operations, ports, registers, socket and so on.
%
The data contained in BEM affects only generation of the control path of the
target processor. The binary encodings are needed by the GCU generator and
the IC generator. The reference to BEM file is an optional entry of PCF. If
the reference is omitted, a BEM is generated automatically using
architecture definition, HBL and, if available, implementation information
as input data.

\paragraph{Format overview and general conventions.}
The BEM file format is based on XML. The bit positions of instruction fields
are counted starting from the least significant bit (position zero) at the
right end of the bit vector. The bits occupied by a field are to the left of
the starting position.

An example of the breakdown of a TTA instruction into its fields is
depicted in Figure~\ref{fig:tta-instruction}. Subfields in the
instruction can be in any order except \emph{opcode} and {port ID}
within the \emph{socket code} field. The figure depicts just one
example.

\begin{figure}[tb]
\centerline{\psfig{figure=eps/instruction.eps,width=1.0\textwidth}}
\caption{TTA instruction format.}
\label{fig:tta-instruction}
\end{figure}

The binary codes can be expressed in three different bases: decimal,
hexadecimal (prefix `0x'), and binary (prefix `b'). All other numbers must
be expressed in decimal base.

The following sections define the format of the BEM file.

\subsubsection{Top Level}

The root element of a BEM file is \emph{adf-encoding}. This element contains
all the encoding data and in addition specifies the version of the BEM file
format and the oldest version that is compatible with current version.

Version information is stored in the form of attributes of the top-level
declaration element. For example, the following root element declares that
the BEM format is version ``1.1'' and that all versions of the parser
starting from ``1.0'' can correctly read this file.
\begin{verbatim}
<adf-encoding version="1.1" required-version="1.0">
    ...
</adf-encoding>
\end{verbatim}

The \emph{adf-encoding} element contains the following types of elements:
\begin{enumerate}
\item%
  \emph{map-ports} (Section~\ref{ssec:map-ports});
\item%
  \emph{long-immediate-tag} (Section~\ref{ssec:lit});
\item%
  \emph{immediate-register-field} (Section~\ref{ssec:irf}).
\item%
  \emph{slot} (Section~\ref{ssec:slot}).
\item%
  \emph{immediate-slot} (Section~\ref{ssec:immediate-slot}).
\end{enumerate}

\subsubsection{Socket Code Table}
\label{ssec:map-ports}

A socket code table declares the control codes carried by a socket.
Each control code identifies one of the ports the socket is connected
to and possibly an operation code. The corresponding field in the TTA
instruction is \emph{socket~code}, shown in
Figure~\ref{fig:tta-instruction}.
%
A table can be shared by several sockets, on condition that all are
connected to exactly the same types of ports.

The socket code table has the following format:
\begin{verbatim}
<map-ports name="string">
    <extra-bits> number </extra-bits>
    <fu-port-code name="string" fu="string" operation="string">
        <encoding> number </encoding>
        <extra-bits> number </extra-bits>
    </fu-port-code>
     ...
    <rf-port-code rf="string" index-width="number">
        <encoding> number </encoding>
        <extra-bits> number </extra-bits>
    </rf-port-code>
     ...
    <iu-port-code iu="string" index-width="number">
        <encoding> number </encoding>
        <extra-bits> number </extra-bits>
    </iu-port-code>
     ...
</map-ports>
\end{verbatim}

The socket code table is identified by a unique string, given by its
\emph{name} attribute. Element \emph{map-ports} contains the
following types of element:

\begin{enumerate}
\item%
  Element \emph{extra-bits} defines the number of extra (zero) bits
  reserved for the \emph{socket code} field. That is, the width of the
  field is stretched the given amount of bits. In normal case, the
  value should be '0'.

\item%
  Element \emph{fu-port-code} maps the port identified by the
  \emph{name} attribute of the function unit identified by the
  \emph{fu} attribute to a control code.

  Certain FU ports may include an operation code that is passed on to
  the target unit.  The optional attribute \emph{operation}, when
  present, gives the name of the operation whose operation code is
  carried along with the port data. There must be a separate
  \emph{fu-port-code} element for each of the operations carried by
  the port. The control code is defined in two parts: the
  \emph{encoding} element defines the encoding for the port and the
  \emph{extra-bits} element defines the number of prepending zero bits
  in the encoding. For example, to define encoding '0011' for a port
  the \emph{encoding} element must have value '3' and the {extra-bits}
  element must have value '2'. Note that the encodings must not be
  ambiguous. For example, encodings '111' and '11' are ambiguous. If
  you examine the two rightmost bits of the field, you cannot say
  which one of the encodings is the case. However, encodings '111' and
  '011' are not ambiguous.

\item%
  Element \emph{rf-port-code} maps one of the ports of the register file
  identified by the \emph{rf} attribute to a control code.

  The data that passes through a register file port is paired with a
  register index. Attribute \emph{index-width} specifies the number of bits
  of the control code that are interpreted as register index. For example,
  the port of a register file with $2^3 = 8$ registers will have an
  \emph{index-width} attribute with value `3'.

  The \emph{encoding} and \emph{extra-bits} elements have the same
  meaning as in the FU port code declaration. They are used to define
  the encoding for the port.

\item%
  Element \emph{iu-port-code} maps one of the ports of the immediate unit
  identified by the \emph{iu} attribute to a control code.

  The data that passes through the port of an immediate unit is paired with
  a register index. Attribute \emph{index-width} specifies the number of
  bits of the control code that are interpreted as register index. For
  example, the port of an immediate unit with $2^0 = 1$ register will have
  an \emph{index-width} attribute with value `0'.

  The \emph{encoding} and \emph{extra-bits} elements have the same
  meaning as in the FU port code declaration. They are used to define
  the encoding for the port.
  
\end{enumerate}

\emph{fu-port-code}, \emph{rf-port-code} and \emph{iu-port-code}
elements are optional and may appear multiple times.
%
A socket code table must contain at least one port code for each
port connected to the socket it provides the encoding for. For opcode
setting ports of FU's, socket code table must contain a port code
for each of the operations. All types of port codes can be mixed in
the same socket code table.  However, there must be no more than one
\emph{rf-port-code} (\emph{iu-port-code}) element for any given value
of the \emph{rf} (\emph{iu}) attribute.\footnote{
%
  This restriction reflects the fact that a socket cannot be connected with
  two ports of the same register file. Further restrictions may be required
  due to limitations in the architectures supported.}

\paragraph{Special case: 0-bit port ID's.}
The control code carried by a socket may consist purely of index or opcode
bits. This is the case when the socket is connected to just one port of a
register file or immediate unit, or one opcode-setting port of a function
unit. In the case of register file or immediate unit port, the
\emph{encoding} and \emph{extra-bits} elements are omitted. In the
case of opcode-setting port of an FU, each operation is normally
defined by \emph{fu-port-code} elements.

\paragraph{Restriction on encoding of RF and IU ports.}
All ports of a given RF or a given IU are totally identical from the
functional point of view, therefore they are not distinguished in BEM. The
ports are physically distinguished by the fact that each must be connected
to a different socket.

\subsubsection{Long Immediate Tag}
\label{ssec:lit}

The long immediate tag declares the encoding of immediate templates. An
immediate template is used to encode long immediate in the instruction
stream.

A long immediate tag has the following format:
\begin{verbatim}
<long-immediate-tag>
  <pos> number </pos>
  <extra-bits> number </extra-bits>
  <map name="string"> number </map>
   ...
</long-immediate-tag>
\end{verbatim}

The \emph{long-immediate-tag} contains the following elements:

\begin{enumerate}
\item%
  The \emph{pos} element defines the relative position of the field in the
  instruction.
\item%
  The \emph{extra-bits} element defines the number of extra (zero)
  bits reserved for the field. In normal case the value should be '0'.
\item%
  The \emph{map} element defines the encoding of the instruction template
  referred to by the \emph{name} attribute. Element \emph{map} can be
  repeated several times.
\end{enumerate}

\subsubsection{Immediate Register Field}
\label{ssec:irf}

An immediate register field declaration describes one field of an
instruction word that contains a register index of an immediate unit
to which the long immediate given in the instruction is written.

The declaration has the following format:

\begin{verbatim}
<immediate-register-field>
  <pos> number </pos>
  <width> number </width>
  <instruction-template name="string">
    <reg-index-of> string </reg-index-of>
  </instruction-template>
   ...
</immediate-register-field>
\end{verbatim}

An immediate register field declaration contains the following
elements:

\begin{enumerate}
\item%
  The \emph{pos} element defines the relative position of the field
  within the instruction.
\item%
  The \emph{width} element defines the bit width of the field.
\item%
  The \emph{instruction-template} element defines an instruction
  template which uses the field. It has a mandatory \emph{name}
  element which gives the name of the instruction template. The
  element contains \emph{reg-index-of} element which defines the name
  of the immediate unit of which register index is encoded in the
  field.
\end{enumerate}

\subsubsection{Move Slot}
\label{ssec:slot}

A move slot declaration describes one field of an instruction word
that is used to program the transports of data over the move busses.

The declaration has the following format:

\begin{verbatim}
<slot name="string">
  <pos> number </pos>
  <extra-bits> number </extra-bits>
  <guard> ... </guard>
  <source> ... </source>
  <destination> ... </destination>
</slot>
\end{verbatim}

A move slot declaration has a mandatory \emph{name} attribute and contains
four or five elements. The \emph{name} attribute gives the name of the
transport bus programmed by this move slot.

The slot element contains the following elements:

\begin{enumerate}
\item%
  The \emph{pos} element defines the relative position of the slot within the
  instruction.
\item%
  The \emph{extra-bits} element defines the extra (zero) bits reserved
  for the move slot field. In normal case the value should be '0'.
\item%
  The optional \emph{guard} element declares encoding of the move guard
  field. This field exists only if the bus referred to by the \emph{name}
  attribute of \emph{slot} element supports at least one guard.
\item%
  The \emph{source} element declares encoding of the move source field.
\item%
  The \emph{destination} element declares encoding of the move destination
  field.
\end{enumerate}

\paragraph{Guard field of move slot.}
The \emph{guard} element has the following format:

\begin{verbatim}
<guard>
  <pos> number </pos>
  <extra-bits> number </extra-bits>
  <reg-guard-code rf="string" index="number" inv="enum"> number </reg-guard-code>
   ...
  <port-guard-code fu="string" port="name" inv="enum"> number </port-guard-code>
   ...
  <always-true-guard-code> number </always-true-guard-code>
  <always-false-guard-code> number </always-false-guard-code>
</guard>
\end{verbatim}

The meaning of the elements is the following:

\begin{enumerate}
\item%
  The \emph{pos} element defines the relative position of the guard
  field within the move slot.
\item%
  The \emph{extra-bits} element defines the number of extra (zero)
  bits reserved for the guard field. In normal case the value should be '0'.
\item%
  The \emph{reg-guard-code} element specifies the encoding for a guard
  expression computed out of a GPR.

  The register file of the GPR is given by attribute \emph{rf}; the register
  index by attribute \emph{index}.
%
  The \emph{inv} attribute indicates whether the Boolean value resulting
  from the evaluation of the guard expression must be inverted or not.
  Possible values of the attribute are `true' and `false'.

\item%
  The \emph{port-guard-code} element specifies encoding for the guard
  expression computed out of a function unit output port.

  The function unit is identified by attribute \emph{fu}; the port by
  attribute \emph{port}.
%
  The \emph{inv} attribute indicates whether the Boolean value resulting
  from the evaluation of the guard expression must be inverted or not.
  Possible values of the attribute are `true' and `false'.

\item%
  The \emph{always-true-guard-code} element specifies encoding for the
  always true guard.
\item%
  The \emph{always-false-guard-code} element specifies encoding for
  the always false guard. 
\end{enumerate}

\paragraph{Source field of move slot.}
The \emph{source} element has the following format:
\begin{verbatim}
<source>
    <pos> number </pos>
    <extra-bits> number </extra-bits>
    <id-pos> string </id-pos>
    <immediate width="number">
        <map extra-bits="number"> number </map>
    </immediate>
    <socket name="string">
        <map extra-bits="number" codes="string"> number </map>
    </socket>
     ...
    <bridge name="string">
        <map extra-bits="number"> number </map>
    </bridge>
     ...
    <no-operation>
        <map extra-bits="number"> number </map>
    </no-operation>
</source>
\end{verbatim}

The meaning of the elements declared within the \emph{source} element is the
following:

\begin{enumerate}
\item%
  The \emph{pos} element defines the relative position of the source
  field within the move slot.
\item%
  The \emph{extra-bits} element defines the number of extra (zero)
  bits reserved for the source field. In normal case the value should
  be '0'.
\item%
  The \emph{id-pos} element defines whether the socket, bridge,
  immediate and NOP IDs are left or right aligned. If they are left
  aligned, socket codes are on the right side of them. Respectively, if
  they are right aligned, socket codes are on the left side of them. The
  element may have two values: \emph{left} or \emph{right}.
\item%
  The optional \emph{immediate} element specifies the bit pattern that
  indicates the source field contains an in-line immediate. The width
  of the in-line immediate is given in the \emph{width} attribute.
\item%
  The \emph{socket} element declares a socket address, that is, a bit
  pattern that identifies the source socket. The \emph{name} attribute
  gives the name of the socket. The \emph{extra-bits} attribute of the
  \emph{map} element defines the number of extra zero bits in the
  socket ID which is given in the value of the \emph{map} element.

  The socket address is not always sufficient to define the encoding of a
  transport source. Certain move source ID's include additional bits. These
  bits are carried through the source socket itself, and are necessary to
  address one of several ports attached to the socket. In turn, certain
  ports can be paired with control bits that specify a register index or an
  operation code. In all these cases, the encoding is specified in a socket
  opcode table (Section~\ref{ssec:map-ports}), which is referred to by the
  optional attribute \emph{codes}. In practise, the encodings defined
  by the referred socket code table appear in the left or right side
  of the socket ID. This depends on the alignment of socket ID. That
  is, the bits that are not taken by the socket ID express, depending
  of the type of the source, port address, register index and operation 
  code.

\item%
  The \emph{bridge} element declares a bridge address for the bridge
  given in the \emph{name} attribute. The \emph{extra-bits} attribute
  of the \emph{map} element defines the number of extra zero bits in
  the bridge encoding.

\item%
  The \emph{no-operation} element specifies the bit pattern that
  indicates there is no data move on the bus.

\end{enumerate}

\paragraph{Destination field of move slot.}
The \emph{destination} element has a format similar to the \emph{source}
element, except that it does not contain \emph{immediate},
and \emph{bridge} elements.

\subsubsection{Immediate Slot}
\label{ssec:immediate-slot}

Immediate slot declaration describes a dedicated immediate field in
the TTA instruction. The field is used to carry bits of a long
immediate. Immediate slot declaration has the format as follows:

\begin{verbatim}
<immediate-slot name="string">
  <pos> number </pos>
  <width> number </width>
</immediate-slot>
\end{verbatim}

The \emph{immediate-slot} element has a mandatory \emph{name}
attribute which defines the name of the immediate slot programmed by
this field. \emph{Immediate-slot} element contains the following
elements:

\begin{enumerate}
\item The \emph{pos} element gives the relative position of the field within
      the instruction.
\item The \emph{width} element gives the width of the field.
\end{enumerate}

\subsection{Implementation Description File}
\label{subsec:IDF}

Certain building blocks of the target architecture may have more than one
hardware description file, each defining a different but functionally
equivalent hardware architecture.
%
The Implementation Description File (IDF) declares which hardware
architecture should be used for those building blocks.

For certain building blocks, in addition to a HDL source file, there may be
one or more synthesised versions.
%
IDF can define which synthesized version of those blocks should be
used. If no implementation is defined for a building block, an error
message is given.

IDF is an XML file that has the following syntax:
%
\begin{verbatim}
<adf-implementation>
  
  <ic-decoder-plugin> 
    <name> string </name>
    <file> string </file>
    <hdb-file> string </hdb-file>
  </ic-decoder-plugin>
  
  <decompressor-file> string </decompressor-file>
  
  <fu name="string">
    <hdb-file> string </hdb-file>
    <fu-id> number </fu-id>
  </fu>
   ...
  <rf name="string">
    <hdb-file> string </hdb-file>
    <rf-id> number </rf-id>
  </rf>
   ...
  <iu name="string">
    <hdb-file> string </hdb-file>
    <rf-id> number </rf-id>
  </iu>
   ...
  <bus name="string">
    <hdb-file> string </hdb-file>
    <bus-id> number </bus-id>
  </bus>
   ...
  <socket name="string">
    <hdb-file> string </hdb-file>
    <socket-id> number </socket-id>
  </socket>
   ...
   
</adf-implementation>
\end{verbatim}

The following elements are contained in \emph{adf-implementation}
element:

\begin{enumerate}
\item%
  The optional \emph{ic-decoder-plugin} element defines the IC/Decoder
  generator plugin to be used when generating the processor. 
  
  Name, file, and HDB file of the plugin may be entered in
  this element. In case plugin file name is not an absolute path, the 
  plugin file is searched in the paths described in 
  Section~\ref{sec:plugin-generators}.

\item%
 \emph{decompressor-file} can be used to enter information of the
 location of the GCU decompressor HDL file.
  
\item%
  The \emph{fu} element defines the implementation to choose from
  HDB for the FU identified by the \emph{name} attribute. It contains
  the following elements:
  \begin{enumerate}
  \item%
    The \emph{hdb-file} element gives the database file to get the
    implementation from.
  \item%
    The \emph{fu-id} element gives the ID of the \emph{fu} entry in the
    database.
  \end{enumerate}
\item%
  The \emph{rf} element defines the implementation to choose from
  HDB for the RF identified by the \emph{name} attribute. It contains
  elements corresponding to the elements contained by \emph{fu}
  element.
\item%
  The \emph{iu} element defines the implementation to choose from
  HDB for the IU identified by the \emph{name} attribute. It contains
  elements corresponding to the elements contained by \emph{fu}
  element.
\item%
  The \emph{bus} element defines the implementation to choose from
  HDB for the bus identified by the \emph{name} attribute. It contains
  elements corresponding to the elements contained by \emph{bus}
  element.
\item%
  The \emph{socket} element defines the implementation to choose from
  HDB for the socket identified by the \emph{name} attribute. It contains
  elements corresponding to the elements contained by \emph{socket}
  element.
  
\end{enumerate}

The \emph{fu}, \emph{rf}, \emph{iu}, \emph{bus} and \emph{socket} elements are
repeated for each FU, RF, IU, bus and socket in the ADF, respectively. If the
element is missing for some FU, RF or IU in the ADF, an error message is given
by ProGe and the process is aborted.



\chapter{OTHER PRODUCT REQUIREMENTS}

\section{Applicable Standards}

%% Remove this section (not needed) if no extra standards in addition to
%% those specified in the main functional specification document of the
%% project.

The common subset of the programming interface to dynamic linking loader
provided by Sun Solaris and Linux/GNU operating systems will be used.

The format of all data files used or created by the processor generator and
its helper applications is based on the XML standard \cite{XMLSpec}.



\chapter{DOCUMENTATION}

\section{User Manual}

%% Remove this section (not needed) for functional specifications of
%% modules or other parts of a larger system that don't have user manual.

\section{Online Help}

%% Remove this section if not applicable. Only applications and, as general
%% description, larger subsystems of a project have online help.

\section{Installation and Configuration Guide}

%% Remove this section (not needed) when the functional specifications refer
%% to a module or application of a larger project.  May apply to some
%% complex subsystems of a larger project when installation.

%   Detailed instructions to install a complete product (and possibly,
%   separately, complex subsystems).

\section{System Reference and Developer's Guide}

%  Usually, application programming interface (API) hypertext reference
%  generated by Doxygen. Sometimes also a separate manual (or part of
%  another manual).

%  Usually (always?) must refer to a companion document of these functional
%  specifications, less formal but more extensive in depth, about
%  nonfunctional specifications, what we call "design notes".



\chapter{KNOWN PROBLEMS AND RISKS}



\chapter{REJECTED OR ABANDONED IDEAS}

\begin{description}
\item[11.06.2004 --- Customizable GCU template.]%
  GCU could be a templated HDL file that only implements certain interface
  (port list), that is instantiated into a custom GCU case by case. However,
  GCU implementation is so configuration specific that it makes no sense to
  try to come up with generic implementations where the certain parts are
  parametrized.
\item[11.06.2004 --- Creating GCU from building blocks.]%
  Idea to provide building blocks, that is, some common operations and HW
  blocks for implementing the GCU might not be easy because the GCU is too
  complex a structure. For example, the decoder in case of compressed
  instructions can be very complex and may be in different places.
\item[25.11.2004 --- Separated IC and GCU plugin s.]%
  The idea of separate IC and GCU generator plugin s is problematic because
  there is no simple and clear mechanism to define when two plugin s generate
  IC and GCU blocks that are 100\% compatible with each other.
\item[17.01.2005 --- Reference to source ADF file in BEM file.]%
  There are two reasons why it is better to avoid a reference to ADF (file
  name and, in more sophisticated variant, fingerprint or CRC code):
  \begin{enumerate}
  \item%
    In principle, the BEM file is not uniquely related to a single ADF.
    Several target architectures may be compatible with the same binary
    encoding map. For example, two architectures differing only by the type
    of sign extension or the bit width of a transport bus are identical from
    the point of view of instruction encoding.
  \item%
    The configuration file already captures the relation BEM-ADF at a higher
    level.
  \end{enumerate}
\end{description}



\chapter{IDEAS FOR FURTHER DEVELOPMENT}

% Ideas that are not part of the specifications yet but that might be
% required in future are listed here. This is to "leave room" in design for
% later extensions and to have ideas written down somewhere.  This list
% should contain the date of addition, the idea described briefly and the
% inventor of the idea.

\begin{description}
\item[14.01.2005 --- Control codes for register indices]%
  Provide a map for encoding of GPR and immediate unit indices. There could
  be a real need for other types of mapping, for example in multi-ported
  register files implemented by means of interleaved register banks.

  If indices are encoded using a one-to-one relation, the index map can be
  omitted: a register with index `0' (whichever its RF) is encoded with a
  bit pattern 0\ldots000, a register with index `1' with bit pattern
  0\ldots001, and so on.
%
  ---A.~Cilio
\item[12.02.2005 --- Hardware Block Repertoire ]%
  Treat HBL's in different paths with the same name as containers of part of
  a common ``pool of hardware blocks'', or repertoire. The repertoire is
  identified by the name of the root directories of the HBL's that
  contribute to the pool. All HBL's of the pool are searched for a given
  block until it is found.

  To make this extension more flexible, a third search path:
  \begin{quote}
    \emph{ROOT}/share/tce/hbl/custom
  \end{quote}
  should be introduced, to distinguish those blocks (and HBL's) that are not
  in the distribution but are meant to be accessible for any toolset user.
%
  ---A.~Cilio
\item[09.03.2005 --- Converter from ADF to block-info \& implementation-info]%
  An application which converts FU's and RF's defined in ADF to
  block-info and implementation-info files would be useful when adding
  blocks to HBL with HBLmanager.
%
  ---L.~Laasonen
\item[11.03.2005 --- Option for ``Fuzzy'' Block Selection]%
  Provide an option that forces ProGe, when HBL does not contain an exact
  match for an FU defined in ADF, to search for an alternative FU with
  compatible architecture. Blocks with compatible architectures could, for
  example, implement additional operations, or have pipelines with same
  latency but less hardware constraints.
%
  ---L.~Laasonen
\end{description}



\chapter{PENDING ISSUES}

% Pending issues concerning these specifications, a sort of TO DO list.
% This chapter should be empty when the final product is ready.

% ------------------------------------------------------------------------

%  References are generated with BibTeX from a bibtex file.
\bibliographystyle{alpha}
\cleardoublepage
%% Equivalent to a chapter in the table of contents
\addcontentsline{toc}{chapter}{BIBLIOGRAPHY}
\bibliography{Bibliography}


\end{document}


%%% Local Variables:
%%% mode: latex
%%% mode: auto-fill
%%% TeX-master: t
%%% End:
