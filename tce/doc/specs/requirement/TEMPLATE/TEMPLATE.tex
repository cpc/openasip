\documentclass[a4paper,twoside]{tce}
\usepackage{pslatex}

\begin{document}
\author{[author's name]}
\title{[name of the requirements subject]}
\ver{0.1}
\firstday{[creation date]}
\lastday{[last modified]}
% id number in S- sequence
\docnum{\#\#\#}
% draft/complete/committed
\state{draft}
\maketitle


\chapter*{Version History}

\begin{HistoryTable}

 0.1    & dd.mm.yyy & [A. Name]   &   [what was added]\\
 ver    & dd.mm.yyy & [A. Name]   &   [what was added]\\

\end{HistoryTable}


\tableofcontents



\chapter{INTRODUCTION}

\section{Purpose}

%  Describe the purpose of the document - example:
%  This document describes the project "TTA Codesign Environment".  Its
%  purpose is to give an overview of the proposed TTA Codesign Environment,
%  its main features and requirements, and a summary of the project
%  motivations and goals.
\section{Product Idea}

% (PJ: product)

%  Describe the project motivation in few words and the context in which
%  this product should have a reason to exist - see TCE Project Plan for an
%  example
\section{Product Overview}

% (PJ: product + overview)

% This section describes as briefly as possible the system that implements
% the functionality wanted.

%Only main features/requirements of the product, no details.
\section{Definitions}

\begin{description}
\item[term]%
  term definition
\end{description}

\section{Acronyms and Abbreviations}

\begin{longtable}{p{0.15\textwidth}p{0.85\textwidth}}
TCE   & TTA Codesign Environment. \\
TTA   & Transport Triggered Architectures. \\
\end{longtable}
\section{Document Overview}
% (PJ: summary)



\chapter{USER DESCRIPTION}

\section{User Demographics and Market Target}

%% Remove this section (not needed) for functional specifications of
%% modules or applications of a larger project.

%  Where is this product going to be used? how many users work on a single
%  installation?
\section{User Profile}

%% Remove this section (not needed) for functional specifications of
%% modules; may be needed for applications or subsystems.

%  What's the user like? Student? Researcher? Professional? Expertise in
%  hardware or software expert? How much skilled?

% PEKKA:
% Describe who are the or users of the module/application.
% The users can be "a software engineer" or "a researcher that is developing
% new scheduling algorithms".
\section{User Environment}

%% Remove this section (redundant, not needed) for functional specifications
%% modules or applications that are part of a larger project.

%  Physical place where the product is used - example from TCE main
%  functional specifications:
%  The typical environments in which TCE is expected to be used is academic
%  research laboratories and offices, but any workplace with a computer is
%  suitable. TCE is expected to be run on workstations as well as personal
%  computers, and possibly installed over a network file system.
\section{Key User Needs}

%% Remove this section if the description in INTRODUCTION/Product Idea
%% is sufficient (usually it is for functional specifications of modules)

%  Important section, it's like a condensed way to formulate
%  requirements as user desiderata

%  Extract from the TCE Project Plan document as example:

%  Expertise in a package. The toolset is designed to allow users that do
%  not possess in-depth expertise in code generation for embedded processors
%  or in processor design to successfully carry out application and embedded
%  processor codesign.  Users can concentrate their effort on application
%  development and evaluation of architecture trade-offs.
%  [...]
%  Design of application-specific custom hardware.  TCE supports definition
%  of user-defined instructions and relative hardware at all levels of the
%  codesign process: code generation, simulation, design space exploration
%  and processor generation.
\section{Alternatives and Competition}

%% Avoid this section if alternatives have not been considered at all.
%% 'Competition' applies only to functional specifications of bigger
%% subsystems of a project.

%  Describe alternatives and motivate why it's better to do it in-house.
%  Alternatives will be whole products in case of TCE requirements document,
%  libraries and off-the-shelf software tools in case of modules and
%  subsystems.



\chapter{PRODUCT OVERVIEW}

\section{Product Perspective}

% (external connections)

%  High-level description of the product with focus on its communication
%  with external products/modules. What external modules does it depend
%  from?  What is provided in the product and what is required and assumed
%  available as external service?

% PEKKA:
% This section describes the environment where the product belongs to, for
% example what the module is part of. Also client relationships to other
% modules or applications are described in detail.
%
% Example:
%

% This module is part of the software subsystem of the TCE. Program Model
% constructs itself using this module. Applications that inspect or modify
% binary files (such as the linker) are direct clients of the module.
%
% This module uses the binary stream module to load and store binary files
% from/to disk. Uses Reference Manager to handle references.
%
% This module is used by different binary file handling tools of the
% project. Example of a such tool is TPEFDumper \cite{tpefdumper-fspecs},
% which prints useful information on the binary files.

\section{Product Position Statement}

% (goals)

%% Remove this section if the description in INTRODUCTION/Product Idea
%% is sufficient (it should be so for functional specifications of modules)

%  In few words, almost a slogan, the reason for this product to exist

\section{Summary of Capabilities}

% (advantages)

%  Use table format and possibly extra comments if possible.

%  Example extracted from TCE Project Plan:

\begin{center}
\begin{longtable}{p{0.45\textwidth}p{0.55\textwidth}}
  User Benefit                 &       Supporting Feature \\
% productive application programming & high-level language compiler
% high-performance code generation   & TTA instruction scheduler
% accurate application analysis      & profiling tools, simulation feedback
% application code fine-tuning       & options controlling code generation
\end{longtable}
\end{center}

\section{Assumptions and Dependencies}

% (portability)
% (PJ: design constaints)

%  Assumed environment, third-party applications or library assumed to be
%  present. External services (like OS services) that are required.

% PEKKA:
% For example "the GUI should use wxWindows GUI library to ease portability
% to other operating systems and environments" or "XML should be used for
% all clear text persistent data storage".

%  In case of functional specifications of most modules, it's sufficient to
%  specify what is required in addition to assumption/dependencies given in
%  main functional specifications document of TCE Project, and refer to it
%  for the rest.

\section{Distribution}

% (IPR and copyrights)

%%  Remove this section (not needed) for most modules and applications,
%%  because the conditions of the whole project apply.



\chapter{PRODUCT FEATURES}

% (PJ:functionality)

% PEKKA:

% This chapter describes the functionality of the product in detail. In case
% modules, the required client interfaces are defined in broad manner. These
% are the minimum interfaces the object should provide for the client. The
% format or argument list of methods should not be in detailed level, that's
% a thing that should be in the design notes instead.

% In case of an application, all command line arguments and user interface
% screens (GUI) are described in detail.  Also, all tasks the module or tool
% must be able to accomplish are described here.

% Example:
%
% This module must be able to read a.out and TPEF completely. No other
% binary reading support is required at this point. Only writing of TPEF
% format is required.

%  Organise in sections.  Dedicate one section for each main feature.
%  For example: program input/output modules (what it's usually called
%  Binary Handling Module) could be divided in four parts:
%  - common features
%  - reader features
%  - writer features
%  - reference managment
%% Some modules may be so simple that a single section is sufficient.



\chapter{DATABASES}

%% Remove this whole chapter (not needed) for functional specifications of
%% applications that do not define relevant data structures and that use
%% only data from library modules.

% PEKKA:
% This chapter describes the data that is processed and used by the module
% or application in broad but still complete way.  More detailed
% descriptions like database tables or XML field names should be in the
% design document.

\section{Contents of Information}

% (PJ:data)

% PEKKA:
%
% This section describes the data types defined and used by the module and
% their relations.  The description is at a high-level, for example using
% UML class diagrams that reflect the real world data, not the classes of
% the target program.

% Example:
%
% Binary(1)--(*)Section <--TextSection(1)--(*)Instruction(1)--(1..*)InstrPart
%                       <--DataSection
%                       <--RelocationSection
%
% For exact format of the data in section, see TPEF specifications \cite{}.

\section{Intensity of Use}

\section{Files and Configuration}

% (PJ:I/O + format)

% PEKKA:

% I/O of the module or application. Examples:
%
% Input is a file that describes instructions and their behavior.
%
% Outputs information to stdout.
%
% The reader of the TPEF binary handling module takes a binary stream as
% input and outputs a binary object model. The writer takes a binary object
% model as input and outputs a TPEF format binary file to binary stream.



\chapter{OTHER PRODUCT REQUIREMENTS}

\section{Applicable Standards}

%% Remove this section (not needed) if no extra standards in addition to
%% those specified in the main functional specification document of the
%% project.

\section{System Requirements}

%% Remove this section (not needed) for most modules of a larger project.
%% May be required in some critical applications if requirements are
%% significantly different from those of the rest of the system.

%  Memory space
%  Disk space
%  Processor Performance

\section{Performance Requirements}

%% Remove this section (redundant) for most modules and applications of a
%% larger project.

%  response times to user

\section{Environment Requirements}

%% Remove this section (not needed) for functional specifications of
%% modules or applications of a larger project.

%  Requirements in terms of computer environment, administrator rights for
%  installation, level of expertise - only for complete products.

\section{Security, Recovery, Usability}

%% Remove this section (not needed) for functional specifications of
%% modules or applications of a larger project.



\chapter{DOCUMENTATION}

\section{User Manual}

%% Remove this section (not needed) for functional specifications of
%% modules or other parts of a larger system that don't have user manual.

\section{Online Help}

%% Remove this section if not applicable. Only applications and, as general
%% description, larger subsystems of a project have online help.

\section{Installation and Configuration Guide}

%% Remove this section (not needed) when the functional specifications refer
%% to a module or application of a larger project.  May apply to some
%% complex subsystems of a larger project when installation.

%   Detailed instructions to install a complete product (and possibly,
%   separatedly, complex subsystems).

\section{System Reference and Developer's Guide}

%  Usually, application programming interface (API) hypertext reference
%  generated by Doxygen. Sometimes also a separate manual (or part of
%  another manual).

%  Usually (always?) must refer to a companion document of these functional
%  specifications, less formal but more extensive in depth, about
%  nonfunctional specifications, what we call "design notes".



\chapter{KNOWN PROBLEMS AND RISKS}



\chapter{REJECTED OR ABANDONED IDEAS}

% PEKKA:

% The abandoned requirements should be listed here with the reasoning and
% date the requirement was dumped. This is for future reference.
%
%11.11.2003 HTTP connection to external server to fetch something. 
%           This is simply not sensible and should be forgotten.



\chapter{IDEAS FOR FURTHER DEVELOPMENT}

% Ideas that are not part of the specifications yet but that might be
% required in future are listed here. This is to "leave room" in design for
% later extensions and to have ideas written down somewhere.  This list
% should contain the date of addition, the idea described briefly and the
% inventor of the idea.

% Example:
%
% 11.11.2003 Possibility to also write the legacy a.out format to maintain
%            compatibility to old toolset. -P.J��skel�inen



\chapter{PENDING ISSUES}

% Pending issues concerning these specifications, a sort of TODO list.
% This chapter should be empty when the final product is ready.


% ------------------------------------------------------------------------

%% Remove this part if there are no references.  Usually there will be at
%% least a reference to the functional specifications of the TCE Project
%% (S-001).

%  References are generated with BibTeX from a bibtex file.
\bibliographystyle{alpha}
\cleardoublepage
%% Equivalent to a chapter in the table of contents
\addcontentsline{toc}{chapter}{BIBLIOGRAPHY}
%\bibliography{Bibliography}


\end{document}


%%% Local Variables: 
%%% mode: latex
%%% TeX-master: t
%%% End: 
