\documentclass[twoside]{tce}

\usepackage{pslatex}

\begin{document}
\author{Andrea Cilio}
\title{TCE Architecture Template Programming Interface}
\ver{0.14.1}
\firstday{07.01.2004}
\lastday{26.10.2006}
% id number in S- sequence
\docnum{010}
% draft/complete/committed
\state{draft}
\maketitle

% highlighted style for parameters/variables or otherwise non-fixed parts
% of a command syntax or of lines of text displayed by the application
\newcommand{\parm}[1]{\textsl{#1\/}}

\chapter*{Version History}

\begin{HistoryTable}

 0.1    & 11.06.2004 & A. Cilio   &
 Introduction, 1. First fragments, 2, 3.\\
 0.1.1  & 21.10.2004 & L. Laasonen &
 Changed name of MDF to ADF. \\
 0.2    & 10.01.2005 & A. Cilio   &
 Extended syntax  of source and destinations of moves.\\
 0.3    & 23.03.2005 & A. Cilio &
 Added assembly language syntax, first draft.\\
 0.4    & 24.03.2005 & A. Cilio &
 Completed specification of assembly language syntax.\\
 0.5    & 04.04.2005 & A. Cilio &
 Revised syntax of bridge terms and data initialisation. Added
 disambiguation rules.\\
 0.6    & 11.04.2005 & A. Cilio &
 Added code conventions.\\
 0.7    & 03.08.2005 & A. Cilio &
 Added assembly directives. Made grammar free-form.\\
 0.8    & 17.10.2005 & A. Cilio &
 Updated grammar to actual implementation, restyle.\\
 0.9    & 25.10.2005 & A. Cilio &
 Interim revision. Review by M.~Lepist� grammar.\\
 0.10   & 28.10.2005 & A. Cilio &
 Major revision. Completed changes to grammar. Many corrections.\\
 0.11   & 28.10.2005 & A. Cilio &
 Assembler syntax independent of whitespace.\\
 0.11.1 & 26.01.2006 & A. Cilio &
 Made issues on assembly syntax clearer.\\
 0.12   & 14.03.2006 & M. Lepist\"o &
 Moved section about GPR and IU naming from pendig issues to \
 disambiguation rules. Fixed long immediate syntax.\\
 0.13   & 30.05.2006 & P. J\"a\"askel\"anen &
 Revised the Universal Machine chapter. \\
 0.14   & 01.06.2006 & M. Lepist\"o &
 Added information about address spaces of universal machine. \\
 0.14.1   & 26.10.2006 & H. Kultala &
 Corrected information SP and RV register indeces \\
 0.14.2   & 08.12.2006 & P. J\"a\"askel\"anen &
 The load/store operations are now bound to MAU multiples. Do not mention
 'word' as its meaning is ambiguous. \\

\end{HistoryTable}

% Table of contents
\tableofcontents

% Document text
\chapter{INTRODUCTION}

\section{Purpose}

This document describes the programming interface of the TTA processor
architecture template used by TCE.  These specifications do not include
anything that is already specified by the ADF specification
document~\cite{ADF-specs}.

Architecture programming interface specifications consist of architectural
conventions concerning the layout of call stack, registers, the format of
disassembled instructions and so on.  This document provides a reference for
any application of the TCE toolset that depends on such conventions.

A special group of conventions applies to the restrictions of (completely)
unscheduled TTA code.

All TCE toolset applications should comply with the conventions described in
this document.  When a certain specification is not a strict prescription
but rather a suggestion, this is clearly noted.

\section{Product Idea}

The TTA processor architecture template is \emph{the} fundamental entity of
TCE.  The template specifications given in~\cite{ADF-specs} define
components and parameters of the template.  This, however, is not sufficient
to define a usable architecture.  In addition to its structure, an
architecture is also characterised by a number of \emph{conventions}
concerning the use of the hardware resources.

\section{Acronyms and Abbreviations}

\begin{center}
\begin{longtable}{p{0.10\textwidth}p{0.85\textwidth}}
ABI   & Application Binary Interface.\\
FU    & Function Unit.\\
GCU   & Global Control Unit.\\
IU    & Immediate Unit.\\
MAU   & Minimum Addressable Unit.\\
NW    & Natural Word.\\
TCE   & TTA Codesign Environment. \\
TTA   & Transport Triggered Architectures. \\
SPR   & Stack Pointer Register. \\
PC    & Program Counter. \\
PSR   & Processor Status Register. \\
RF    & Register File.\\
UFU   & Universal Function Unit.\\
UM    & Universal Machine.\\
\end{longtable}
\end{center}


\chapter{PRODUCT OVERVIEW}

\section{Perspective}

The specifications found in this document are restricted to the TCE toolset.
Despite the fact that it defines a vast space of processor architectures,
the architectural template is nowhere near covering all the possible
processor architectures that can be defined with the principle of transport
triggering.

A near perfect subset of the architecture space defined by the TCE
architecture template and a base for its development is the Move-TTA
architecture template.

The specifications contained in this document affect practically all TCE
applications.  In particular, the architecture conventions define the
program-processor interface and restrict the type of acceptable TTA
programs.

\section{Assumptions and Dependencies}

The specifications found in this document apply to the TCE processor
architecture template defined in ADF Specification Document~\cite{ADF-specs}
and depend on it.

\chapter{APPLICATION BINARY INTERFACE}

\section{Stack Frame Layout}
\label{sec:stack-layout}

\begin{figure}[t]
  \centerline{
  \psfig{figure=eps/StackFrames.eps,width=9.75cm}
  }
  \caption{Stack Layout.}
  \label{fig:stack-frame}
\end{figure}

The stack of a program is divided up into blocks of contiguous memory called
\emph{frames}. Each frame is activated upon entering a function and is
destroyed (its storage is made available for new frames) when that function
invocation returns. The stack frame of a function contains the arguments
with which the function is called and variables local to the procedure.

The stack pointer register (SPR) always points to the the first word past
the area of the current frame that is dedicated to local variables.  The
area of the stack past the local variables area contains the outgoing
arguments of called functions (if any).  The stack of TTA's supported by TCE
grows downwards, therefore the incoming arguments and the local variables on
the local stack frame have positive offsets when accessed from the current
function, whereas the outgoing arguments have negative stack offsets.

Figure~\ref{fig:stack-frame} depicts the memory layout of part of a TTA
stack that contains two frames: A and B.  When function A frame is
activated, SPR points to the first word past the top of the local variable
area of the caller frame (this location is shown as ``SPR'' in figure).  At
function entry and before accessing the stack, function A initialises SPR
with the top of its frame (shown as ``SPR-A'' in figure).  When function A
prepares a call to function B, it sets up the outgoing arguments in the
stack area past its local variable area.  At function entry, function B
initialises SPR with the first word past the top of its local variable area
(shown as ``SPR-B'' in figure).  In case function B does not use any stack
for passing outgoing arguments, the location SPR-B corresponds to one word
past the top of the frame.

[[to do]] the local variable area may contain: addressed local variables,
local data structures that cannot be allocated in registers, values of
registers that are spilled to memory.

\section{Universal Machine: a Target for Unscheduled TTA Code}
\label{sec:umachine}

The unscheduled TTA code generated by front-end compilers can be considered
as valid code for a particular target TTA, called \emph{Universal Machine}.
The Universal Machine (UM) consists of components (hardware resources) much
like real target TTA's.  However, the meaning of an assignment to these
components is peculiar.  In certain contexts (for example, during
instruction scheduling), an element of a the TTA (data transport, source,
destination, etc.) assigned to a component of UM may be considered as not
assigned at all.  In this case, it is possible to ignore violations of what
usually are hardware constraints.

\subsection{Structure of Universal Machine}

\begin{table}[tb]
\begin{tabular}{|l|l|c|}
  \hline
  operation type & mnemonic  & optional\\
  \hline
  integer add and subtract    &
    \texttt{add}, \texttt{sub} & no \\
  load/store 4 minimum addressable units (MAU) at once & 
    \texttt{ldw}, \texttt{stw} & no \\
  integer compare             &
    \texttt{eq}, \texttt{gt}, \texttt{gtu} & no \\
  shift                       &
    \texttt{shl}, \texttt{shr}, \texttt{shru} & no \\
  logical                     &
    \texttt{and}, \texttt{ior}, \texttt{xor} & no \\
  absolute jump       &
    \texttt{jump} & no \\
  logical negation            &
    \texttt{not}  & yes \\
  arithmetic negation         &
    \texttt{neg} & yes \\
  procedure call &
    \texttt{call} & yes \\
  integer multiply and divide &
    \texttt{mul}, \texttt{div}, \texttt{divu}, \texttt{mod},
    \texttt{modu} & yes \\
  load/store 1 (q)/2 (h)/8 (d) minimum addressable units at once
     &
    \texttt{ldq}, \texttt{ldqu}, \texttt{ldh}, \texttt{ldhu}, \texttt{stq},
    \texttt{sth}, \texttt{ldd}, \texttt{std} & yes \\
  sign-extend                 &
    \texttt{sxbw}, \texttt{sxhw} & yes \\
  floating-point arithmetic   &
    \texttt{addf}, \texttt{subf}, \texttt{negf}, \texttt{mulf},
    \texttt{divf}  & yes \\
  floating-point compare      &
    \texttt{eqf}, \texttt{gtf} & yes \\
  type conversion             &
    \texttt{cfi}, \texttt{cif}, \texttt{cfd}, \texttt{cdf} & yes \\
  minimum and maximum         &
    \texttt{min}, \texttt{max}, \texttt{minu}, \texttt{maxu} & yes \\
  bit rotation                &
    \texttt{ror}, \texttt{rol} & yes \\
  \hline
\end{tabular}
\caption{The operation repertoire of the Universal Machine (The Base
Operation Set).}
\label{tab:op-repertoire}
\end{table}

The UM consists of the following components:
\begin{enumerate}
\item %
  One transport bus.  The bus is wide enough to carry any data transport
  that can occur in a program. Currently, the maximum width is the one
  of a double precision floating point value, that is, 64 bits.
\item %
  One function unit, the Universal Function Unit (UFU).
  This unit implements any operation that occurs in a program, all with
  latency one cycle.  The results of any operation is available one cycle
  after the last of its operands has been moved to UFU.  The predefined
  operation set supported by UFU is a subset of the operations listed in
  Table~\ref{tab:op-repertoire}.  In addition to this set, UFU can support
  any user-defined operation. The operand width of any user-defined
  operation is defined to be 64-bit (a double-precision float) so the
  operation is able to implement floating point operations. The operand
  width of base-operations except for the floating point operations is
  32-bits. All floating point operations in universal machine are assumed
  to be double-precision floats (64-bits).
\item %
  One control unit, that implements the supported control flow operations,
  currently 'CALL' and 'JUMP'.
\item %
  One unbounded integer register file (IRF) for integer GPR's.
\item %
  One unbounded floating-point register file (FRF) for floating-point GPR's.
\item %
  A Boolean register file (BRF) of size and width of 1 for the Boolean
  register used in guard expressions of sequential code.
\item %
  No immediate units. All immediates are defined as inline immediates
  (see the next item).
\item %
  The source field of the move slot is wide enough to carry the widest
  immediate that can occur (see Section~\ref{ssec:data-types} for details on
  the data sizes of UM components).
\item %
  The transport bus supports a simple guard expression consisting of
  invert/noninvert flag and a term that can be either the only register of
  the Boolean register file or the constant zero.  The constant zero is
  necessary to generate unconditional (move always) data transports.
\item %
  The number of UFU ports, their bit width and their binding to operation
  input/outputs is not hard-coded.  It depends on the subset of the operation
  repertoire that appears in the program and the implementation of the
  client applications that build UM.\footnote{
%
    In practice, the UFU ports have no significance, since there is no
    possibility of ``resource'' conflict.  Client applications must refer to
    operation inputs and outputs instead.}
\end{enumerate}

\subsection{Data Types}
\label{ssec:data-types}

The UM supports a restricted set of word sizes:
\begin{enumerate}
\item Integer: 32 bits.
\item Floating-point: 64 bits (IEEE-754 format, double precision).
\item Boolean: 1 bit.
\end{enumerate}

The natural word width (NW) is 32 bits.  Consequently, the sizes of the
words handled by load and store operations are:
\begin{enumerate}
\item For quarter-NW operations \emph{ldq}, \emph{ldqu}, \emph{stq}: 8 bits.
\item For half-NW operations \emph{ldh}, \emph{ldhu}, \emph{sth}: 16 bits.
\item For natural word operations \emph{ldw}, \emph{stw}: 32 bits.
\item For double word operations \emph{ldd}, \emph{std}: 64 bits.
\end{enumerate}

The size of pointer is equal to the size of integer words, 32 bits.

\section{Unscheduled TTA Code Conventions}
\label{sec:um-conventions}

The unscheduled code generated by TCE front-end compilers is expected to
satisfy a number of constraints.  These constraints are meant to make
analysis and transformation of sequential code easier.

\subsection{No Software Bypasses}

Any move that has a FU input as destination must have an (inline) immediate,
the return address register (RA) or a GPR as source. Any move that has a FU
output as source must have a GPR as destination.

\subsection{Restrictions to Instruction Addressing}

Control transfer moves (jumps and calls) can only have two types of source:
\begin{enumerate}
\item %
  Constant, absolute addresses (which are inline immediates).
\item %
  Indirect address (a GPR).
\end{enumerate}

PC relative, page relative or other types of instruction addressing mode are
not currently supported.

\subsection{Restrictions to Guarding}

Only jump moves (not calls!) can be guarded.

\subsection{Predefined Calling Conventions}
\label{ssec:sequential-calling}

The calling conventions described in Section~\ref{ssec:calling-conventions}
apply exactly to unscheduled code.  However, the register aliases reserved
for passing parameters across function calls are fixed.  Assuming that the
integer and the floating-point unbounded register files are called,
respectively `R' and `F', the assignments are as follows:
\begin{center}
\begin{tabular}{|p{0.35\textwidth}@{\hspace{\fill}}|p{0.35\textwidth}|}
\hline
  aliases               & assigned registers \\
\hline
  \texttt{ires0}                     & \texttt{R0} \\
  \texttt{fres0}                     & \texttt{F0} \\
  \texttt{iarg1}\ldots\texttt{iarg4} & \texttt{R2}\ldots\texttt{R5} \\
  \texttt{farg1}\ldots\texttt{farg4} & \texttt{F1}\ldots\texttt{F4} \\
\hline
\end{tabular}
\end{center}

\subsection{Predefined Register Conventions}
\label{ssec:sequential-reg-save-restore}

The register assigned to the stack pointer register (alias `sp') is register 
number 1 of the integer register file.

All registers that need to be preserved across a function call are stored on
the stack by the program.  Registers with index 0 and 2 to 5 
(reserved to passing and returning function parameters) and 6 to 511 are 
caller-saved.  Registers with index 512 on are callee-saved.  See
Section~\ref{ssec:register-convention} for an accurate descriptions of
register conventions.

\subsection{Address Spaces}
\label{ssec:address-spaces-of-umach}

Universal machine contains two address spaces: one for data, and one 
for instructions. The instruction address space must be named as 
"universal\_instructions". The data address space MAU size is 8 bits
and the name of the data address space is "universal\_data".

\section{General TTA Code Conventions}
\label{sec:code-conventions}

\subsection{Alignment of Memory Words}
\label{ssec:align-conventions}

The TCE-TTA template restricts the addresses of data word to multiples of a
certain size.  The memory alignment is always equal to the size (in MAU's)
of the (sub)word being addressed.  For example, if a Floating-point word
takes 8 MAU's, its address must be a multiple of 8.

Applications that process operations that access memory should at least warn
about the presence of mis-aligned memory addresses.

The effect of mis-aligned addresses on processor implementations are
undefined. Typically: (1) the processor accesses the nearest (lower) aligned
address instead; (2) the processor halts or rises an exception. Less likely
but also possible, a processor could enter a slower operation mode and
perform a mis-aligned memory access.

\subsection{Word Order}
\label{ssec:word-order}

The TCE-TTA targets may order MAU's of words stored in memory in either of
the usual conventions:
\begin{enumerate}
\item %
  Most significant MAU in word start address (``big endian'' word order).
\item %
  Least significant MAU in word start address (``little endian'' word
  order).
\end{enumerate}

Unusual word orders, such as ``2-MAU half-word big endian, MAU little
endian'', are not supported.

\emph{Example: Memory Layout and Word Order.}
%
The following three examples show the memory layout of a 32-bit, 4-MAU word
containing 0x11223344 in the two supported word orders and one unusual
(unsupported) word order.
\begin{quote}
  \begin{tabular*}{0.8\textwidth}{rrrr@{\extracolsep{\fill}}p{0.4\textwidth}}
  \texttt{0x11} & \texttt{0x22} & \texttt{0x33} & \texttt{0x44} &
    ``big endian'' word order \\
  \texttt{0x44} & \texttt{0x33} & \texttt{0x22} & \texttt{0x11} &
    ``little endian'' word order \\
  \texttt{0x22} & \texttt{0x11} & \texttt{0x44} & \texttt{0x33} &
    odd word order\footnotemark \\
  \end{tabular*}
\end{quote}
%
\footnotetext{Found in PDP-11.}

The word order is hard-coded during set up and cannot be changed once the TCE
toolset is installed without a fresh build.

\subsection{Calling Conventions}
\label{ssec:calling-conventions}

The calling conventions define how parameters and results are exchanged
between functions.  The calling conventions form an important part of the
Application Binary Interface (ABI).  Two TTA programs can be linked together
only if they conform to the same ABI.
%
\note{DISCUSS}
%
The ABI of TTA programs is not completely defined by these specifications.
Since certain compilers may restrict the ABI (or otherwise work much more
efficiently if a certain ABI is chosen), part of the TCE-TTA ABI is left
dependent on the front-end compiler.  The part of the ABI described in this
section is fixed and mandatory.

\begin{enumerate}
\item %
  Two registers are reserved for passing results back to a function's call
  site: the integer register can be indicated with alias `ires0', the
  floating-point register can be indicated with alias `fres0'.
\item %
  Up to 4 integer and 4 floating-point registers (aliases
  `iarg1'\ldots`iarg4', `farg1'\ldots`farg4', respectively) can be used to
  pass parameters of suitable bit width to called functions.  When not
  needed to pass arguments, argument registers can be assigned to variables
  like any other GPR.  In this case, however, it must be noted that these
  registers necessarily belong to the caller-saved convention (see
  Section~\ref{ssec:register-conventions}).
\item %
  All parameters after the 5th integer parameter of the argument list and
  the 5th floating-point parameter of the argument list are passed on the
  stack.  The alignment rules defined in
  Section~\ref{ssec:align-conventions} apply also to parameters passed on
  the stack.
\item %
  All parameters of aggregate type that do not fit in an integer GPR are
  always passed on the stack.
\end{enumerate}

The real registers corresponding to the aliases are a free choice of the
back-end tools.  As long as enough registers are reserved and are of the
right bit width, any assignment is valid.

The following aspects of the ABI are not part of the TTA program convention
and depend on the third party compiler:
\begin{enumerate}
\item %
  Passing and returning aggregate data types of the structure or record sort
  across functions.  Depending on their size, aggregates might be passed in
  GPR's or forced on the stack.
\item %
  Passing a variable list of parameters to a function (the C `\ldots'
  argument list, also called `vararg').  Depending on the front-end compiler
  implementation, parameters of a variable list may be all forced on the
  stack or may follow the conventions of normal parameters, defined above.
\item %
  Passing parameters that would fit in a GPR and that occupy positions below
  the fifth in the argument list, but which follow a parameter passed on the
  stack.  Certain compilers may still pass these parameters on a GPR of the
  appropriate type, whereas other compilers may force into the stack all the
  parameters that follow the first parameter passed on the stack.
\end{enumerate}

\subsection{Register Conventions}
\label{ssec:register-conventions}

The register conventions define how registers are used and who is
responsible to save registers that may be used across function calls.  An
important part of the register conventions is related to the mechanism for
passing argument, which is described in
Section~\ref{ssec:calling-conventions}.

One integer GPR is reserved throughout the program for a special use: it is
the stack pointer register (SPR), which is conventionally indicated by the
register alias `spr'.  This register always points to a location of the
program stack, described in Section~\ref{sec:stack-layout}.

All GPR's except for SPR may have to be saved and restored as a result of a
function call.  When a register carries a variable in the caller code that
remains data-live across a function call, and is redefined in the callee, it
needs to be saved and restored.  Registers are subdivided in two groups:
those that are saved and restored around a function call, and those that are
saved at the beginning of a function and are restored before it returns.
The first group is termed caller-saved registers, the second group
callee-saved registers.

Unlike the unscheduled code, which must rely on a fixed convention (see
Section~\ref{ssec:sequential-reg-save-restore}, parallel TTA code does not
need to prescribed a fixed number of callee- and caller-saved registers.
Depending on the properties of the program and the capacity of the register
allocator, a varying fraction of the total GPR's in a register file may be
assigned to either groups.  The register allocator may even dedicate one
register file to caller-saved registers and another to callee-saved
registers.

\chapter{REJECTED OR ABANDONED IDEAS}

\begin{description}
\item[27.10.2005] -- Alternative syntax of general purpose registers.

  If the name of register files and immediate units of a target architecture
  do not end with a decimal digit, the dot separator between the register
  file name and the register index can be omitted. In this case, the port
  specifier must be omitted, too.
%
  For example, `i0' could indicate register with index 0 on immediate unit
  `i', and `r5' register with index 5 on register file `r'.

  This alternative has been rejected because it makes conflict with label
  names more likely and because it impacts readability (two forms to express
  the same term).

  The alternative syntax could be less detrimental if its application were
  restricted to the (reserved) names of the universal machine variable
  pools: `r', `f', `b'.
\end{description}



\chapter{IDEAS FOR FURTHER DEVELOPMENT}
\label{sec:further-ideas}

% Ideas that are not part of the specifications yet but that might be
% required in future are listed here. This is to "leave room" in design for
% later extensions and to have ideas written down somewhere.  This list
% should contain the date of addition, the idea described briefly and the
% inventor of the idea.

\begin{description}
\item[23.03.2005]%
  Explicit specification of address space in literals that represent memory
  addresses. Instead of being implicitly bound to the address space of the
  FU that contains move destinations that receive addresses, the address
  could be specified explicitly. This is necessary to support multiple
  address spaces in unscheduled code. Currently, it is possible to avoid
  this by rearranging segments of a unique address space of sequential code
  into multiple address spaces of the target architecture (in a fashion
  similar to linker map files).

  The address space could be added to the address literal by prepending its
  name. For example, address `0x1FAFF' of address space `bram' would be
  specified as follows:
\begin{verbatim}
    [bram]0x1FAFF -> fu2.p3
\end{verbatim}
  Address space specifier is not necessary when the address is symbolic,
  that is, is given as a label expression, because it is implied. Labels
  refer to the address space of the memory area declaration they belong to.
%
  ---A.~Cilio
\item[23.03.2005 -- Register Specifier Prefixes.]%
  On some applications, the registers names may be prepended by a character
  to avoid clashes with the name space of other entities.  For example, in
  case architected registers and variables can be freely mixed in
  expressions of the language of a CLI-based application, such character
  could be `\$'.
%
  ---A.~Cilio

\item[24.03.2005 -- Register Aliases.]%
  For convenience, a number of registers that are reserved for special
  purpose can be referred to by an alias:
%
\begin{quote}
\begin{tabular}{lp{0.75\textwidth}}
register         & description \\\hline
\texttt{spr}     &
  stack pointer register\\
\texttt{iar\parm{n}} &
  integer argument register \emph{n}, with \emph{n} in [1,4] \\
\texttt{far\parm{n}} &
  floating-point argument register \emph{n}, with \emph{n} in [1,4] \\
\texttt{iar0} &
  integer return value register\\
\texttt{far0} &
  floating-point return value register\\
\hline
\end{tabular}
\end{quote}
%
  Register aliases make it easier to code (the mnemonic effort demanded from
  assembly programmers is lower) and makes the code more readable.  Pending
  issue is how to specify the alias declaration and whether a default
  convention can be defined (like taking the first registers or the first
  integer register file, for example).
%
  -- A.~Cilio

\item[27.10.2005] --- Forcing the character set of string literals.

  String literals are not portable. To make sure a string is interpreted
  correctly, the assembler could provide a directive to define the character
  set to used:
  \begin{quote}\tt
    :charset \parm{charset-name}
  \end{quote}

  The character set could be forced on each individual string literal. For
  example, to force a literal string to be interpreted in UTF-16 encoding:
  \begin{verbatim}
    " M. Lepist�[UTF-16]
  \end{verbatim}
%
  A.~Cilio

\item[28.10.2005] --- Support for ABI-free scheduled code.

  The TTA code generation back-end does not have, in principle, to be
  restricted to the conventions described in
  Sections~\ref{ssec:calling-conventions} and
  \ref{ssec:register-conventions}. These conventions are primarily meant to
  make sequential code produced by front-ends unambiguous and standard.
  Scheduling passes, if not followed by code transformation passes that rely
  on ABI, are free to redefine calling, register and other architectural
  conventions.

  However, in order to support ABI-free back-end passes in a more general
  way, additional support is needed. For example, ABI (when not default)
  could be declared in a special program section, or the conventional uses
  of registers and operations could be annotated.
%
  A.~Cilio
\end{description}



\chapter{PENDING ISSUES}

This chapter contains pending issues concerning this specification document,
and should be empty when the final product is ready.

\section{Ambiguity in Unscheduled Code}
\label{ch:pending:disambiguation}

The assembly syntax for TTA code with (partially) unassigned move terms is
potentially more ambiguous than in case of parallel TTA code.

Ambiguity is increased if the architecture template allows for open-ended
special register ports in the global control unit of the target
architecture. Currently, only one name is reserved for a special register
port: `rar', the return address register.

Ambiguity is increased also if the architecture template allows custom
variable pools, to be assigned to equally-named register files of the target
architecture. Currently, only three names are reserved for as many
predefined variable pools: `r', `f', and `b'.

The architectural features described cause the following ambiguous
interpretations in unscheduled TTA code:
\begin{enumerate}
\item %
  Label name and name of a special register port conflict. A move with a
  source term such as `\verb|xreg|' can represent: (1) a label called
  `xreg'; (2) an unassigned special register port called `xreg' (which will
  eventually be assigned to the GCU of the target architecture).
\item %
  A move with a source or destination term such as `\verb|xx.2|' can mean:
  (1) variable with index 2 from the custom variable pool `xx'; (2) input or
  output identified by index 2 of operation `xx'. Which of the two to
  choose? Unscheduled code comes without target architecture specification,
  so no assumption can be made as the name and type of register files or
  operations defined in the target architecture.
\end{enumerate}

The assembler could refuse to decide and signal an error condition whenever
a name conflict of the types listed above occurs, or could apply a
disambiguation rule.

If scheduled and unscheduled TTA code is mixed, a move term `\verb|xx.2|' is
highly ambiguous, as it can be: (1) a variable in a custom variable pool; (2)
a GPR in a register file; (3) an unassigned special register port; (4) an
unassigned operation input or output.
%
In this case, the parser is greatly helped by the existence of a target
architecture definition, because it can test:
\begin{enumerate}
\item %
  Whether there exists a GPR in the target architecture with name `xx'.
\item %
  Whether there exists a FU in the target architecture that supports an
  operation called `xx'.
\item %
  Whether there exists a special register port in the GCU of the target
  architecture that is called `xx'.
\end{enumerate}

Still, when `xx' is the name of a reserved (custom) register file, the
ambiguity is harder. How to tell whether a term `xx.2' refers to the custom
variable pool of the universal machine or to the actual reserved register
file of the target architecture? This might be a minor problem, since the
binding of custom variables and GPR's of a reserved RF is straightforward
and fixed (both name and index must match).



\appendix



\chapter{ASSEMBLY GRAMMAR}

The assembly language is defined by the following grammar. With the
exception of comments, which are terminated by the newline character, the
physical lines are completely ignored and treated as whitespace.

The assembly grammar is completely free-form.
%
Whitespaces and newlines can be freely mixed in the input program with all
tokens recognised by the grammar. Conversely, whitespaces can be removed
everywhere \emph{except between tokens that have no lexical separation}. For
most tokens, lexical separation is provided by tokens consisting of reserved
characters such as comma's, colons, plus, minus or equal sign. Two tokens
such as the guard term and the source term have no lexical separation, so
they must be separated by whitespace.

\section{Program Areas}

\begin{grammar}

\grrule{program}
    \grsym{area}*

\grrule{area}
    \grsym{data-area} \gror \grsym{code-area}

\grrule{data-area}
    \grsym{data-header} \grsym{data-lines}

\grrule{code-area}
    \grsym{code-header} \grsym{code-lines}

\end{grammar}

\section{Data Declarations}

\begin{grammar}

\grrule{data-header}
    \grlit{DATA} \grsym{addr-space} \grsym{number} \grlit{;}

\grrule{data-lines}
    (\grsym{data-line} \grlit{;})*

\grrule{data-line}
    (\gropt{\grsym{label}})*
    \grlit{DA} \grsym{number} \gropt{\grsym{init-data}}
    \\\gror \grsym{directive}

\grrule{init-data}
    \grsym{init-chunk}*

\grrule{init-chunk}
    \gropt{\grsym{init-size}} \grsym{init-term}

\grrule{init-size}
    \grsym{number} \grlit{:}

\end{grammar}

\section{Labels and Symbolic Expressions}

\begin{grammar}

\grrule{label}
    \grsym{name} \grlit{:}

\grrule{init-term}
    \grsym{imm-term}
    \gror \grsym{string-literal}
    \gror \grsym{float-literal}

\grrule{imm-term}
    \grsym{num-literal}
    \gror \grsym{expression}

\grrule{num-literal}
    \grsym{number} \gror \grsym{hex-number} \gror \grsym{bin-number}

\grrule{expression}
    \grsym{sym-expression}
    \\\gror \grsym{sym-expression} \grlit{=} \grsym{num-literal}

\grrule{sym-expression}
    \grsym{name} \gropt{\grsym{offset}}

\grrule{offset}
    (\grlit{$+$}|\grlit{$-$}) \grsym{num-literal}

\end{grammar}

\section{Code Area}

\begin{grammar}

\grrule{code-header}
    \grlit{CODE} \gropt{\grsym{number}} \grlit{;}

\grrule{code-lines}
    (\grsym{code-line} \grlit{;})*

\grrule{code-line}
    (\gropt{\grsym{label}})*
    \grsym{instruction}
    \\\gror \grsym{directive}

\grrule{instruction}
    (\grsym{empty-instruction}
    \\\gror \grsym{moves}) \grsym{immediate-specifier}*

\grrule{empty-instruction}
    \grlit{.\ \ .\ \ .}

\grrule{moves}
    \grsym{move} (\grlit{,} \grsym{move})*

\grrule{move}
    \grsym{empty-move}
    \\\gror \grsym{immediate-specifier}
    \\\gror \grsym{transport}

\grrule{empty-move}
    \grlit{\ldots}

\grrule{immediate-specifier}
    \grlit{[}(\grsym{rf-without-port}
    \gror (\grsym{rf-without-port} \grlit{=} \grsym{imm-term}))\grlit{]}

\grrule{transport}
    \gropt{\grsym{guard}} \grsym{source} \grlit{->} \grsym{destination}

\grrule{guard}
    \grsym{invert-flag} \grsym{guard-term}

\grrule{invert-flag}
    \grlit{?} \gror \grlit{!}

\grrule{guard-term}
    \grsym{rf-without-port}
    \gror \grsym{real-fu}

\grrule{source}
    \grsym{reg-term}
    \gror \grsym{imm-term}

\grrule{destination}
    \grsym{reg-term}

\grrule{reg-term}
    \grsym{rf-term}
    \gror \grsym{fu-term}
    \gror \grsym{bus-term}

\grrule{rf-term}
    \grsym{rf-with-port} \gror \grsym{rf-without-port}

\grrule{rf-with-port}
    \grsym{unit} \grlit{.} \grsym{port} \grlit{.} \grsym{number}

\grrule{rf-without-port}
    \grsym{unit} \grlit{.} \grsym{number}

\grrule{fu-term}
    \grsym{universal-fu} \gror \grsym{real-fu}

\grrule{universal-fu}
    \grsym{operation} \grlit{.} \grsym{number}

\grrule{real-fu}
    \grsym{fu-with-port} \gror \grsym{fu-without-port}

\grrule{fu-with-port}
    \grsym{unit} \grlit{.} \grsym{port} \gropt{(\grlit{.} \grsym{operation})}

\grrule{fu-without-port}
    \grsym{unit} \grlit{.} \grsym{operation} \grlit{.} \grsym{number}

\grrule{bus-term}
    \grlit{\{prev\}} \gror \grlit{\{next\}}

\end{grammar}

\section{Assembler Directives}

\begin{grammar}

\grrule{directive}
    \grlit{:} \grsym{name-directive} \grsym{directive-parameters}

\end{grammar}

The assembler recognises the following command directives:
\begin{description}
\item[:global \emph{label}] %
  Mark a given (defined) label as globally visible.
\item[:extern \emph{label}] %
  Mark a given (undefined) label as globally visible.
\item[:procedure \emph{label}] %
  Declare a starting point for a procedure. Only allowed in a code area.
\end{description}

The number and format of the directive parameters depends on the specific
command directive. The list of parameters is terminated by a semicolon.

\section{Name of Labels}
\label{sec:names}

The name of labels matches the grammar symbol \emph{name}, which represents
a string of characters with the following regular expression:
\begin{quote}
  [a-zA-Z\_][0-9a-zA-Z\_]*
\end{quote}

\section{Numbers}
\label{sec:numbers}

The numbers that appear in the grammar can be represented in various forms.

The symbol \emph{number} represents a decimal integer number and matches a
string of characters with the following regular expression:
\begin{quote}
  [0-9]+
\end{quote}

The symbol \emph{hex-number} represents an integer number in base 16 and
matches a string of characters with the following regular expression:
\begin{quote}
  0x[0-9A-Fa-f]+
\end{quote}

The symbol \emph{bin-number} represents an integer number in base 2 and
matches a string of characters with the following regular expression:
\begin{quote}
  0b[01]+
\end{quote}

\section{Strings}
\label{sec:strings}

The string literals are always quoted. Two quoting styles are possible:
\begin{enumerate}
\item %
  Single quote:
  \begin{quote}
    \verb|'|[\verb|^'|]*\verb|'|
  \end{quote}
\item %
  Double quote:
  \begin{quote}
    \verb|"|[\verb|^"|]*\verb|"|
  \end{quote}
\end{enumerate}

Empty strings are ignored.
%
\note{DISCUSS: forbid empty strings?}

A quoted string can contain any printable character of the host instruction
set. End-of-line characters (like ASCII LF and CR characters) are not
allowed and generate a parsing error.

\section{Symbols That Refer to Target Architecture}

The symbols \emph{addr-space}, \emph{unit} and \emph{port} match a string of
characters that represent the name of an element of the target architecture
definition. The set of characters allowed in these names is specified in the
Architecture Definition File Format~\cite{ADF-specs}. \emph{Addr-space}
represents the name of an address space. \emph{Unit} represents the name of
a function unit, register file or immediate unit. \emph{Port} represents the
name of a function unit port.

The symbol \emph{operation} represents a valid operation name and matches a
string with the restrictions specified in ``Operation Set Abstraction Layer
Functional Requirements''~\cite{OSAL-specs}.

\section{Syntax Variants}

The assembly language allows a number of syntax variants, in order to suit
the preferences of users and client applications

\subsection{Inverted Source/Destination}

The order of source and destination of a move can be inverted.  The token in
between, which represents the transport action, is replaced by `<-'.

The modified production is:
\begin{grammar}
\grrule{transport}
    \gropt{\grsym{guard}} \grsym{destination} \grlit{<-} \grsym{source}
\end{grammar}



%  References are generated with BibTeX from a bibtex file.
\bibliographystyle{alpha}
\cleardoublepage
%% Equivalent to a chapter in the table of contents
\addcontentsline{toc}{chapter}{BIBLIOGRAPHY}
\bibliography{Bibliography}


\end{document}


%%% Local Variables:
%%% mode: latex
%%% TeX-master: t
%%% End:
