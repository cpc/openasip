\documentclass[a4paper,twoside]{tce}

\usepackage{pslatex}

\begin{document}
\author{Ari Mets�halme, Andrea Cilio, Vladim�r Guzma}
\title{TTA Instruction Scheduler}
\ver{0.18.1}
\firstday{03.09.2004}
\lastday{09.03.2006}
% id number in S- sequence
\docnum{015}
% draft/complete/committed
\state{draft}
\maketitle

% highlighted style for parameters/variables or otherwise non-fixed parts
% of a command syntax or of lines of text displayed by the application
\newcommand{\parm}[1]{\textsl{#1\/}}


\chapter*{Version History}

\begin{HistoryTable}

 0.1    & 03.09.2004 & A. Mets�halme &
 Initial draft.\\
 0.2    & 08.09.2004 & A. Mets�halme &
 Wrote Chapters 1 and 2.\\
 0.2.1  & 21.09.2004 & A. Mets�halme &
 Fixes to Chapters 1 and 2.\\
 0.2.2  & 15.10.2004 & V. Guzma &
 First words in Chapter 4, Appendix B.\\
 0.2.3  & 17.10.2004 & A. Cilio &
 Cosmetic changes.\\
 0.2.4  & 02.11.2004 & V. Guzma &
 Fixed some of the issues pointed out by Andrea.\\
 0.2.5  & 04.11.2004 & A. Cilio &
 Added pending issues: configuration file, compression. More corrections
 after discussion with V.\ Guzma. \\
 0.2.6  & 05.11.2004 & V. Guzma &
 Added information on SSA form. Corrections and comments.\\
 0.2.7  & 23.11.2004 & V. Guzma &
 Configuration file now part of requirements. Various corrections.\\
 0.3    & 24.11.2004 & V. Guzma &
 Added part of notes from last scheduler meetings: rejected ideas, required
 features, pending issues.\\
 0.3.1  & 02.12.2004 & V. Guzma &
 Added more pending issues. Rejected alignment. Added references.\\
 0.3.2  & 08.12.2004 & V. Guzma &
 Moved trigger-on-result to required features. Added relative jump support.
 Reorganised pending issues.\\
 0.3.3  & 05.01.2005 & V. Guzma &
 Compression moved from pending issues to future research.\\
 0.3.4  & 10.01.2005 & V. Guzma &
 Added use cases. Separated early from later requirements.\\
 0.3.5  & 27.01.2005 & V. Guzma &
 Added details on configuration file and command line options.\\
 0.3.6  & 31.01.2005 & V. Guzma &
 Changes according to comments from J.\ Takala. \\
 0.3.7  & 31.01.2005 & V. Guzma &
 Resolved issue: interrupt support.\\
 0.3.8  & 31.01.2005 & A. Cilio &
 Minor corrections throughout the document.\\
 0.3.9  & 03.02.2005 & V. Guzma &
 Minor corrections. Added instruction replacement issue. Scheduler also as
 a module of Explorer.\\
 0.3.10 & 03.02.2005 & A. Cilio &
 Clarified compression-scheduling interaction.\\
 0.3.11 & 22.03.2005 & V. Guzma &
 Added compulsory functionality to configuration file.\\
 0.4    & 26.06.2005 & A. Cilio &
 Major update according to last discussions, 1--4. Rejected
 compression-unaware scheduler and instruction encoding size. Many
 corrections and extensions.\\
 0.5    & 28.06.2005 & A. Cilio &
 Major update according to last discussions, 5--8. Added chapter ``Helper
 Tools''. Revised configuration file. Solved instruction replacement
 issue.\\
 0.6    & 04.07.2005 & A. Cilio &
 Solved, corrected or extended various pending issues. Rejected script
 language for configuration file.\\
 0.7    & 08.07.2005 & A. Cilio &
 Extensive changes. Extended interrupt support. Added pending issues,
 ``Usage'' section, log file and interactivity specifications.\\
 0.8    & 10.07.2005 & A. Cilio &
 Added specification of configuration file format.\\
 0.8.1  & 12.07.2005 & A. Cilio &
 Minor change to software pipelining appendix.\\
 0.8.2  & 19.07.2005 & A. Cilio &
 Reorganisation of text (interactivity), corrections.\\
 0.9    & 20.07.2005 & A. Cilio &
 Changed configuration file specification.\\
 0.10   & 21.07.2005 & A. Cilio &
 Document reorganisation, moved material to other documents.\\
 0.10.1 & 01.08.2005 & A. Cilio &
 Added few Scheduler options and default search paths.\\
 0.11   & 18.08.2005 & A. Cilio &
 Rejected pass ordering constraints and ``minimal'' sequence of passes.
 Spelling corrections.\\
 0.12   & 19.08.2005 & A. Cilio &
 Added incomplete list of ``optimisations''. Added documentation chapter, 6.
 Revised Chapters~1,2,3. Moved text to other documents.\\
 0.12.1 & 22.08.2005 & L. Laasonen &
 Removed a paragraph from section 3.1.1.\\
 0.12.2 & 23.08.2005 & A. Cilio &
 Added definition section, 1.2. Minor changes.\\
 0.13   & 27.09.2005 & A. Cilio &
 Added a number of command line options.\\
 0.14   & 30.09.2005 & A. Cilio &
 Added command line options and intermediate output files.\\
 0.15   & 05.10.2005 & A. Cilio &
 Clarified module options and configuration file utilisation.\\
 0.15.1 & 10.10.2005 & V. Guzma &
 Added SSA tracing points, comments on some pending issues. Spelling
 corrections.\\
 0.15.2 & 10.10.2005 & A. Mets�halme &
 Added future extension: module-specific options in command line.\\
 0.15.3 & 10.10.2005 & A. Cilio &
 Added logging levels. Other minor changes.\\
 0.16   & 19.10.2005 & A. Cilio &
 Defined structure of log file.\\
 0.17   & 11.11.2005 & A. Cilio &
 Added module file name in configuration file.\\
 0.17.1 & 14.11.2005 & A. Cilio &
 Cleanup of configuration file description.\\
 0.18   & 15.11.2005 & A. Cilio &
 Clarified: compare and auto-increment operations.\\
 0.18.1 & 09.03.2006 & A. Cilio &
 Added necessary pass: node splitting.\\
\end{HistoryTable}


\tableofcontents



\chapter{INTRODUCTION}

\section{Purpose}

This document describes the functional requirements of the TTA instruction
scheduler of the TCE toolset, and gives an overview of the scheduler
application.

\section{Product Idea}

The TTA instruction scheduler is a fundamental part of the TCE compiler
backend. It reads unscheduled (or ``sequential'') TTA code, the description
of the target architecture and possible support data (execution profile,
access trace, etc.), and translates the sequential code to parallel TTA code
for target architecture, exploiting available ILP as well as possible.

The user of the scheduler is given full control of the scheduling process:
he/she is able to plug in new scheduling algorithms and configure the type
and
number of pre- and post-scheduling code analysis and transformation phases.

Scheduling can also be carried out in interactive mode, which can take
several forms from step-by-step scheduling of each move to a minimal request
for user interaction in case of difficult decisions.

\section{Product Overview}

The scheduler is the main step of the TCE backend compiler; transforms
sequential code as produced by the TCE front-end
compiler~\cite{TCE-frontend} into parallel and ``optimised'' TTA code.

The scheduler is implemented as an application controlled via a command-line
interface, and takes the following inputs:
\begin{enumerate}
\item%
  Description of the target architecture.
\item%
  Target application as sequential, partially scheduled or fully scheduled
  parallel TTA code.
\item%
  Auxiliary information, such as program profiling information and memory
  access trace data.
\item%
  Scheduler configuration file and command line options.
\end{enumerate}

From the given inputs the scheduler produces as outputs:
\begin{enumerate}
\item%
  The full or partially scheduled target application.
\item%
  Feedback from the scheduling process, such as a trace of the scheduling
  process, a list of resources that were not used, information on how well
  jump delay slots were filled, how much if-conversion was done in a given
  spot of the program or what was the highest register pressure.
\item%
  Processed auxiliary information.  For example, execution profile adjusted
  for the scheduled code.
\end{enumerate}

\section{Definitions}

\begin{description}
\item[Move]%
  Programmed data transport on a transport triggered architecture.
\item[Node]%
  The minimum program element in the program representation (a move, in case
  of a bus-programed TTA) that the Scheduler handles as an atomic unit.
\item[Resource]%
  A hardware component of the target TTA processor that can be used to carry
  out part of the program's activity.
\item[Resource Allocation]%
  Determining whether a resource can be assigned to a program element.
\item[Resource Assignment]%
  Reserving a resource to a program element (move, variable, operation,
  \ldots) for a given number of cycles.
\item[Result Move]%
  Move that reads the result of an operation from a port of a function unit.
\item[Scheduler]%
  The retargettable TTA instruction scheduler of the TTA templated
  architecture.
\item[Scheduling (general)]%
  The process of mapping a program to a given target architecture.
\item[Scheduling (proper)]%
  Placing instruction elements (in a bus-programmed TTA, moves) into
  instruction slots.
\item[(scheduling) Scope]%
  A part of the program whose moves are evaluated, processed and scheduled
  in the same phase. During scheduling, moves are not moved across scope
  boundaries.
\item[Trigger Move]%
  Move that writes a port of a function unit and thereby initiates a
  new operation on that unit.
\end{description}

\section{Acronyms and Abbreviations}

\begin{center}
\begin{tabular}{p{0.15\textwidth}p{0.85\textwidth}}
SCP   & Singe Copy on a Path (rule).\\
SSA   & Static Single Assignment.\\
ILP   & Instruction-Level Parallelism.\\
ISR   & Interrupt Service Routine.\\
FIFO  & First-In First-Out (queues).\\
MAU   & Minimum Addressable Unit (of a memory address space). \\
PDG   & Program Dependence Graph. \\
SSA   & Static Single Assignment. \\
TCE   & TTA Codesign Environment. \\
TPEF  & TTA Program Exchange Format. \\
TTA   & Transport Triggered Architectures. \\
\end{tabular}
\end{center}



\chapter{USER DESCRIPTION}

\section{User Profile}

The basic user uses scheduler as a part of design space exploration, he
experiments with different scheduling parameters and plug-in algorithms, and
tries to come up with the most suitable combination for his needs.
%
As a standalone application, the scheduler is used to map the target
application into the target TTA processor. Common users may also experiment
with different scheduling algorithms and optimisation techniques to come up
with most effective combinations for a given target architecture.
%
A more advanced type of user is one who develops new scheduling algorithms
and techniques.

\section{Key User Needs}

\begin{enumerate}
\item%
  Translate sequential TTA code to parallel TTA code for a target TTA.
\item%
  Translate partially scheduled TTA code to parallel TTA code.
\item%
  Evaluate the effects of different parameters that control the scheduling
  process on target application performance.
\item%
  Evaluate the effects of different scheduling algorithms and code
  transformations on target application performance, and select the
  best-performing combination of algorithm and heuristic
  parameters.\footnote{%
    Users are encouraged to try different scheduling algorithms and
    experiment with heuristic parameters on a per-procedure basis, possibly
    even per basic block.}
\end{enumerate}

\section{Use Cases}

This section describes the typical use cases of the Scheduler backend. The
main use of scheduler is to translate sequential TTA code to parallel TTA
code.
\subsection{Scheduling as part of Architecture Exploration}

The user of the framework in this case does not work with the Scheduler as a
stand-alone application. Instead, the Scheduler is part of design space
exploration.

After the exploration, preferred target architectures are selected from
those found. Then, the user may try to improve the application performance
by means described in Section~\ref{sec:UseCase:GivenTarget}.

\subsection{Scheduling for a Given Target}
\label{sec:UseCase:GivenTarget}

The target architecture is known and fixed, and the user wants to compile
and schedule the target program as efficiently as possible in order to
improve given objective functions, which are usually based on clock count,
energy consumption, and code size.

The Scheduler reads a TTA program in TPEF and the configuration file, and
also parses any command line parameters. Usually, the Scheduler is
configured so as to perform several code analysis and transformation passes
that try to improve the target program and improve the result of the
subsequent instruction scheduling process.

Afterwards, the user evaluates the performance of the scheduled program
(with the Estimator) and, if needed, repeats the whole process with
different scheduling parameters, different analysis and code transformation
passes, or different scheduling algorithms.

This process iterates until the user is satisfied with the performance
of the program on the target architecture.

\paragraph{Alternative scenario: exploiting profile and trace data.}
Profile and trace data (such as execution counts and memory access traces)
stored in TPEF for the given input application are used to improve the
scheduling process. When possible, the scheduler adjusts part of the
auxiliary data sets (for example, the execution counts) so as to keep them
consistent and (within approximation error) correct for the scheduled
program. If this is not possible, the scheduler invalidates
%
[[just remove?]]\note{DISCUSS}
%
the auxiliary data which is made wrong by the scheduling process.
%
Scheduling applications using auxiliary data collected from simulation runs
makes it possible to obtain ``perfect'' schedules (upper bounds) with
respect to certain scheduling techniques. For example, memory access traces
make it possible to compute memory-carried dependencies between all memory
operations of the program.  On condition that the program will be run on the
same input data set used to collect the trace data, this gives the scheduler
the equivalent of an ``oracle'' to determine memory dependencies. This is a
useful feature to estimate the effectiveness of the scheduling techniques
under examination.

\paragraph{Alternative scenario: interactive scheduling.}
When in interactive mode, the scheduler prompts the user for scheduling
decisions. The regions of the programs that are scheduled in interactive
mode, the level of interactivity (how much the scheduler prompts the user)
and the type of questions asked are controlled by parameters and may vary.

Whenever certain predefined conditions on the scheduling process state are
met, the scheduling algorithm halts and prompts the user for decisions or
hints. Different levels of interaction with users are possible: from a
step-by-step guidance of the entire scheduling process (a useful teaching
tool) to requests of interactions restricted to the most difficult
decisions, when the heuristics employed by the scheduling algorithm are
stymied.

\subsection{Developing a New Instruction Scheduling Algorithm}

In this case, an advanced user develops or modifies a scheduling algorithm
or a code transformation pass and verifies its correctness.

Any new pass implements a standard interface and is loadable at run time.
%
When developing a new scheduling algorithm, the developer can also choose to
implement a set of predefined interfaces for scheduling subtasks defined by
the scheduling software framework.
%
A number of software utilities (in form of helper objects) are also
available to simplify the development process.

The user can verify the correctness of new passes using the TTA Processor
Simulator~\cite{SimulatorSpecs}.



\chapter{PRODUCT OVERVIEW}

\section{Product Perspective}

TTA instruction scheduling is one of the most complex and critical
functionality offered by TCE. A powerful TTA processor and a highly
optimised target application are nothing without effective instruction
scheduling. The Scheduler is the application that implements instruction
scheduling. It can be used as a stand-alone tool, or as a module under
control of the design space explorer.

The main input to Scheduler consists of two data files: an unscheduled or
partially scheduled (for the same target architecture) TTA program and a
complete specification of the target architecture which the program has to
be mapped to.
%
Additional data files and command line parameters provide more input data.

Unscheduled TTA code is usually generated by the compiler front-end. This
code is targeted for special transport triggered architecture, an
abstraction that corresponds to TTA code where the data transports have not
been assigned to components of the target processor.
See~\cite{TCE-progTemplate} for details.

\subsection{Scheduler's Interaction with Code Compression}

Code compression is a post-scheduling process that applies to binary
parallel code. The instruction scheduler cannot always be unaware of the
code compression applied to the scheduled code. Some code compression
methods do not work at all if the scheduler does not take into account their
effect on the code.  Other compression methods are less demanding, they may
just produce better results if scheduler is aware of the compression
mechanism.  See~\ref{ch:future:compression} for a further discussion of this
issue.  Since code compression can be architecturally visible, a
compression-aware scheduler will not work in case the expected compression
algorithm is not applied.

While compressing program, the target address of jumps will change.
Independent of the scheduled algorithm, compressors usually recompute the
addresses of compressed instructions that are target of control flow
operations (jump targets, jump table slot initialisation values). Certain
compressors will rely on run-time hardware control to adjust instruction
addresses while the program runs (typically, during the fetch-decode
stage). Other compressors, however, may have to update the addresses of
compressed instruction at the time of compression, when the compressed
binary code is emitted.

\section{Product Position Statement}

Map an application expressed in a low-level, processor-oriented format (the
unscheduled TTA code) into parallel TTA code for a specific target TTA
processor.

\section{Summary of Capabilities}

\begin{center}
\begin{longtable}{p{0.45\textwidth}p{0.55\textwidth}}
  User Benefit                 &       Supporting Feature \\

 handle unforeseen code information &
 system to annotate instructions and moves\\

 high-performance scheduling   &
 list-based, operation-based, heuristic scheduling approach\\

 experiment with options and parameters &
 uniform system based on options specified in a configuration file\\

 experiment with different code transformation and scheduling passes &
 configuration file where it is possible to add, remove and reorder passes\\

 flexible and extensible scheduling infrastructure &
 standard module interface for pass implementations, possibility to develop
 and use new modules, module linkage at run time\\

\end{longtable}
\end{center}

%\section{Assumptions and Dependencies}

% (portability)
% (PJ: design constraints)

%  Assumed environment, third-party applications or library assumed to be
%  present. External services (like OS services) that are required.

% PEKKA:
% For example "the GUI should use wxWindows GUI library to ease portability
% to other operating systems and environments" or "XML should be used for
% all clear text persistent data storage".

%  In case of functional specifications of most modules, it's sufficient to
%  specify what is required in addition to assumption/dependencies given in
%  main functional specifications document of TCE Project, and refer to it
%  for the rest.

%\section{Distribution}

% (IPR and copyrights)

%%  Remove this section (not needed) for most modules and applications,
%%  because the conditions of the whole project apply.



\chapter{PRODUCT FEATURES}

% This chapter describes the functionality of the product in detail.

% In case of an application, all command line arguments and user interface
% screens (GUI) are described in detail.  Also, all tasks the module or tool
% must be able to accomplish are described here.

\section{Main modules}
In order to generate parallel TTA code, the instruction scheduler has to
implement at least following operations:
\begin{enumerate}
\item%
  Process command line parameters and configuration file.
\item%
  Read a description of the target architecture and generate a Machine
  Object Model (MOM).
\item%
  Read TTA program data from a TPEF file~\cite{TPEF-specs}, and generate a
  Program Object Model (POM).
\item%
  Perform pre-scheduling code analysis and code transformation passes on the
  program. The program representation is not fixed. It may be the Program
  Object Model itself, or a scheduler-specific object model (for example, a
  data dependency and control flow graph). The user can insert new passes.
  The exact number and type of passes is not fixed; some of the passes may
  be:
  \begin{enumerate}
  \item%
    Controlled node splitting, to ensure reducible control flow graphs.
  \item%
    Control-flow analysis.
  \item%
    Dataflow analysis.
  \item%
    Memory dependence analysis.
  \item%
    Interprocedural data and memory dependence analysis.
  \item%
    Control-flow transformations (loop unrolling, loop peeling, function
    inlining, dead code elimination, and so on). See
    Appendix~\ref{app:optimisations} for an extended list.
  \item%
    Dataflow code transformations (constant propagation, constant folding,
    value numbering, SSA conversion, vectorisation\footnote{
%
      Vectorisation is loop transformation that detects operation happening
      on multiple array elements and packs the data into vector registers
      allowing to execute such operation in vector unit (MultiVec in PPC,
      SSE, SSE2, SSE3 in Intel).}
%
    and so on). See Appendix~\ref{app:optimisations} for an extended list.
  \end{enumerate}
\item%
  Instruction scheduling proper and resource assignment. The scheduling
  scope, the scheduling algorithm, the resource assignment strategies and
  their supporting data structures may vary.
\item%
  Post-scheduling and post-resource assignment code transformations. These
  transformations are of such type that are not possible or very difficult
  until the entire program has been scheduled. For example, importing moves
  into function call delay slots may be an effective strategy to fill call
  delay slots and speed up function calls.
\item%
  Write the newly generated program (POM format) out as a TPEF file.
\end{enumerate}

\subsection{Scheduling Algorithm}

There are several aspects of the scheduling algorithm that can vary within
the framework: the scope, the resource allocation strategy, the operation
selection algorithm, the action taken when an attempt to schedule an
operation or a node thereof fails.

The scheduling scope may be one basic block or a group of basic blocks
(trace, superblock, decision tree, region).
%
\note{initial scheduling scope that will be delivered: single basic block}
%
Also, the scheduling scope may be cyclic or acyclic. The cyclic scope has a
deep effect on the entire scheduling algorithm and on resource assignment.
See Appendix~\ref{ap:SW-pipelining} for a discussion on a possible cyclic
scheduling algorithm.

Resource assignment defines which components of the target processor (called
resources) are used to carry out moves. It is tightly connected with the
scheduling algorithm and may be performed (partly) before, during or after
scheduling.

\section{``First Milestone'' Capabilities}

The following capabilities are expected to be supported early on in TCE
Scheduler framework.

\subsection{Multiple Instances of Operations with State}

In order to support programs that use multiple instances (that is, hardware
implementations) of the same operation with state, the
backend must determine which operations with state in the program belong to
the same dependency chain and which are independent.
%
\note{PENDING: \ref{ch:pending:state-operation}}
%
In practise, the dataflow dependencies are carried by internal state, not by
explicit program variables.  The state is indeed internal storage in
operation hardware, and can be considered a specialised variable.

The scheduler thus can determine the
number of operations with state needed at same time (number of independent,
overlapping dependency chains).

For practical purposes,\note{Implementation detail!} it is recommended to
use special type of node dependencies.\footnote{
%
  Such dependencies can tie all operations in a dependency chain that use or
  define the same internal state, thus creating a ``live range'' of the
  operation state.}
%
The dependencies, determined by properties \emph{affected-by} and
\emph{affect}, can be confirmed or rejected only when decision about FU
assignment is made.

The dependency does not assume any assignment, it only shows that the state
used or defined by two operations implies one of the following conditions:
\begin{enumerate}
\item%
  The two operations share the same instance of state. The operations must
  be assigned to the function unit that implements the state instance.
\item%
  The two operations access two unrelated instances of the same state type
  that are data-live at the same time. The operations must be assigned to
  independent function units (that implement the same operation state).
\item%
  The two operations access two completely unrelated instances of the same
  state type. The operations may be assigned to the same function unit, on
  condition that the live ranges of the two state variables are
  nonoverlapping.
\end{enumerate}

Significant implication
%
\note{see pending issue \ref{ch:pending:state-interprocedural} for
  additional problems of inter-procedure live state}
%
of operations with state is that the function units that implement
operations with state can not be allocated interchangeably. This is a major
restriction to resource function unit assignment.

\subsection{Support for Custom Operations that Access Memory}

All operations in TCE framework have memory-related properties (\emph{use
  memory}, \emph{mem-data}, \emph{mem-address}). This allows user-defined
operations to access memory same way as standard load/store operations.
Properties \emph{mem-data}, \emph{mem-address} indicate that an operand (or
result) is derived or used to compute, respectively, a data word transferred
between processor and memory or a memory address. This definition is
intentionally vague. The exact meaning of inputs and outputs marked by these
properties is completely dependent from the specific operation. Therefore,
any memory-related analysis requires specific and precise knowledge of the
operation (and ultimately of its addressing mode).

\subsection{Extensible Instruction Annotations}
The file format used throughout the TCE framework (TPEF) to store TTA
programs supports variable number of data words that encode tool-dependent
information about the instructions in a TTA program. A group of data words
form an annotation and is a mechanism for extending a TPEF code description
with additional information.

\subsection{Interactive Scheduling}
\label{ssec:interactive-mode}

The Scheduler can interact with users. Depending on the scheduling algorithm
modules, the amount of interactivity can vary, but a minimum level should be
offered by any scheduling algorithm.
%
Scheduling algorithms do not control interactivity, but have an influence on
it.

The Scheduler supports three levels of interactivity:
\begin{enumerate}
\item %
  Level 0: no interaction at all. All practical scheduling algorithms should
  work without requiring user interaction. If a scheduling algorithm needs
  user interaction, it will abort when this level (the default) is selected.
%
  \note{alternative: allow plug-ins to set default values for Scheduler
    global parameters}

\item %
  Level 1: on-request (algorithm-controlled) interactivity. Interaction in
  this case is under control of scheduling algorithms. Some algorithms may
  prompt users only in a limited set of cases, other may never need user
  interaction. If and only if a scheduling algorithm offers optional
  ``prompt points'', this level will make the algorithm stop and prompt the
  user for confirmation or alternative input. In this mode, algorithms
  should always propose a default choice when prompting the user.
%
  Usually, prompt point enabled by level 1 interactive move are concerning
  the most difficult decisions, such as:
  \begin{enumerate}
  \item%
    Whether to increase the size of a basic block and favour scheduling
    freedom (move importing) or restrict code motion and keep basic blocks
    compact.
  \item%
    Whether to overcome a violation of the single-copy-on-a-path rule by
    control flow transformation, and favour scheduling freedom or restrict
    code motion.
  \end{enumerate}

\item %
  Level 2: forced interaction mode. Every algorithm must stop at every
  standard ``prompt point'' of the scheduler framework. Such points are:
  \begin{enumerate}
  \item %
    Upon selecting a new procedure or scheduling scope.
  \item %
    Upon selecting a candidate node (group) for scheduling.
  \item %
    Upon defining the minimum cycle in which scheduling is tried for a node.
  \item %
    For assigning all required processor resources to the current candidate
    node.
  \item %
    Upon applying various scheduling optimisations: bypassing, operand
    sharing, move motion across control flow, code duplication, block
    duplication (to overcome SCP rule), and so on.
  \end{enumerate}
  Those algorithms that normally do not need to prompt the user will provide
  a list of choices and a default choice (which is what the algorithm would
  have selected had interactivity not been forced).
\end{enumerate}

Interactivity applies to core modules exclusively. Interactive scheduling
mode is an optional capability expressly requested by the user. No
scheduling algorithm should, by default, require interactivity. However,
there could be algorithms that simply work by prompting user for data and
decisions (intrinsically interactive algorithms). Such algorithms, mostly
for teaching purposes or experimental, should simply abort when the
Scheduler is invoked without enabling interactive mode.

\subsection{Tracking System}
\label{ssec:tracking}

The Tracking System consists of a single class that defines methods for
tracking the scheduling process.

Tracking can be configured along three different lines:
\begin{enumerate}
\item %
  Space: tracing of the scheduling process can be enabled or disabled
  independently for each procedure or scheduling scope (even single basic
  blocks).
\item %
  Phases: tracing of the scheduling process can be enabled or disabled at
  level of single macro-phases of the scheduling process such as scope
  selection, scheduling proper, post-scheduling compaction.
\item %
  Detail: the amount of detail (data) to be logged at each tracing point can
  be tuned from a minimum (which corresponds to no logging at all) to a
  maximum (all the data available).
\end{enumerate}

\paragraph{Scheduling macro-phases.}
The following macro-phases of scheduling are considered for independent
logging enabling/disabling:
\begin{enumerate}
\item %
  Control flow analysis.
\item %
  Dataflow analysis.
\item %
  SSA form analysis (when applicable).
\item %
  Scope selection.
\item %
  (Node) selection.
\item %
  Resource assignment (to the extent it is implemented as a separate pass).
\item %
  Scheduling proper (including resource assignment to the extent it is
  integrated into scheduling proper).
\item %
  Post-scheduling code transformations (for example, filling of call delay
  slots).
\end{enumerate}

Additional passes can register themselves as macro-phases that allow
independent control of logging.

\paragraph{Scheduling tracing points.}
Below is an incomplete list of the steps of the scheduling process that can
be logged and the type of data that should be recorded in the log file:
\begin{enumerate}
\item %
  \emph{Control flow analysis.} Detection of basic blocks, entry and exit
  points. Computation of (post)dominance relations between every basic block
  (square array).
\item %
  \emph{Dataflow analysis.} Liveness analysis: Computation of generate,
  kill, in and out liveness sets\footnote{%
    Assuming a standard algorithm for liveness analysis is used.},
%
  propagation of liveness information between basic blocks. Memory data
  dependency analysis. Stack layout analysis.
\item %
  \emph{SSA form analysis.} Construction phase: Computation of dominance
  frontiers (if implemented, other methods exists), where are phi functions
  added (and variables involved). Reverse phase: register register
  copies.
%
  \note{more details when applicable}
\item %
  \emph{Scope selection.} Build phase: detection of scheduling scopes (sets
  of basic blocks). Scheduling phase: logging of procedure and scope
  selection.
\item %
  \emph{Selection.} Data ready set at each scheduling step. List of
  candidate node groups (each group may relate to a single operation).
  Selected candidate. Possible heuristic details (such as internal priority
  used for selection).
\item %
  \emph{Resource assignment.} Every attempt failure and success in
  allocating a resource type to a program element. Reasons for failure
  (conflicting resources). Possible choices between multiple legal
  assignments, with possible heuristic details (such as internal priority
  for choosing one resource over another).
\item %
  \emph{Scheduling proper.} Order in which single nodes (of a selected
  group) are scheduled. Possible rejections of selections. Failures in
  scheduling attempts. Delays or anticipations of scheduling cycle.
\item %
  \emph{Post-scheduling code transformations.} Delay slots successfully,
  partially or not filled. Which target instructions are used to fill the
  slots.
\end{enumerate}

\subsection{Instruction Selection and Replacement}

The TCE front-end generates sequential code using a very rich basic operation
set. In practise, all complex operations that can be known to front-end are
used by it.

Although unlikely, it is possible that in target parallel architecture,
there could be function units supporting additional base operations not well
exploited by the front-end.\footnote{
%
  For example, some versions of gcc-front-end do not recognise as a
  ``minimum'' operation the pattern \texttt{a>b?b:a} but only
  \texttt{a<b?a:b}.}

More generally, the instruction selector and replacer must identify single
operations or patterns of sequential code that could be replaced by a
sequence of simpler operations supported by target architecture, or, in case
such sequence would be too complex, by function invocations.

\emph{Example}: Replacement of divide operations. The sequential divide
operation
\begin{verbatim}
  r12 -> div.1
  r13 -> div.2
  div.3 -> r14
\end{verbatim}
is replaced by
\begin{verbatim}
  r12 -> r2
  r13 -> r3
  __divSI3 -> call.1
  r0 -> r14
\end{verbatim}

Instruction selection and replacement is a simple peephole optimisation
pass; it is not intended as a pass for identifying operation patterns to be
replaced by complex (compound) operations or user-defined operations.

The operation replacer must take place early in the code transformation and
scheduling chain, so that the resulting operations can be further optimised
by subsequent passes.

The selector and replacer is very important, because it is what ensures that
target architectures that do not support the entire base operation set used
by the TTA front-end compiler can still be scheduled for.

\subsection{Support for Code Generation During Scheduling}
\label{ssec:fundamental-operations}

TTA instruction scheduling cannot be limited to reordering moves. In certain
situations, a scheduling algorithm (especially if sophisticated) will have
to generate or replace operations of the target program: for example, insert
save and restore code around function calls and register spilling to memory.

In order to correctly handle the semantics of the program, any scheduler
algorithm must be aware of a minimal, fundamental operation set. Below is a
list of operations for which scheduler specific knowledge is required:
\begin{enumerate}
\item %
  CALL (and similar). Motivation: handle save and restore code, mark certain
  registers as clobbered by the callee.\footnote{%
    As many call-specific aspects as possible should however be captured by
    pre-computed data dependencies and not be managed by scheduler as part
    of call-specific activity.}
\item %
  JUMP (and similar). Motivation: handle optimising control flow code
  transformations and if-conversion.
\item %
  ADD. Motivation: compute memory addresses for save and restore code, for
  arguments passed on the stack and for spilling.
\item %
  LDW, STW. Motivation: generate save and restore code (around calls and for
  spilled registers), recognise function arguments saved on the stack. The
  word size is defined as the size of the (most numerous) integer GPR's and
  (most numerous) function units.
\item %
  EQ, GT, GTU. Motivation: handle if-conversion and optimising code
  transformations related to control flow.
\item %
  AND, OR. Motivation: (optionally) generate complex guards for optimising
  control flow code transformations and if-conversion.
\end{enumerate}

\note{more fundamental operations may be necessary}

Custom, user-defined operations cannot require specific knowledge from the
scheduling algorithms.\footnote{%
  This would require updating all existing algorithms whenever a new custom
  operation requiring scheduler awareness is added!}
%
\note{PENDING \ref{ch:pending:custom-awareness}}

\paragraph{Semantics of Compare Operations}

The predefined compare operations of the base operation repertoire are
assumed to have a well-defined behaviour. These operations generate only two
possible values, independent of the bit width of their result: 1 (which
usually stands for condition \emph{true}) and 0 (which usually stands for
condition \emph{false}). Notice that, usually, any value different from zero
is treated as a \emph{true} condition.

\section{Additional Capabilities}

This section lists noncritical capabilities to be implemented in later
versions of Scheduler.

\subsection{Trigger On Result}

For practical purposes, trigger-on-result operations make sense only if they
have an internal state. This means that the state must be initialised before
the first trigger is executed, so trigger-on-result operations need a
support operation to initialise the state. There is need for control and
data dependency between such initialisation operation and following trigger
on result operations.
%
\note{"affects property"}
%
\note{This feature may be implemented later, if time allows.}

This is very similar to operations with state as such. Difference is that
result move triggers next execution of unit. The main problems are:
\begin{enumerate}
\item%
  The operand moves are related to the result move of previous operation.
  This result move is different from the result move that will read outcome
  of operation started with operand.\footnote{%
    While the possibility of operand moves with operations triggered on
    results may be retained out of orthogonality reasons, it is very
    impractical to define trigger-on-result operations with any operands at
    all. These could be turned into conventional input-triggered operations,
    which are much less problematic.}
\item%
  If operand is also triggering, then there is a time limit within which the
  result has to be read before it is overwritten. In practise this is not
  much more difficult than in a normal input trigger move, except that now
  the trigger and the result move of the previous operation are compound in
  one.
\end{enumerate}

\subsection{Multiple Memory Banks}

Memory of TTA processor may consist of several address spaces.
%
\note{specification of address space in source code not considered/solved.
  For start, multiported shared memory space is sufficient.}
%
Each space can be accessed by one or more load-store units. Each load-store
unit can however access only one memory address space (the address space is
a parameter associated with each instance of load-store unit).

\subsection{Relative Instruction Addressing Modes}

Relative jumps are a required feature of TCE.

Computing the address of forward jumps is a well-known problem. Probably
first solution to this problem was by Grace Murray Hopper, in her A-0
compiler (only single pass compiler built). Solution as presented by her
words:
\begin{quote}
  ``And here comes in the curious fact that sometimes something totally
  extraneous to what you are doing will lead you to an answer. It so
  happened that when I was an undergraduate at college I played basketball
  under the old women's rules which divided the court into two halves, and
  there were six on a team; we had both a center and a side center, and I
  was the side center. Under the rules, you could dribble only once and you
  couldn't take a step while you had the ball in your hands. Therefore, if
  you got the ball and you wanted to get down there under the basket, you
  used what we called a `forward pass.' You looked for a member of your
  team, threw the ball over, ran like the dickens up ahead, and she threw
  the ball back to you. So it seemed to me that this was an appropriate way
  of solving the problem I was facing of the forward jumps! I tucked a
  little section down at the end of the memory which I called the `neutral
  corner.' At the time I wanted to jump forward from the routine I was
  working on, I jumped to a spot in the `neutral corner.' I then set up a
  flag for the [forward operation] which said, `I've got a message for you.'
  This meant that each routine, as I processed it, had to look and see if it
  had a flag; if it did, it put a second jump from the neutral corner to the
  beginning of the routine, and it was possible to make a single-pass
  compiler and the concept did come from playing basketball.''\footnote{%
    This solution is added for completeness and educational purposes. There
    were days when debugging code meant finding longest hall in building and
    unroll pile of paper to see what is happening to program. This solution
    doubles jump latency for forward jumps, which is only acceptable if
    single pass compiler is necessary!}
\end{quote}

\subsection{Interrupt Support}

Support for interrupts in TTA processors is a complicated problem. Before
final decision is made, it is necessary to confirm that such support is
really needed. This feature is not considered for early versions of TCE
framework, perhaps never.\note{%
  Jarmo Takala: ``It might be good to have the interrupts but it's not high
  priority yet''.}
%
It seems necessary to find out what are possible ways how to implement
interrupts in future.

One solution proposed by J. Takala is to have reserved word which indicates
that procedure is an ISR. Front-end then generates sequential code for saving
and restoring context. Scheduler will schedule moves, without necessarily
knowing that procedure is ISR.

\paragraph{Known problems.}
The main problem of interrupt support for TTA's is that the processor state
is larger and less ``regular'' than in an equivalent operation-triggered
processor architecture. Irregularity is due to the fact that, in addition to
GPR's, part of the FU inputs and outputs can be machine state. Worse yet,
also internal pipeline stages may be part of the state.

Depending on the level of hardware support for interrupts, interrupt support
can have a varying impact on code generation and instruction scheduling,
from no effect at all to some (or all) the effects listed below.
\begin{enumerate}
\item %
  Additional scheduling constraints (such as, maximum distance between
  operation inputs and opcode-setting input).
\item %
  Knowledge of ``can throw'' property of operations. The scheduler uses this
  property to adjust or limit code motion.
\item %
  Specific code patterns to fulfil. For example, creation of context save
  and restore around interrupt handlers, code to flush and restore execution
  pipelines.
\item %
  Specific scheduling patterns to fulfil. For example, forced software
  bypassing of save and restore code around interrupt handlers; restricted
  availability of GPR's; schedule to flush execution pipelines.
\end{enumerate}

\paragraph{Possible solutions.}
\begin{enumerate}
\item \emph{Partial Unit Locking} (P. J��skel�inen) %
  Some kind of partial locking of the unused parts of the processor, partial
  context saving of the parts needed by the interrupt handler, could help.

  This means that the interrupt handler code could run on a subset of the
  processor resources. The interesting property of this type of solutions is
  that it leaves a degree of freedom: the trade-off between fast and cheap
  state save and restore (in general, context switching) and the performance
  at which exception handler code can run. This solution could be completely
  unacceptable if exception handler code needs to meet stringent speed
  requirements.

\item \emph{Complete Hardware-based Save-restore} (A. Cilio) %
  The brute-force approach is, of course, to simply save the part of the
  processor state that is not directly accessible (FU inputs, pipeline
  stages and outputs) in hardware. The rest of the processor state,
  consisting of GPR's contents, could be saved in software.

  This approach could be a suitable candidate for those units that are not
  locked in the ``Partial Unit Locking'' method sketched above.

\item \emph{Busflashing} (H. Corporaal~\cite{HCorp97}) %
  This approach consists in saving in a FIFO (the \emph{busflash}) any
  signal that crosses the transport busses for a finite number of cycles.
  Upon interrupt triggering, the busflash is frozen. After the processor
  returns from interrupt handling, the contents of the busflash is
  ``replayed'' onto the transport busses, thus recreating the state at the
  time the interrupt was triggered.

  There are many variants, improvements or details to this approach:
  \begin{enumerate}
  \item %
    Operations with side-effects (such as store) or with state (such as, in
    a broad sense, load from memory) cannot be simply replayed. The busflash
    must be managed by logic prevent triggering these operations during
    busflash replay. The logic could invalidate triggering inputs and
    restore the operation results only.
  \item %
    The length of the busflash must be at least as long as the latency of
    the slowest operation in the target operation set.
  \item %
    There is a trade-off between the length of the busflash (and thus the
    cost of control logic) and the scheduling freedom. This is a positive
    property that allows to trade-off performance potential with hardware
    cost.
  \item %
    Optimisations are possible. For example, the busflash could avoid
    recording those transports for which software save-restore is trivial.
    Probably, complete avoidance is impossible, but the busflash could
    optimise away certain transports. For example, GPR writes that are
    data-killed by a subsequent write to same GPR recorded in the busflash
    could be just ejected from the busflash. Similarly, the last GPR write
    still live could be always ignored during replay, since the software
    save-restore can be used instead.
  \end{enumerate}
\end{enumerate}

\section{Usage}
\label{sec:usage}

In addition to the standard command line options accepted by every
application of the TCE toolset [[document to define]], the CLI-based
Scheduler accepts the command line options listed below. All command line
options are prepended by two hyphens: `\verb|--|'. For each option, the
second column indicates the type of argument (if any) of the option:
\emph{i} (integer number), \emph{k} (keyword), \emph{s} (string), \emph{f}
(fractional number), \emph{b} (no option, that is, boolean). A Boolean
option disables the flag it controls if prepended by `\verb|no-|'. Unless
otherwise stated, other option types cannot be prepended by the `\verb|no-|'
prefix.

\begin{center}
\begin{longtable}[htb]{@{}p{.20\textwidth}@{}p{.05\textwidth}%
                     @{}p{.75\textwidth}}

\verb|target| (\verb|t|)  & \emph{s} &
%
Define the name of the target architecture description file. If not
specified, a default file called `target.adf' is searched in current
directory. If no file is found, Scheduler can only run target-independent
passes (passes that simply restructure the code).\\

\verb|output| (\verb|o|)  & \emph{s} &
%
Define the output TPEF file name. By default, the Scheduler creates a file
with the name `\emph{basename}-sched\emph{ext}', where \emph{basename} is
the base name of the input file, and \emph{ext}, if present, it the
extension of the input file name (prepended by `.'). If a file with the
same name exists, it will be overwritten.\\

\verb|proc| (\verb|p|)    & \emph{s} &
%
Select procedure for scheduling. If `no-' is prepended, exclude procedure.
This option can be repeated, as long as versions with and without `no-'
prefix are mixed in the same command line. See
Section~\ref{ssec:proc-selection} for details on the format of procedure's
names.\\

\verb|interactive| (\verb|i|) & \emph{i} &
%
Set interactivity level: no interaction at all (0), interaction only if
requested by scheduling algorithm (1), forced interaction, turn algorithm
decisions into user prompts (2). All values
greater than 2 are treated as 2.\\

\verb|config| (\verb|c|) & \emph{s} &
%
Define the configuration file name. If not specified, the default file name
is used. See Section~\ref{ssec:files:config} for details.\\

\verb|log| (\verb|l|) & \emph{i} &
%
Set logging verbosity. Level 0 disables any logging of scheduling activity.
Passes should react differently at least to levels 1--3.\\

\verb|logfile|        & \emph{s} &
%
Define the log file name. If not specified, a default log file name is
created (see Section~\ref{ssec:files:log} for details). If logging is not
enabled, this option has no effect.\\

\verb|ignore-missing| & \emph{b} &
%
Do not abort if any of the plug-in modules specified in the configuration
file is missing. By default, the Scheduler must abort with error message. \\

\verb|timeout| & \emph{i} &
%
Define the maximum total run time (in seconds) permitted. If the limit is
exceeded, Scheduler terminates with an error condition. The run time is
implementation- and system-dependent, and should be (with approximation) the
real clock time. Thus it could include the time spent on other processes by
the host system. By default, the Scheduler runs without time limits.\\

\verb|warning| (\verb|w|) & \emph{w} &
%
Enable printing of warnings. By default, Scheduler does not output any
warning messages.\\

\verb|save-temps| & \emph{s} &
%
Save the target program as TPEF file each time a pass of the code
transformation and scheduling chain completes. See
Section~\ref{ssec:files:intermediates} for the naming scheme of these
intermediate files.\\
\end{longtable}
\end{center}

\note{DISCUSS: warnings to stdout or to file? Command line choice?}
\note{DISCUSS: warnings, logging, tracing - how do they relate to each other,
how are they distinct?}

\subsection{Code Selection}
\label{ssec:proc-selection}

Selection of code can be applied at level of procedures. Procedures are
usually identified by their name. In some cases, however, the name is not
sufficient, because multiple procedures with the same name are linked from
different object files. In that case, the name of the object file (with or
without extension), followed by a colon `:', should be prepended to the
procedure's name.

The name of the procedure, in case of C++ code, must be the actual symbol's
name, which is ``mangled'' by the frontend compiler.
%
\note{DISCUSS: possible extension}

In addition to plain procedure's names, a number of special shortcuts are
supported to make selection of code easier. All shortcuts consist of a
string enclosed in `(', ')' characters, to prevent any possibility of name
conflicts with existing procedures.
\begin{description}
\item[(\emph{number})] %
  Select (or exclude) the first \emph{number} most executed procedures. This
  option causes an error if the input program is not profiled.
\item[(\emph{number}\%)] %
  Select (or exclude) the minimum set of procedures that together cover no
  less than \emph{number}\% of the total cycle count. Since the set must be
  minimum, only the most executed procedures are considered. This option
  causes an error if the input program is not profiled.
\item[(executed)] %
  Select (or exclude) all procedures that have been executed. If the input
  program is not profiled, this option has no effect.
\item[(application)] %
  Select (or exclude) all procedures that belong to user code. As a result,
  all procedures linked from standard libraries are excluded (or selected).
\end{description}



\chapter{DATABASES}

%\section{Contents of Information}

\section{Intensity of Use}

The Scheduler is the single most crucial application of the TCE toolset. Its
task is, generally, complex and time-consuming.

The typical use of Scheduler is a single invocation by user, therefore some
speed can be traded off for efficiency of parallel code. However,the core
modules of the scheduling algorithm can be invoked automatically and
repeatedly by the Design Space Explorer~\cite{} or
Estimator~\cite{EstimatorSpecs}, even thousands of times for a single
exploration run. For this reason, efficient and reasonably fast scheduling
algorithms are an important requirement.

\section{Files and Configuration}

The Scheduler is highly configurable. First of all, the
scheduling process can be carried out by different algorithms for code
selection, resource assignment and scheduling (move placement). In addition,
a vast set of quantitative parameters affect the heuristics that guide the
algorithms.

The algorithms that perform instruction scheduling and the parameters that
regulate the algorithms are defined by means of a configuration file.
%
The instruction scheduler reads and processes the configuration file. Since
any kind of code analysis and transformation pass can be added to the
configuration file, the Scheduler really becomes a backend, and scheduling
proper can be considered as just one pass.

\subsection{Scheduler Configuration File}
\label{ssec:files:config}

A configuration file controls the process of instruction scheduling and all
the auxiliary phases of code analysis and code transformation.

The configuration file specifies which (plug-in) modules form the ``code
transformation and scheduling chain''. Modules can cooperate and form a
hierarchy. A module implementing an optimisation pass may, internally,
invoke other modules repeatedly to carry out part of its work. Even
different modules in different phases of the code generation and scheduling
process may invoke the same module as part of their activity.

The configuration file defines:
\begin{enumerate}
\item%
  Which code analysis and transformation passes (including scheduling) to
  apply and in which order.
\item%
  Custom parameters (even data files with parameters) for specific modules.
\end{enumerate}

Since it is impossible to anticipate all future configurations of the
scheduling chain nor the interaction between future passes, the format of
the configuration file does not provide any mechanism to force or verify the
order of passes. Even if it does not make much sense to perform certain
passes before others, (for example, register assignment before by loop
unrolling and dead code elimination), this type of ordering problem are left
to the users and the developers of pass modules. A pass will simply fail or
(in the worst case) work incorrectly if applied in the wrong order with
respect to another pass.
%
Modules should be implemented so that when they cannot carry out their work
normally they can let the client decide whether to skip or stop with error.

The configuration file has XML format. In its simplest, trivial form, the
configuration file is completely empty and defines a ``null scheduler'',
which simply outputs the same target program it received as input.
%
Generally, the configuration file consists of a list of pass declarations:
\begin{verbatim}
  <pass> . . . </pass>
\end{verbatim}

Each pass consists of one or more dynamically linked modules (plug-ins). The
actual activity performed by a pass can be anything. For example: resource
allocation, instruction scheduling, code analysis and annotation, optimising
code transformations. There are no restrictions in what a single pass of the
chain can do, except that it works on an input target program and (usually)
modified it. A single module (for example, an integrated scheduler and
register allocator) can implement operations that are usually separated in
two or more passes.

The plug-in modules that implement a given pass are declared by the
mandatory (empty passes are not allowed) \emph{module} element:
\begin{verbatim}
  <module> . . . </module>
\end{verbatim}


When a pass declaration contains two or more module declarations, it means
that the pass is implemented by means of a group of cooperating objects
implemented in different modules. One and only one of the modules in a pass
declaration, called the main module, takes care of coordinating the activity
of the other modules (called helper modules). The main module is the only
module of a pass that can be started \emph{independently}.

Each module is identified by a name, given by the mandatory \emph{name}
element. The file that contains the module implementation is independent of
the module name and is given by the mandatory \emph{file} element.
Optionally, a module can accept a set of run time options, which are given
as strings by means of the \emph{options} element:
\begin{verbatim}
  <name> string </name>
  <file> string </file>
  <options> string </options>
  <options> string </options>
  . . .
\end{verbatim}

Module options are processed as follows. First, all whitespace before the
first and after the last non-blank character are removed from the string of
each \emph{options} element. Then, the strings are concatenated and a blank
is inserted between a string and the next. The string obtained is then
pre-parsed and split into tokens with the usual rules applied to command
lines, which are:
\begin{enumerate}
\item %
  The backslash followed by a blank `\verb|\|\ ' defines a hard space that
  is part of the current token.
\item %
  All characters enclosed in a pair of double quotes `"', including blanks,
  belong to the same token.
\end{enumerate}
The list of tokens obtained by pre-parsing is then passed on to the module
for parsing proper.

\emph{Example.} Specification of passes and tasks for register and bus
allocation.
\begin{verbatim}
<pass>
  <module>
    <name>EarlyRegisterAllocator</name>
    <file>ERA_simple_01</file>
    <options>--import-aggressiveness=12</options>
    <options>--scp-optimize --optimize-leaf-calls</options>
  </module>
</pass>
\end{verbatim}
%
The plug-in module called ``\verb|EarlyRegisterAllocator|'' is loaded and
initialised with the command line options represented by the following
tokens: ``\verb|--import-aggressiveness=12|'', ``\verb|--scp-optimize|'',
``\verb|--optimize-leaf-calls|''.

If the configuration file is not specified in the command line option, the
Scheduler runs with an empty scheduling chain, with no modules at all, and
does not modify the input TTA program.

When the given configuration file does not have an absolute path, the
Scheduler searches for the file name (and its relative path, if present) in
the following paths:
\begin{enumerate}
\item %
  Current directory
\item %
  \texttt{\parm{HOME}/.tce/conf}
\item %
  \texttt{\parm{ROOT}/share/tce/conf}
\end{enumerate}

The parameter \emph{ROOT} contains the path where the toolset is installed
and \emph{HOME} contains the path to the home directory of the user who is
running the toolset applications.

\subsection{Scheduler Log File}
\label{ssec:files:log}

The scheduling algorithms can optionally log the activities performed.
Logging stores in an output stream information such as:
\begin{enumerate}
\item %
  The ordered list of which candidate moves are selected.
\item %
  Which scheduling attempts succeed, in which cycle that happens, which
  resources are assigned.
\item %
  Which scheduling attempts fail, in which cycle and for what reason (which
  hardware resources are unavailable).
\end{enumerate}

\paragraph{Log file name.}

By default, the scheduler log file is created in the current directory and
is given a name created as follows:
\begin{quote}\tt
  \parm{basename}.schedlog.\parm{nn}
\end{quote}
where \emph{basename} is the name (without extensions) of the input TPEF
file (the TTA application), \emph{nn} is a 2-digit number assigned to ensure
that the file name is unique (up to 99 unique logs are possible). When a new
log file with default name is created, the lowest number not yet assigned to
another file name is assigned.

\paragraph{Log file format.}

The scheduling log file does not have a format.
%
\note{DISCUSS: format details}
%
Every client can log its textual messages in a free format. There is,
however, a minimal structure in the log. The file is divided in sections.
Each section contains one or more messages coming from the same client (pass
module) and is introduced by a section header with the following format:
\begin{quote}\tt
  [\parm{pass}] \parm{module}
\end{quote}
%
where \emph{pass} is a 3-digit number that identifies the pass currently run
and \emph{module} is the name of the module that implements the pass. If the
same module is run multiple times during the scheduling process, each time
the section header will show a different \emph{pass} number.

Optionally, each module can organise its messages in sub-sections.
Presumably, each sub-section represents a different phase of the activity
implemented by the module. For example, a module could define a sub-section
for each major class or component. Sub-sections can be interleaved. Each
sub-section is introduced by a header with the following format:
\begin{quote}\tt
  [\parm{pass}:\parm{index}] \parm{subsection-tag}
\end{quote}
%
where \emph{index} is a unique 10-digit integer that identifies the current
sub-section, and \emph{subsection-tag} is a string that describes or
identifies the component that generated the message.

\emph{Example. Log text.} The following text shows a possible example of log
messages generated by three components of module ``Integrated Register
Allocator On-demand'' that cooperate to assign registers. One component is
responsible for finding free registers for program variables, and identified
by the string ``register assignment''; another is responsible for spilling
registers to memory; the last component is given the task of adding or
removing code.
%
\begin{verbatim}
  [005] Integrated Register Allocator On-demand
   . . .
  [005:0000000341] register assignment
  Assigned register to program variable r.355: RFa.12.
  No register available for program variable: r.367, trying spilling.
  [005:0000000342] register spilling
  Selecting program variable to spill.
  Spill candidate found: r.355. Spill cost: 506 (heuristic: minimize
  loop spilling).
  [005:0000000343] code factory
  Adding save and restore points for program variable: r.355.
  Stack location assigned: +8.
  Save point for definition at cycle 13: add.3 -> r.355.
  Restore point for use at cycle 38: r.355 -> eq.2.
  [005:0000000343] register assignment
  Assigned register to program variable  r.367: RFa.12.
  . . .
\end{verbatim}

\subsection{Scheduler Command Line Parameters}

The purpose of command line options is to control how the scheduler behaves
with respect to the outside world.  Command line options are not meant for
controlling the scheduling process (algorithms used, heuristic parameters,
iteration of analysis and transformation passes).

The command line parameters includes:
\begin{enumerate}
\item %
  Verbosity, how much of the information about its work scheduler shows to
  the user. There are different levels of verbosity, depending on amount of
  information about internal workings of scheduler required.
\item %
  Configuration file, the location of file to be used to control
  scheduling.  See Section~\ref{ssec:files:config} for details.
%
  In case no configuration file is specified, the scheduler reads a default
  configuration file.  Stored in a standard location of the toolset
  installation, the default configuration file contains the most ``common''
  basic configuration options.
\item %
  Architecture description file (ADF) defining a target architecture.
\item %
  Target TTA program to schedule, possibly including auxiliary data such as
  memory access trace and execution profile.
\item%
  Enabling interactive mode, level of interactivity.  See
  Section~\ref{ssec:interactive-mode} for details.
\item %
  Tracing and logging.  Level of detail, selection of what to trace, output
  stream.  The scheduler \emph{never} outputs any information (except during
  user interaction) to standard output. See Section~\ref{ssec:tracking} for
  details.
\end{enumerate}

\subsection{Intermediate Output Files}
\label{ssec:files:intermediates}

The Scheduler can optionally save a copy of the target application after
each pass of the code transformation and scheduling chain is completed. The
name of each intermediate TPEF file is defined as follows:
\begin{quote}\tt
  \parm{basename}-\parm{num}-\parm{pass}\parm{ext}
\end{quote}
where \emph{basename} is the base name of the input TPEF file, \emph{num} is
the two-digit ordinal number of the pass (starting from `01'), \emph{pass}
is the name of the pass, and \emph{ext}, if present, is the extension of the
input TPEF file, prepended by `.'.



%\chapter{OTHER PRODUCT REQUIREMENTS}

%\section{Applicable Standards}

%% Remove this section (not needed) if no extra standards in addition to
%% those specified in the main functional specification document of the
%% project.

%\section{System Requirements}

%% Remove this section (not needed) for most modules of a larger project.
%% May be required in some critical applications if requirements are
%% significantly different from those of the rest of the system.

%  Memory space
%  Disk space
%  Processor Performance

%\section{Performance Requirements}

%% Remove this section (redundant) for most modules and applications of a
%% larger project.

%  response times to user

%\section{Environment Requirements}

%% Remove this section (not needed) for functional specifications of
%% modules or applications of a larger project.

%  Requirements in terms of computer environment, administrator rights for
%  installation, level of expertise - only for complete products.

%\section{Security, Recovery, Usability}

%% Remove this section (not needed) for functional specifications of
%% modules or applications of a larger project.



\chapter{DOCUMENTATION}

\section{User Manual}

The Scheduler is fully documented by a users' manual. The user manual
describes the general command line options of the Scheduler, the graphical
elements, the definition of the configuration file.

Detailed description of passes of the code transformation and scheduling
chain [[to define: single document, separate documents, appendix of users'
manual]]

\section{Online Help}

The GUI-based Scheduler front-end is complete with a built-in browser for
on-line help documentation. The documentation is integrated within the
Scheduler.

The CLI-based Scheduler provides an on-screen summary of the command line
options and, when applicable, their default values.

\section{Installation and Configuration}

The installation of the Scheduler is an integral part of the TCE system, and
is documented in the TCE Installation and Configuration
Guide~\cite{ProjectPlan}.


\section{System Reference and Developer's Guide}

System Reference of the Scheduler consists of two parts. In the fist part,
an introductory document (based on the design notes written during
development) gives an overview of the Scheduler design and a developer's
guide instructs users who want to develop new Scheduler passes and integrate
them into the Scheduler system. This part contains developer-level, accurate
documentation of all software utilities and interfaces that for the
scheduling framework. The second part documents the application programming
interface (API); it is automatically generated in hypertext form and is
integrated within the TCE API documentation.%
\note{DISCUSS: better separate API docs of each application?}

The developer's guide is completed with a tutorial that teaches how to build
and integrate a new pass into the Scheduler, and how to use it.

%\chapter{KNOWN PROBLEMS AND RISKS}



\chapter{HELPER TOOLS}

This chapter describes standalone applications that are required in order to
help scheduler operation.

\section{Execution Profile Editor}

The execution profile editor gives users complete control on profile data,
and makes it possible to modify, create from scratch or merge execution
profiles. As a result of every modification of an execution count, the
profile editor should update the execution counts according to control flow
information so as to keep the profile consistent.

The interface of this tool should be graphical. The tool should provide
convenient navigation through program procedures and control flow graph.

Updating of definition of execution counts should be provided in two main
ways: direct definition of a basic block execution count or definition of
the frequency (fraction of overall execution count) of each path of a branch
point.

[[add more]]



\chapter{REJECTED OR ABANDONED IDEAS}

\section{Ideas related to Hardware}

\subsection{Optional Operands}
Optional operands are not supported by TCE framework.
\begin{enumerate}
\item%
  Optional operands imply that operation has a state.
\item%
  Property \emph{affected-by} could be different depending on presence of
  optional operand.
\end{enumerate}

In practical cases, optional operands can be ``simulated'' by two operations:
\begin{enumerate}
\item%
  An operation that initialises or sets the state defined by the optional
  operand.
\item%
  An operation that carries out the ordinary action on the state.
\end{enumerate}

\subsection{Register Files Local to Function Units}

Support for local operand and result register files in function units
implies that a trigger destination specifies a RISC operation (which
registers in input register file holds the values to be used and to which
registers in output register file the results will be written). Unit ``looks
like'' complete independent RISC processor.

\subsection{Hardware Loops}

A new type of control flow operation is introduced. From the point of view
of hardware implementation and instruction scheduling, loop operations are
is difficult to support. A loop operation defines a control flow transfer
that occurs several cycles away, in a place in the code not immediately
predictable from the position of the loop operation.

Hardware loops are even harder to implement in processors that support
exceptions.

\subsection{Architecturally visible alignment}

No alignment restrictions apply to instruction addresses. Instruction
addresses are anyway not handled by the Scheduler (which works with
instruction indices or references).

In any case, alignment of instructions is rejected for following reasons:
\begin{enumerate}
\item%
  It makes scheduling harder.
\item%
  It requires architectural support for encoding padding MAU's of
  instruction memory.
\item%
  It detracts from instruction compression methods based on variable-length
  encoding because it adds padding bytes.
\end{enumerate}

\subsection{Operation-level Abstraction for Program Representation}

The reasons to prefer a representation of the program at level of operation
rather than single moves are the following:
\begin{enumerate}
\item%
  Smaller graph representing program.
\item%
  Easier on unscheduled ``well-behaved'' code
\item%
  Reflects close relation of moves of particular operation.
\end{enumerate}

The decision is to prefer the move-level of abstraction. In practical
implementation, operations may still exist as helper objects that capture
properties shared by a group of moves referring to the same operation.  For
example, moves are picked up from the ready set based on the operation they
belong to.

\subsection{Support for Combined Compare and Logic Operation}

Direct support for combined compare and logic operations comes in two kinds:
3-input operations and operations with state (see blow).
%
In both cases, defining the operations is of little use unless the
scheduling algorithm is aware that such operations exist and is capable of
exploiting them effectively.

Support can be added as a custom operation subset or as a part of base
operation set. Only in the latter case can the operations be expected to be
exploited by any scheduling algorithms.

Both styles of combined operations are rejected. It seems much cleaner to
support these operations in the general framework of complex (compound)
operations (see Section~\ref{ch:pending:compound-operations}). A scheduling
algorithm that can handle compound operations can also exploit this class of
combined operations.

\paragraph{3-Input combined compare and logic operations.}
A combined operation of this kind takes three inputs: the two operands to be
compared and the intermediate result to be logically combined with the
result of the comparison:
\begin{verbatim}
  r1 -> eq-and.1, r2-> eq-and.2, b -> eq-and.3;
  eq-and.4 -> b;
\end{verbatim}

This style of operation is simple because it has no state and all inputs are
explicit. It makes it possible to combine an intermediate result stored in a
register different from the final result's. On the other hand, this type of
operations requires the value of the intermediate result to be ready earlier
and requires and extra move. This can result in longer critical paths if,
for example, the intermediate result is only defined in the same cycle in
which the result move of the combined operation.

\paragraph{Combined compare and logic operations with state.}
A combined operation of this kind does not overwrite the result (Boolean)
register. Instead, it applies a binary logic operation (\emph{and} or
\emph{or}) to the result register and overwrites the result:
\begin{verbatim}
  r1 -> eq-and.1, r2-> eq-and.2;
  eq-and.3 -> b;
\end{verbatim}

This style of operation is completely unusual (where the result implies also
which state is going to be used ---the result register itself is treated as
operation state). In practise, the result register resides outside the logic
that implements the combined operation, but from the scheduler point of view
it is as if it were part of the operation's state.

Practical implication of adding this kind of operations to base operation
set is that base operation set will include operations with state. When
added as a custom operation, such a feature is of no interest from a point
of view of this document (simple operation with state).
%
However, the scheduler is not expected to be aware of custom
operations.\note{%
  see PENDING \ref{ch:pending:custom-awareness}}
%
In that case, a separate pre-pass (or a plug-in module that can be invoked
during scheduling) must be responsible for exploiting this kind of
operations.

\paragraph{Conclusion.}
The 3-input alternative has the following advantages:
\begin{enumerate}
\item %
  More dataflow freedom (possibility to use different registers for
  intermediate and final result).
\item %
  Easier to analyse the code (no operation state).
\end{enumerate}
%
The operations with state have the following advantages:
\begin{enumerate}
\item %
  Efficiency: one extra move is avoided (the destination of the result move
  implies the state/input).
\item %
  Efficiency: in some cases, it is possible to trigger the operation
  earlier and reduce the critical path.
\end{enumerate}

\section{Ideas Related to Schedule Design}

\subsection{Concurrent Scheduling Runs and Configuration File}

The idea is to support parallel scheduling runs at level of configuration
file. Several scheduling runs on the same code in parallel can be specified
by entries of the configuration file. Each run can use different parameters
and the same scheduling algorithm or different algorithms for one or more
scheduling subtasks. The scheduler controller chooses the ``best'' result
from possible alternatives. The meaning of ``best'' is also to be defined
(smallest number of instructions, smallest average number of moves per
instruction, and so on).

This alternative is superseded by the hierarchical scheduling algorithms.
Also, it is not clear in this alternative what exactly is the task of the
scheduler controller and how does it work in the simpler case of a single
scheduling algorithm.

\subsection{Alternative Scheduling Runs and Configuration File}

The scheduling sub-modules and the code transformation passes, their order
or choice may be different for different scopes.
%
The idea is that the configuration may specify different settings of
submodules and passes for individual program scopes. For example, the kernel
of a program may be optimised more then the rest of program.

This alternative is superseded by the hierarchical scheduling algorithms,
limited to the scheduling algorithm proper (and its sub-modules). General
selection of code transformation passes for individual program scopes is not
supported at all (too complicated).

\subsection{Full-blown Scripting for Scheduler Configuration File}

Define a simple scripting language for the Scheduler configuration file. Such
language would entail simple control flow and hierarchical pass declarations
(that is, passes that invoke other passes). For example, a complex pass
could be declared as follows:
\begin{verbatim}
  pass_const_propagate [parameters]{
  while(do_const_propagate() == 1 ())
      pass_const_fold;
      pass_clear_dead_code;
  do
  }
\end{verbatim}
%
where \verb|do_const_propagate() == 1| indicates that something was changed
during the constant propagation pass. The constant folding and dead code
elimination passes are invoked internally by the constant propagation pass.

This type of functionality has been replaced by internal hierarchical
invocation by code analysis and transformation passes.

\subsection{Simple Text Format for Scheduler Configuration File}

The configuration file format consists of multiple declaration entries. Each
entry can span multiple physical lines and has the following format:
\begin{quote}\tt
  \parm{type}/\parm{task} = \parm{module-name}
      \verb|{| \parm{options}\ldots \verb|}|
\end{quote}
%
where \emph{type}, \emph{task} and \emph{module-name} indicate,
respectively, the name of the type of pass, the (sub)task of the pass and
the plug-in module that implements such subtask.
%
The task can be omitted. If not present, it is implied that all tasks of the
pass are implemented in the given plug-in module.

The module name can be replaced by another type/task pair. It means that the
declared task is implemented by a separate, independent plugin-module, but
is rather integrated together with another pass' task.

This alternative format was rejected in favour of an equivalent XML-based
format because XML can be verified and parsed automatically with existing
code base, it is (in this case) terse enough for manual editing, and is
easier to expand in the future.

\subsection{Formal Declarations of Module-Pass Dependencies}

The Scheduler models and enforces explicitly declared dependencies,
anti-dependencies and preconditions that define the minimal ordering
constraints between passes of the code transformation and scheduling chain.

The passes and cooperating plug-in modules declarations include explicit
dependencies. For example, a scheduling algorithm that requires certain data
structures to be built will declare this explicitly, and so will a pass that
is capable of constructing these data structures.

This framework controls correctness of the configuration file with respect
to modules that are mutually exclusive (incompatible in the same scheduling
chain), the number of parameters (additional input files, for example) for
specific modules, and so on.

This functionality is completely rejected because:
\begin{enumerate}
\item %
  It cannot anticipate dependencies with future modules and predict complex
  ordering conditions.
\item %
  It is fragile. Explicit declarations rely on user being respectful of this
  mechanism and declare dependencies correctly. This increases the chances
  of inconsistencies with respect to the actual module behaviour (forgotten
  or wrong dependencies).
\item %
  It is hard to maintain ordering constraints when passes are updated. An
  ordering can be rejected even if a newer module implementation tolerates
  it. Conversely, an ordering may be allowed even if the module has changed
  and is now dependent on something it didn't depend on.
\end{enumerate}

\section{Ideas Related to Overall System Design}

\subsection{Additional Instruction Encoding Algorithm}

The Scheduler accepts an instruction encoding algorithm as parameter of
scheduler configuration (see~\ref{ssec:files:config}). Such algorithm object
defines the size of each TTA instruction in the target architecture when the
instruction encoding is of variable length.

This solution has been rejected in favour of post-scheduling address
adjustment. The encoding algorithm is left unspecified during instruction
scheduling. From the scheduler point of view, instruction encoding is
assumed to be of fixed length, and instructions have constant size equal to
1 MAU.

\subsection{Configuration File as Backend Driver Input}

Instead of a simple configuration file for the Scheduler application, the
configuration file lists and launches any type of standalone applications
(code analysis and transformation, possibly even profiling runs of
simulation) related to code generation. In practise, such file is read by a
backend equivalent to the usual compiler drivers.
%
In this case, an actual backend driver application is needed. This
application should operate similarly to common compiler drivers (such as
GCC): scan the configuration file, select and run the compilation passes as
necessary. This would possibly allow future extensions, as for instance
specifying in configuration file which of available front end compilers to
use.

This idea is rejected in favour of an application-specific (Scheduler)
configuration file that control only dynamically linked modules that form
the code transformation and scheduling chain.



\chapter{IDEAS FOR FURTHER DEVELOPMENT}

% Ideas that are not part of the specifications yet but that might be
% required in future are listed here.

\begin{description}
\item[27.06.2005] -- scheduling without a target architecture definition.

  This could be an alternative use case scenario for scheduling with a given
  and fixed target architecture. Instead of reading an architecture
  definition and setting up the support structures for scheduling and
  resource assignment, the scheduler works assuming infinite (or loosely
  bound) resources. This scheduling mode could be useful to replace an
  arbitrary initial target architecture for design exploration or to
  estimate a practical ILP upper bound of the target application.
%
  ---A.~Cilio
\item[27.06.2005] -- feedback-guided scheduling.

  The scheduler could read back statistical information and the scheduling
  trace generated during a previous scheduling run, and use such information
  as a guidance for heuristics that control the scheduling algorithm.  This
  process could be iterated several times.
%
  This is an open issue for future research. A major challenge is probably
  finding stable and reliable correlations between feedback data and
  heuristic-based decisions.
%
  ---A.~Cilio
\item[04.07.2005] -- Re-schedule fully scheduled TTA code.

  In practise, this requirement leads to solving a problem (re-scheduling)
  that is radically different from ``pure scheduling'' of sequential code.
  Re-scheduling is a much harder problem.
%
  ---A.~Cilio

\item[20.07.2005] -- Vectorisation.

  Support for vector operations and vectorisation requires fundamental
  extensions to the framework. The Machine Object Model must be extended
  with the introduction of vector registers and vector function units.
  Possibly, also the Program Object Model needs to be extended (TPEF less
  likely so, thanks to its extensible annotation system). At very least,
  some standard annotation seems necessary in order to represent vector
  storage elements and operations in the target program. Finally, it's very
  likely that additional operations for packing and unpacking will be needed
  for convenient support of vectors.
%
  ---A.~Cilio, V.~Guzma
\item[19.08.2005] -- User control on scheduling at scope level.

  \note{DISCUSS: maybe allow user control at level of procedures, not
    scopes. Per-scope selection of algorithms and parameters could be
    automatically handled by sophisticated scheduling passes}
%
  The user can specify different code transformations and scheduling
  algorithms for different scopes of program (for example, per procedure or
  basic block). This feature is useful to perform the strongest
  optimisations on the most critical parts of the program (kernel of
  application).
\item[27.09.2005] -- Support for C++ demangling.

  It would be user-friendlier to support demangled names when using option
  `-p' of Scheduler. However, the mangling scheme is highly dependent on the
  frontend compiler, and it is not quite clear how such information could go
  down to Scheduler.
%
  ---A.~Cilio
\item[10.10.2005] -- Module-specific options in command line.

  Currently, the user needs to edit the scheduler configuration file
  whenever he wants to change options specific to a single pass module. To
  improve usability, the Scheduler could give the possibility to enter
  module options in its command line. These options would take priority over
  the options specified in the configuration file.

  A possible syntax: ``\verb|--modoption|
  \texttt{\parm{module}:\parm{option}}'', where \emph{module} is the module
  name, and \emph{option} is the option given to module. Double quotes may
  be needed when the option contains spaces, or \emph{modoption} may be
  repeated.
%
  ---A.~Mets�halme
\item[10.10.2005] -- Changing log level while the scheduler is running.

  Problem is how to let users interact to change the level. One possibility
  is to ``hijack'' the prompts from interaction points. For example, a
  yes/no prompt string from the scheduler could be: ``Proceed with register
  spilling? (Y/n/h/o)'', where `h' stands for help, and `o' brings to a menu
  for changing global scheduler options.
%
  ---V.~Guzma

\end{description}

\section{Concurrent Scheduling Runs}

The Scheduler uses heuristic-based algorithms that can be tuned by means of
parameters. Users may set the boundary values for those parameters.
Scheduler then create copies of a scope/program and run scheduling algorithm
on each copy with different heuristic parameters. After all schedules are
finished, scheduler/user will choose the most suitable result.

This would be possible to implement as distributed computation on several
computers.

This feature is internal to hierarchical scheduling algorithms. At a higher
level, external client applications (such as Explorer) can, in turn, handle
multiple distributed scheduling processes (each one with a different target
architecture or possibly a different parameter and pass set).

\section{Cyclic Scheduling}

Cyclic scheduling (software pipelining) is not likely to be implemented in
early version of scheduler.  However, it should be kept in mind while
designing modules that such a method is to be implemented in future.

It could be advisable to investigate different cyclic scheduling methods
other than the most popular and tested, which is Modulo Scheduling.

\section{Compression-Aware Scheduling}
\label{ch:future:compression}

Some code compression methods may produce better results if scheduler is
aware of their presence. A compression-aware scheduler may try to increase
the potential for compression and thus improve the performance of the
compressor.

An example is offered by compression methods based on instruction templates.
The scheduler can try to make similar instructions use the same move slots,
and therefore the same instruction template. This may reduce the number of
templates and the ``entropy'' of template occurrence, and therefore increase
the potential compression.

It seems that the two parts of the scheduler most affected by compression
effects are the selection heuristic and the resource assignment heuristic.
%
\note{left for future research}

Code compression may also have a visible, restricting effect on the target
architecture.  Unless special mechanism to escape compression is provided,
dictionary-based compression methods, once the dictionary is built, limit
quite heavily the programmability of a processor. For example, if the
dictionary is built at level of single move slots, a compressed program can
be modified only if all the moves of the modified code are in the dictionary
built for the original version of the code.

A compression-aware scheduler could be informed about what kind of moves are
allowed/not allowed so that it could compile code for the processor that
executing compressed instructions. The performance would decrease, but the
processor could still be programmed. This kind of a feedback mechanism would
be good to be supported.

This type of restriction, however, could be modelled as a generic set of
scheduling restrictions that are added to the base architecture definition
found in ADF. A possible design would be a separate file that specifies
architectural constraints not due to the hardware structure, much in the
same way ADF specify the architecture for a plain instruction encoding
algorithm.

This approach can become very complicated due to complex restriction
relations between moves and processor resources (mutual exclusive,
``2-out-of-3'' types of relations, etcetera).



\chapter{PENDING ISSUES}

\section{Scheduler application}

This section lists issues related to Scheduler application itself.

\subsection{Custom Operation Semantics}
\label{ch:pending:custom-awareness}

Few operations, as explained in Section~\ref{ssec:fundamental-operations},
are so fundamental that require that the scheduling algorithms be aware of
their precise behaviour.

It may occur that new custom operations (user-defined or anyway not handled
by front-end compiler) would profit considerably by a scheduling algorithm
that is aware of their behaviour. Some operations may not even be utilised
outside the scheduling algorithm. However, the scheduling algorithms cannot
anticipate behaviour of yet-to-define custom operations, nor we can ask a
toolset user to extend all existing scheduling algorithms with awareness of
the new custom operations.

Still, a (perhaps limited) form of modular scheduler-awareness of custom
operations would be very useful.

\section{Software algorithms}

This section lists open issues related to algorithms implemented in
Scheduler.

\subsection{Multiple Instances of Operations with State}
\label{ch:pending:state-operation}

In sequential code, a mechanism to specify when multiple operations with
internal state are related (that is, are expected to access the same
instance of operation state) and when not. Such mechanism must somehow
convey the information about the state usage implied by an operation with
state.

In addition, this mechanism should be completely independent and unrelated
with the properties of the target architecture. For example, identification
of state instances cannot be implemented as a partial assignment to
predefined function units of the target architecture.

\subsection{Operations with State and Function Calls}
\label{ch:pending:state-interprocedural}

The live range of operation state may cross scheduling barriers.  A
programmer may wish to use operation with state across function calls.
\emph{Example}:
\begin{quote}
OPSTATE - operation with state, scheduler assigns it to unit S1\\
CALL X  - function call, function X uses or changes operation with state,
scheduler assumes that it will use unit S1\\
OPSTATE     - another operation with state in caller procedure, uses
unit S1 if there was no dataflow break\\
...\\
// Another procedure that contains a call to X\\
OPSTATE     - procedure contains a call to X, and X is already marked to
be using unit S1, thus this operation also has to use unit S1\\
CALL X  - Already marked to be using unit S1\\
\end{quote}

Possible solutions to this problem are:
\begin{enumerate}
\item %
  Interprocedural propagation of state assignment to units. Once a unit is
  assigned to an operation with state in a procedure which does not
  initialise the state, the assignment must be back-propagated to every call
  site of the procedure. This approach leads to efficient scheduling, but
  may fail in some complicated cases (for example, when assignment to a
  given unit is impossible in one of the call sites because the same unit is
  already assigned to another operation with state).
\item %
  Additional support for operation state and restore across call.  This
  solution could be implemented as additional operation properties that mark
  operations that provide means to read state data from program variables
  (GPR's, immediate values or FU outputs) and to write it back to program
  variables. The problem is, how can the system --- should it at all? ---
  force users to define state initialiser and state saver support operations
  for every variable of a given operation with state. This solution will
  work always when the support save/restore operations are available, but
  such operations probably cannot be assumed to be always present.
\end{enumerate}

It seems that a combinations of both solutions is necessary and, in the
worse cases, even this combination cannot warrant scheduling. A simpler and
rougher solution is to pre-assign operation state across function calls
before scheduling.

\section{Supported Hardware}

This section lists open issues related to hardware (FU's) supported by
Scheduler.

\subsection{Compound (Complex) Operations}
\label{ch:pending:compound-operations}

Compound operations are operation patterns made of two or more (base or even
custom) operations. Such operations do not exist in the TTA operation
repertoire, but are synthesised by TCE toolset in combination with ADF
data.\note{%
  ADF support for compound operations not yet defined}

Scheduling algorithms or pre-scheduling passes (such as peephole
optimisation passes) may be aware and take advantage of available compound
operations.

\subsection{Autoincrement Load and Store Operations}

In its most efficient and specialised form, an auto-increment operation
(load, for example) looks like this:
\begin{quote}
\begin{verbatim}
ldwi.1 -> r.5
\end{verbatim}
\end{quote}

This one-move operation will write the value loaded from memory by previous
load operation into register `r.5', and at the same time will (1) begin to
update the internal memory address for next load operation, adding a fixed
increment to it, (2) initiate a new load request from current internal
memory address. Alternatively, steps (1) and (2) above can be swapped,
creating one more stage in the operation: (a) increment, (b) initiate load,
(c) read the loaded value.

This type of operations requires support for operations with state (for the
internal memory address) and trigger-on-result.
%
\note{left for future development}
%
Since optional operands are not supported (rejected feature), this type of
operations requires a special ``helper'' operation for setting the internal
state and trigger the first operation, so that the result is ready for the
first triggering result move:
\begin{quote}
\begin{verbatim}
0x100 -> sbai.1 # set auto-incremented base address
\end{verbatim}
\end{quote}

In a more general version of the auto-increment load of above example, the
increment can be a parameter (and treated as internal operation state) set
by a special ``helper'' operation:
\begin{quote}
\begin{verbatim}
2 -> slinc.1 # set auto-increment value for load operation
\end{verbatim}
\end{quote}

In this version, the auto-increment operations are harder to analyse, but
may be very useful in loops that access memory with regular but not fixed
patterns.
%
The increment value may change within a chain of auto-increment load
operations.\footnote{
%
  Many applications access memory in regular pattern more complex then
  simple increment. However, Steven Pekarich suggested that use of
  auto-increment load-store is less beneficial than expected.}

A less efficient version of the auto-increment load could avoid
trigger-on-result and require an explicit operand move to specify the
increment of the next address.
%
There are two versions, identically-looking but very different, of this
operation: pre-increment and post-increment.

With a pre-increment, the
operation:
\begin{quote}
\begin{verbatim}
2 -> ldwi.1
ldwi.2 -> r.5 
\end{verbatim}
\end{quote}
is equivalent to a compound operation (see
Section~\ref{ch:pending:compound-operations}): an add operation that
computes the memory address followed by a load operation. This version of
auto-increment load defeats the main purpose of auto-increment: to pre-load
the next memory word so that its latency can be hidden.

With post-increment, the first move triggers an increment of the internal
memory address and, after the new address is computed, a load from memory,
while the result move reads the result of \emph{previous} load-increment
operation. In fact, the two moves are not related from dataflow point of
view, and their order can even be reversed:
\begin{quote}
\begin{verbatim}
ldwi.2 -> r.5 
2 -> ldwi.1
\end{verbatim}
\end{quote}

Any of the versions of auto-increment operations shown above cannot be
replaced by a standard load-store unit and a custom function unit that
modifies or computes the memory address.\footnote{
%
  As suggested by Jarmo Takala.}
%
The reason is simple: if the address is computed externally, it must be move
through the programmable transport network, so there is no significant gain
in term of efficiency (power consumption, reduced number of moves, reduced
register pressure).

Compare:
\begin{quote}
\begin{verbatim}
0x300 -> ldw-base.1 # set base address to increment (state)
. . .
0 -> ldw-incr.1     # dummy trigger
ldw-incr.1 -> r3    # triggering result
\end{verbatim}
\end{quote}
with the less efficient:
\begin{quote}
\begin{verbatim}
 0x300 -> incr-base.1 # set base address to increment (state)
 0 -> incr.1          # dummy trigger
 incr.2 -> ldw.1
 ldw.2 -> r3
\end{verbatim}
\end{quote}
which costs at least one more move, and maybe two (if software bypassing is
not applied). (Note that in both versions the dummy trigger is avoided if
trigger-on-result is applied.) The second version has all the requirements
of the load-increment version (operation state, a separate base-setting
operation, trigger-on-result if applicable) but little of its benefits.

\subsection{Custom Operations Cannot be Affected by Base Operations}

This rule applies to operations with state\footnote{This is design
decision in a way.}. The rule needs to be also
extended other way around, so no base operations can be affected by custom
operations. Reason is that support for combined compare and logic
operations will add operations with state to basic operation set.

% ------------------------------------------------------------------------

\appendix


\chapter{OPTIMISATIONS}
\label{app:optimisations}

This chapter describes the set of code-improving transformations
(``optimisations'' in short) for which the Scheduler provides at leas one
implementation. Some of these optimisations may be implemented as separate
independent passes. Since many optimisations benefit from or even require
other optimisations in order to deliver any code improvement, it is not
unusual that several optimisations are grouped into a single pass.

\section{Loop Peeling}
\label{sec:opt:loop-peel}

Loop peeling is a code transformation that replaces one or more iterations
from a loop with an equal number of copies of the loop body. The code thus
created form a loop preamble (if the first iterations are peeled) or a loop
post-amble (if the last iterations are peeled).

\paragraph{Benefit:}
More aggressive optimisations can be applied on the peeled copies of the
loop body (acyclic code). Under certain conditions (minimum number of
iterations guaranteed) the control flow between the peeled copies of the
loop body can be removed.

\paragraph{Drawbacks:}
Code increase, higher resource pressure.

\paragraph{Requires or benefits from:}
Loop pre-conditioning.

\section{Loop Pre-conditioning}
\label{sec:opt:loop-condition}

\paragraph{Benefit:}

\paragraph{Drawbacks:}

\paragraph{Required by or useful for:}
Loop peeling.

\section{Loop Unrolling}
\label{sec:opt:loop-unroll}

\paragraph{Benefit:}

\paragraph{Drawbacks:}

\paragraph{Requires or benefits from:}
Loop pre-conditioning.

\section{Loop Merging}
\label{sec:opt:loop-merge}

\section{Loop Fission}
\label{sec:opt:loop-fission}

\section{Dead Code Elimination}
\label{sec:opt:dead-code}

\paragraph{Benefit:}
Removed potential execution of useless code and wasted resource usage (by
predication). Code size reduction. Reduced instruction cache pressure.

\paragraph{Drawbacks:}
This is one of the few architecture-independent and nearly-always profitable
optimisations. However, in some cases, the rearrangement of instructions
could increase the instruction cache miss ration.

\paragraph{Requires or benefits from:}
Many optimisations create dead code as a side effect, and therefore require
dead code elimination to fully express their benefit.

\section{Function inlining}
\label{sec:opt:inlining}

\paragraph{Benefit:}
Elimination of the function call overhead (control transfers, stack pointer
setup, argument passing. Potential reduction of register save-restore
overhead.

\paragraph{Drawbacks:}
Increased code size and, on rare occasions, even cycle count increase.
Higher resource pressure, especially register pressure, since inlined code
does not benefit from the save-restore convention.

\paragraph{Useful for:}
Inlined code is often amenable to new and increasingly effective
optimisation opportunities, such as constant folding and propagation, dead
code elimination, etc. For this reason, it is usually beneficial to apply
function inlining before other optimisations (or repeat the optimisations
after function inlining).

\section{Constant Propagation}
\label{sec:opt:const-propagate}

\paragraph{Benefit:}
Reduced register pressure.

\paragraph{Drawbacks:}
Since it replaces usage of a resource such as a GPR with an immediate,
constant propagation results in less efficient resource utilization and
slower code in every situation in which immediate-related resources are more
critical than the alternative.

\paragraph{Required by or useful for:}
Constant folding.

\section{Constant Folding}
\label{sec:opt:const-fold}

\paragraph{Benefit:}
Elimination of an operation and all associated resources. Reduced register
pressure when the result of the operation is not bypassed.

\paragraph{Drawbacks:}
Since it replaces usage of a function unit with an immediate, constant
folding results in less efficient resource utilization and slower code in
those (fairly rare) situations in which immediate-related resources are more
critical than all the resources taken up by the operation that it removes.

\paragraph{Required by or useful for:}
Constant propagation.

\section{Value Numbering}
\label{sec:opt:value-numbering}

Value numbering is the process of assigning a unique identifier (or
``number'') to each variable (or live range) of the program.

Value numbering is not, properly speaking, an optimising code
transformation. If the program representation allows to mark variables or
GPR's with a unique identifier, value numbering may not even modify the
code.

\paragraph{Benefit:}
After value numbering every variable or live range of the program is
uniquely identified, and the code is more amenable to certain dataflow
analysis and code transformations.

\paragraph{Drawbacks:}
If value numbering is applied by actually assigning a different register to
each variable, it must be undone. It unlikely that a processor can even run
a program with such high register demands as those required by value
numbering.

\paragraph{Required by or useful for:}
SSA analysis and conversion, nearly any dataflow analysis.

\section{SSA Form Conversion}
\label{sec:opt:ssa-conversion}

Static Single Assignment (SSA) form conversion consists in two actions: (1)
assigning a new, unique variable or value identifier to each point in the
program were a program variable is defined; (2) inserting a merge node (the
phi-function) wherever, due to control flow merge points, two or more values
coming from different definitions come to be used as the same variable. The
new value produced by an instance of the phi function is also assigned a new
unique identifier.

\paragraph{Benefit:}
After SSA conversion every value used in the program comes from one and only
one definition point (possibly a phi node), and the code is more amenable to
certain dataflow analysis and code transformations.

\paragraph{Drawbacks:}
SSA conversion creates a large number of unique values (in fact, the maximum
possible number in any give program). This increased the complexity of the
register allocation problem. In some cases, phi nodes cannot be absorbed by
register assignment, and result in code overhead (register copies).

\section{Name of Optimisation}
\label{sec:opt:name-optimisation}

\paragraph{Benefit:}

\paragraph{Drawbacks:}

\paragraph{Useful for:}


\chapter {SOFTWARE PIPELINING}
\label{ap:SW-pipelining}

Normally, the operation reordering performed by the instruction scheduler
is limited to forward control flow paths. In other words, the backward
control flow paths of a loop are scheduling barriers.

Software pipelining is method for scheduling instructions of loops without
this barrier.  When effectively applied, software pipelining delivers
results similar to loop unrolling without the corresponding increase in code
size.

Instruction scheduling algorithms that implement software pipelining are
called ``cyclic'' and have a strong influence on several parts of scheduler.
One of such cyclic scheduling algorithms is \emph{Modulo
  Scheduling}~\cite{Rau94}. In the rest of this section only Modulo
Scheduling will be considered.

First of all, a modulo scheduler needs to be informed about an initiation
interval, that is, the number of cycles between the start of two consecutive
iterations of the same loop. This requires a pre-scheduling analysis phase
that estimates the initial, tentative initiation interval.\footnote{
%
  The initiation interval computed is based on data flow in loop and depends
  on the latency of operations. However, operation latency depends on the
  function unit assigned to the operation. This means that resource
  assignment affects the initiation interval.}

%% REMOVED UNTIL MISS SYNTAX MAKES PEACE WITH VLADO AGAIN
%% ------------------------------------------------------
% A cyclic scheduling algorithm manages resource assignment and bookkeeping
% in a special way. It must control the resource check and test if the
% result will not appear after end of basic block [[clarify]]. In such a
% case, with use of modulo operation and initiation interval, the new
% position of result has to be calculated, in following iteration of loop.

The resource manager may not need to know about initiation interval. The
modulo scheduler may do the necessary tests.
%
\note{VERIFY}

In cyclic scheduler, the consumer of a definition may appear before (in an
earlier cycle) than the producer. The scheduler must take this into account.
Also, in cyclic scheduling an operation may be scheduled before the
operations it depends on. In this case, the scheduler needs to test if the
cycle in which the definition is scheduled does not exceed cycle of earliest
consumer. Simply speaking, while in acyclic scheduling algorithms, we deal
with data dependencies that go ``forward'', in modulo scheduler we also
have loop carried dependencies, and we need to deal with loop carried
operation latencies.


%  References are generated with BibTeX from a bibtex file.
\bibliographystyle{alpha}
\cleardoublepage
%% Equivalent to a chapter in the table of contents
\addcontentsline{toc}{chapter}{BIBLIOGRAPHY}
\bibliography{Bibliography}

\end{document}

%%% Local Variables:
%%% mode: latex
%%% mode: auto-fill
%%% TeX-master: t
%%% End:
