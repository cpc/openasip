\documentclass[twoside]{tce}
\usepackage{pslatex}
\begin{document}
\author{Andrea Cilio}
\title{TTA Processor Simulator}
\ver{0.11.2}
\firstday{9.12.2003}
\lastday{11.08.2005}
% id number in S- sequence
\docnum{008}
% draft/complete/committed
\state{draft}
\maketitle

% highlighted style for parameters/variables or otherwise non-fixed parts
% of a command syntax or of lines of text displayed by the application
\newcommand{\parm}[1]{\textsl{#1\/}}
% command to print the type of data fields of data bases
\newcommand{\ftyp}[1]{(\emph{#1\/})}
% standard item vertical separation; temporarily overridden in some lists
\newlength{\stditemsep}
\setlength{\stditemsep}{\itemsep}
% standard paragraph vertical separation; temporarily overridden in some lists
\newlength{\stdparskip}
\setlength{\stdparskip}{\parskip}

\chapter*{Version History}

\begin{HistoryTable}

 0.1  & 09.12.2003 & A. Cilio        &
 First draft. 1, 2 (complete); 3, 4, 5 (parts).\\

 0.2  & 18.12.2003 & A. Cilio        &
 Reorganisation according to comments by P.J��skel�inen.\\

 0.3  & 30.12.2003 & A. Cilio        &
 Simulation statistics, command line options. 4.3.3, 4.5, 5.1.1.\\

 0.4  & 12.01.2004 & A. Cilio        &
 Simulation control language; 4.6 (part).\\

 0.5  & 20.01.2004 & A. Cilio        &
 Complete revision; added typographic conventions.\\

 0.6  & 27.01.2004 & A. Cilio        &
 Corrections after review by P. J��skel�inen.\\

 0.7  & 02.02.2004 & A. Cilio        &
 Modification after further comments by P. J��skel�inen. Removed simulation
 statistics.\\

 0.8  & 25.02.2004 & A. Cilio & Flag to ignore resource conflicts. New `info'
 commands to inspect processor state. List of external data files.\\

 0.9      & 08.03.2004 & A. Cilio &
 Pending issue about memory implementation.\\

 0.10     & 06.04.2004 & A. Cilio        &
 Changed stack initialisation, added initialisation script option. POSIX
 pathname matching. Eliminated `set' subcommands. Added `help' command.\\

 0.10.1   & 21.10.2004 & L. Laasonen &
 Renamed MDF to ADF. \\

 0.10.2   & 20.12.2004 & P. J��skel�inen &
 Renamed `file' command to `prog'. Added `mach' and `conf' commands. \\

 0.10.3   & 02.05.2005 & P. J��skel�inen &
 Renamed commands to avoid clashes with Tcl language or other
 implementation issues.\\

 0.10.4   & 26.07.2005 & P. J��skel�inen &
 Renamed `break' command to `bp'.\\

 0.11     & 04.08.2005 & P. J��skel�inen &
 Added bus trace. Minor updates. Rejected canned command sequences. \\

 0.11.1   & 04.08.2005 & A. Cilio &
 Corrected spelling, minor text formatting.\\
 
 0.11.2   & 11.08.2005 & P. J��skel�inen &
 Updates commands that deal with command history logs.\\

\end{HistoryTable}

% Table of contents
\tableofcontents

% Document text
\chapter{INTRODUCTION}

\section{Purpose}

%  Describe the purpose of the document - example:
%  This document describes the project "TTA Codesign Environment".  Its
%  purpose is to give an overview of the proposed TTA Codesign Environment,
%  its main features and requirements, and a summary of the project
%  motivations and goals.

This document describes the target architecture simulator of the TCE
toolset.  This document contains the functional specifications of the
simulator.

\section{Product Idea}

% (PJ: product)

%  Describe the project motivation in few words and the context in which
%  this product should have a reason to exist - see TCE Project Plan for an
%  example

The target architecture simulator enables the user to evaluate how a given
target processor runs a given TTA application.  The Simulator reproduces the
cycle-accurate behaviour of application and target processor, and the
precise \emph{architectural} state of the processor cycle by cycle, but does
not model implementation details or even the instruction memory image.  The
Simulator is an essential component of the TCE toolset because it provides
several critical functionalities.

On the application side, simulation is needed to:
\begin{itemize}
\item Verify application correctness.
\item Profile application.
\item Analyse application performance and characteristics.
\end{itemize}
To ensure application correctness at any stage of the compilation and
scheduling process, the simulator must be capable of simulating both
unscheduled intermediate code (for a generic target architecture) and code
scheduled for a specific target architecture.

On the target architecture side, simulation is needed to:
\begin{itemize}
\item Evaluate target processor performance.
\item Analyse target processor resource usage.
\end{itemize}
Evaluation of target processor is essential in order to guide the process of
finding an optimal target processor for a given cost-performance trade-off.

The processor model used by the Simulator is purely architectural and thus
implementation-independent.  It does not rely on a model of a possible
hardware implementation.

% Architecture simulation is useful before doing real HW model
% simulation, because it's faster and doesn't require to have an
% implementation of the target processor ready.

\section{Product Overview}

% (PJ: product + overview)

% This section describes as briefly as possible the system that implements
% the functionality wanted.

%Only main features/requirements of the product, no details.

Several different standalone applications may implement the target
architecture simulator.  For example, a simulator application that is
controlled by interactive command line interface and an application
controlled by a graphical user interface.  All applications share the same
target processor model and the same core simulation engine.

In addition to standalone applications, the core simulation engine of the
Simulator may be integrated into an executable model of a larger system (in
practice, may be linked with an external simulator) to provide a model of
the TTA processor for more complex system models.  This external simulator
represents the environment of the TTA processor being simulated.  The core
simulation engine design and its interface are described in the Simulator
Design Document~\cite{SimulatorModDesign}.  The specification of the
system-level simulator is out of the scope of this document.


The rest of this document treats only standalone applications, and in
particular on the CLI-based Simulator.

Standalone applications receive as input data the TTA application to
simulate and (optionally) the target processor to be simulated.  The
CLI-based application accepts also a number of command-line options.  The
GUI-based application, instead, reads and sets up the simulator
configuration by reading a configuration data file.

The target architecture simulator can produce or update several files as
result of a run:
\begin{itemize}
\item Any output file created by the simulated TTA application.
\item Execution trace, machine state snapshot.
\item Profiling information.
\end{itemize}

The Simulator offers the capabilities of a debugger.  It allows to
interactively control the simulation process, and to inspect and modify the
state of the target processor on any cycle of the TTA application execution.

\section{Definitions}

\begin{description}
\item[Application]%
  A standalone executable program that provides services for users of the
  TCE toolset. It may be controlled by a command-line interface or by a
  graphical user interface, or via a driver application that integrates a
  group of related applications.
\item[Multicast]%
  A special type of data transport whereby the signal on a transport bus is
  delivered simultaneously to more than one destination.
\item[Simulator]%
  The target architecture simulator of the TTA templated architecture.
\end{description}

\section{Acronyms and Abbreviations}

\begin{table}[htb]
\begin{center}
\begin{tabular}{p{0.15\textwidth}p{0.85\textwidth}}
ADF   & Architecture Definition File. \\
API   & Application Programming Interface.\\
CLI   & Command Line Interface.\\
GDB   & GNU Source-Level Debugger. \\
GPR   & General-Purpose Register. \\
GUI   & Graphical User Interface.\\
ILP   & Instruction-Level Parallelism (or Processor).\\
MAU   & Minimum Addressable Unit (of a memory address space).\\
SPR   & Stack Pointer Register. \\
TCE   & TTA Codesign Environment. \\
TPEF  & TTA Program Exchange Format. \\
TTA   & Transport Triggered Architectures. \\
\end{tabular}
\end{center}
\end{table}

\section{Typographic Conventions Used in This Document}

\begin{center}
\begin{tabular}{|p{0.30\textwidth}|p{0.63\textwidth}|}
\hline
\textbf{Style} &\textbf{Purpose}\\
\hline
\emph{italic}      & command and parameter names in running text\\
\hline
[brackets]�         & references\\
\hline
\texttt{monotype}   & messages displayed literally on screen\\
\hline
\parm{\texttt{monotype-slanted}} & variable part of messages displayed
                      on screen\\
\hline
`single quoted'     & keywords, file names or literal strings in running
                       text\\
\hline
`\emph{single quoted, italic}' & variable part of file names or other
                      strings used in the application, in running text\\
\hline
\end{tabular}
\end{center}


\chapter{USER DESCRIPTION}

\section{User Profile}

%% Remove this section (not needed) for functional specifications of
%% modules; may be needed for applications or subsystems.

%  What's the user like? Student? Researcher? Professional? Expertise in
%  hardware or software expert? How much skilled?

% PEKKA:
% Describe who are the or users of the module/application.
% The users can be "a software engineer" or "a researcher that is developing
% new scheduling algorithms".

The typical users of the Simulator are \emph{application developers},
\emph{architecture designers} and \emph{researchers of TTA concept}.

Application developers are concerned with verifying the correctness of the
application, pinpointing its critical spots (profiling) and ultimately
fine-tuning the code so that it runs as fast as possible on the target
processor.

Architecture designers are concerned with evaluating the performance of the
target processor architecture so that it runs one or more applications as
fast as possible.  The architecture designer is concerned with pinpointing
and eliminating hardware resource bottlenecks (by adding resources) and
ensuring that the target processor uses available resources efficiently (by
removing underutilised, noncritical resources).

Researchers are interested not only in evaluating performance and
architecture, but also in exploring new scheduling algorithms, new code
optimisations or novel architecture features.  This type of user may need to
modify the Simulator code and extend its capabilities.  For example, a
researcher may implement new data processing tools to extract required
information from Simulation statistics.

\section{Key User Needs}

%% Remove this section if the description in INTRODUCTION/Product Idea
%% is sufficient (usually it is for functional specifications of modules)

Verification of application correctness.  Evaluation of applications basic
characteristics (analysis).  Application profiling and execution tracing.
Evaluation of application performance on a given target processor.
Measurement of target processor performance and efficiency (resource
utilisation). Debugging: step-by-step simulation, examination of machine
state at a given cycle, tracking of program execution.

In addition to the human user, the Simulator produces data that is required
for other TCE applications, in particular:
\begin{enumerate}
\item %
  Profiling data guide the heuristics used by code analysis and
  transformation passes as well as the instruction scheduler.
\item %
  Performance measurements provide a feedback for architecture space
  explorer.
\item %
  Utilisation measurements provide data for applications that estimate power
  consumption.
\end{enumerate}


\chapter{PRODUCT OVERVIEW}

\section{Product Perspective}

% (external connections)

%  High-level description of the product with focus on its communication
%  with external products/modules. What external modules does it depend
%  from?  What is provided in the product and what is required and assumed
%  available as external service?

% PEKKA:
% This section describes the environment where the product belongs to, for
% example what the module is part of. Also client relationships to other
% modules or applications are described in detail.
%
% Example:
%


% This module is part of the software subsystem of the TCE. Program Model
% constructs itself using this module. Applications that inspect or modify
% binary files (such as the linker) are direct clients of the module.
%
% This module uses the binary stream module to load and store binary files
% from/to disk. Uses Reference Manager to handle references.
%
% This module is used by different binary file handling tools of the
% project. Example of a such tool is TPEFDumper \cite{tpefdumper-fspecs},
% which prints useful information on the binary files.

TTA target processor simulation is one of the major functionalities of the
TCE software subsystem.  Several different applications can implement this
functionality.

Depending on the application that implements the Simulator, a different
interface is visible to the user.  Currently, we foresee two simulator
applications: one controlled by an interactive command line interface, and
one controlled by a GUI, henceforth called CLI-based and GUI-based
simulator, respectively.

The Simulator reads two fundamental input files: the TTA application (in a
TPEF file, described in~\cite{TPEF-specs}) and the architecture definition
file~\cite{ADF-specs}.

The Simulator may optionally read several other input files:
\begin{enumerate}
\item Configuration file (GUI-based Simulator only).
\item Debugging session.
\item Machine state snapshot.
\end{enumerate}

The contents and the format of these files are described in the following
sections.

In addition to the output (if any) from the TTA application being simulated,
the Simulator can produce several output files.

The CLI-based simulator recognises a set of command line options for
configuring the simulation environment, controlling the simulation process
and the output produced.  These options are all described in
Section~\ref{sec:usage}.

The CLI-based simulator can run in interactive mode.  In this case, the
simulation process can be controlled by means of a special language,
described in Section~\ref{sec:debug}

\section{Product Position Statement}

%  In few words, almost a slogan, the reason for this product to exist
To provide cycle-accurate simulation of TTA target processors that can be
designed with TCE.  To allow simulation tracing, inspection of machine
state, and simulation-based analysis of target processors and TTA
applications.

\section{Summary of Capabilities}

% (advantages)

The following capabilities motivate the Simulator requirements specified in
this document.

\begin{center}
\begin{longtable}{@{}p{0.45\textwidth}@{}p{0.55\textwidth}@{}}
  User Benefit                 &       Supporting Feature \\
\hline
processor performance analysis &  simulation statistics \\
processor utilisation          &  simulation statistics \\
execution profiling,
application analysis           &  profile section management,
                                  interactive disassemble \\
cycle-by-cycle control of
simulation                     &  interactive CLI language \\
access to machine state        &  interactive CLI language \\
off-line, complex  simulation
analysis                       &  simulation trace file generation \\
stopping and restarting
simulation                     &  machine state snapshot file generation \\
\end{longtable}
\end{center}

\section{Assumptions and Dependencies}

% (portability)
% (PJ: design constraints)

%  Assumed environment, third-party applications or library assumed to be
%  present. External services (like OS services) that are required.
In addition to the general dependencies specified in~\cite{ProjectPlan}, the
graphical front-end of the Simulator requires the \emph{wxWindows} graphical
library.  This library is used as compatibility layer to avoid direct
dependency from more primitive graphical libraries provided by the host
operating system, such as GTK and X.

The core module and the memory module of the Simulator are designed to work
with SystemC models of the entire system.  However, any system-level
modelling language that allows to call C functions of a linked software
module is expected to work.

Where bases of data must be accessible to SQL queries, the C library
SQLite~\cite{SQLite-home} will be used.  This library supports most of SQL92
standard, but not all.  DBMS-specific SQL commands are not supported (at
least, support for them is not guaranteed).

\chapter{PRODUCT FEATURES}
\label{ch:product-features}
% (PJ:functionality)

% PEKKA:

% This chapter describes the functionality of the product in detail. In case
% modules, the required client interfaces are defined in broad manner. These
% are the minimum interfaces the object should provide for the client. The
% format or argument list of methods should not be in detailed level, that's
% a thing that should be in the design notes instead.

% In case of an application, all command line arguments and user interface
% screens (GUI) are described in detail.  Also, all tasks the module or tool
% must be able to accomplish are described here.

% Example:
%
% This module must be able to read a.out and TPEF completely. No other
% binary reading support is required at this point. Only writing of TPEF
% format is required.

%  Organise in sections.  Dedicate one section for each main feature.
%  For example: program input/output modules (what it's usually called
%  Binary Handling Module) could be divided in four parts:
%  - common features
%  - reader features
%  - writer features
%  - reference management
%% Some modules may be so simple that a single section is sufficient.

This chapter contains the detailed functional specifications of the TTA
processor simulator.  Unless otherwise stated, these specifications apply
only to the CLI-based Simulator, henceforth also called simply
``Simulator''.

\section{Overview}

% In some situations (further specified in following
% sections, the ADF name does not need to be given.
%\begin{enumerate}
%\item %
%  The TPEF file contains a copy of the ADF as a file section.
%\item %
%  The TPEF file contains a reference to the external ADF.
%\item %
%  The information contained in ADF is not needed.
%\end{enumerate}

The Simulator provides a user interface to:
\begin{enumerate}
\item Setup the simulation environment.
\item Configure the amount of information collected during simulation.
\item Control the simulation process.
\end{enumerate}

In addition to simple, non-interactive control of the simulation process by
means of the command line options, the Simulator provides advanced
interactive control, basically the same functionality offered by a debugger.
The Simulator makes it possible to control simulation of single TTA
instructions step by step, and to inspect the state of all relevant building
blocks of the target processor at any cycle of the simulation.  Also the
state of memory can be inspected.

To gain advanced control of the simulation process and access to the machine
state, the Simulator must be run in \emph{debug mode}.  When in debug mode,
the Simulator enters the interactive mode and prompts the user for commands.
These commands control the simulation much in the same way a CLI-based
debugger such as GNU GDB controls a running application.  The only
difference is that the execution, in our case, is only simulated by the host
computer system.  The language recognised by Simulator when running
interactively in debug mode is specified in Section~\ref{sec:debug}.

The level of tracing detail and the amount of simulation statistics
collected during simulation can be configured by a number of command line
options.  Similarly, the generation and management of profiling data is
controlled by means of command line options.

\section{Setup of Simulation Environment}

The Simulator sets up the environment in which the simulation of the TTA
application takes place.  This environment consists of the following data:

\begin{enumerate}
\item Program command line arguments and options.
\item Stack pointer register.
\item Pre-initialised stack area. 
\end{enumerate}

Simulated TTA applications can take arguments and options from the command
line.  The Simulator allows to pass strings of arguments and options
to the application.  It is also possible to pass several arguments or
options in a single string.  The string must be surrounded by single or
double quotes.

The Stack Pointer Register (SPR) is a general purpose register (GPR) that is
dedicated, by convention, to keep the address of the current stack frame.
For more information on the architecture conventions and the structure of
the stack, see [[architecture specs]].

The SPR is not initialised directly by the Simulator: the TTA program is
responsible for initialising SPR before using it.  However, for flexibility,
the TTA program can load the initial value of the stack pointer from a
memory location.  The address of this location is predefined and hard-coded
in the TTA program.
% WHO defined this location is a topic for the architecture document
The contents of the memory location (which in the actual processor is
initialised by the external environment, for example a host processor,
before the TTA processor starts running) is computed by the Simulator or
given as a command line option.

The stack can be initialised by means of commands stored in a script file
that may be optionally read before the simulation is started.  The
initialisation script is written in Simulator control language (see
Section~\ref{sec:debug} for an overview of the language,
Section~\ref{ssec:debug-running} for specification of commands that modify
the TTA program environment).  The stack initialisation data allows to
communicate information from the external environment to the TTA processor.
On personal computer operating systems, the environment area contains a
sequence of NULL-terminated strings of characters.

The maximum size of the stack is not defined; it is implied by the physical
boundaries of the address space where the stack is located and the address
of the top of the stack.  For example: given an address space with minimum
and maximum addresses 0x1000 and 0x4000, respectively, and a stack with
hard-coded bottom address 0x3F00 and an area of initialised external data of
0x00A0 MAU's, the maximum size of the stack is 0x2F00 MAU's, and the
available storage is 0x2E60 MAU's (0x2F00 less the initialised area).

The Simulator needs to know%
\note{better would be to store the memory location as auxiliary information
  in the TPEF file and have Simulator read it}
%
the address of the memory location from which the SPR initial value is
loaded, but not the address space of such location, because that is implied
by (and hard-coded within) the program code.  See
Section~\ref{ch:pending:stack} for issues related to defining the address
space of the stack.

\section{Configure Amount of Data Collected During Simulation}

During the simulation process, the Simulator collects a vast amount of
data.  These data fall into three categories:
\begin{enumerate}
\item Execution trace data.
\item Execution profiling data.
\item Processor utilization data.
\end{enumerate}

The execution trace is a temporally ordered log of the activities performed
during the simulation.  Profiling data consists of the execution count of
each instruction of the input TTA program.  Processor utilization data
provides statistics of utilization of different components and operations
of the simulated processor.

\subsection{Execution Trace Data}

The Simulator allows to control which type of activities are traced, the
start and the end cycle of tracing, the sampling frequency of tracing.  For
a complete description of the data stored in an execution trace, see
Section~\ref{ssec:exec-trace}.  The trace file is called
`\emph{application}.\emph{s}-\emph{e}.trace'.  Where \emph{s} is the start
cycle and \emph{e} the end cycle being traced; the end cycle can be omitted.

\subsection{Profiling Data}

Unlike other types of data collected by the simulator, the profiling data of
a given TTA program run is not stored in a separate file; instead, it is
stored in the input TPEF file of the program.

Storing of profiling data can be disabled.  If enabled, the data can be
stored into the input TPEF file in several ways:
\begin{enumerate}
\item %
  Accumulate execution counts with existing profiles (if any) from previous
  simulation runs.
\item %
  Add a new profile data section for this specific simulation run.
\item %
  Overwrite any pre-existing profile sections.
\end{enumerate}

The entire program is profiled: it is not possible to include/exclude parts
of the program, unlike for other types of data collected during simulation
(see Section~\ref{sec:control} for details).

\subsection{Processor Utilization Data}

Utilization data is not stored in any file, but it is available through a
application programming interface after simulation. The API can be 
used, for example, by Estimator to get the information it needs to
calculate the processor cost estimates.

\section{Control Simulation Process}
\label{sec:control}

The Simulator allows a certain degree of non-interactive control of the
simulation process by means of command line options.

\paragraph{Machine state snapshots.}
The Simulator is capable of stopping and resuming the simulation of a TTA
application at any given cycle.  This capability allows to avoid infinite or
exceedingly lengthy simulation runs; it is very useful, when the simulation
crashes, to trace the cause of the problem without starting the simulation
all over.

The capability of resuming simulation is made possible by \emph{machine
state snapshots}.  A snapshot contains the complete machine state at a given
cycle of execution of the simulated TTA application.  Snapshots are stored
in data files and can be taken at a given cycle, or, after a starting cycle,
at given intervals.  After reading a machine state snapshot file, the
Simulator can resume simulation immediately as if the whole simulation had
been run again up to that cycle.

The snapshot file is called `\emph{application}.state.\emph{cycle}',
where \emph{application} is the name of the TTA application being simulated
and \emph{cycle} is the real-time cycle at which the snapshot is taken.  See
Section~\ref{ssec:sim-snapshot} for a description of the requirements on
this data format.

\begin{quote}\textbf{Note:}
   The contents of the snapshot file uniquely identify the name of the
   machine file, which is also stored (as reference or in-line copy) in the
   application TPEF file, if the application is scheduled.
\end{quote}

By default, the simulator ignores any snapshot files and starts simulation
from cycle 0.  This is different from loading a snapshot file taken at cycle
zero.  Such a snapshot file can be taken, and represents the state of the
processor before starting to execute the first instruction of the program.
This state includes the stack initialization memory area.

\paragraph{Selecting which procedures are traced and analysed.}
The Simulator allows to include or exclude any procedure of the
TTA application from simulation analysis and tracing.%
\note{ALT: tag aggregate data in different ``pools'', one for procedure}
%
The excluded procedures are still simulated, but the Simulator does not
collect statistics while simulating their code.  This capability is useful
to reduce the amount of data produced (which may become unwieldy for
long-running simulations), to speed up simulation and to focus analysis on
critical parts of the application.

In languages that allow multiple compilation units and subroutines
(henceforth: procedures) with unit-local scope (e.g., C and C++
\emph{static} non-member functions), a program can contain multiple
procedures with the same name.  In this case, all procedures with the same
name are affected (included or excluded).  It is possible to have a finer
control of the included excluded procedures by prepending the name of the
source file followed by a colon to the procedure name.  For example:
``myModule.c:myfunc'' will affect only the function ``myfunc'' defined in
file ``myModule.c''.

\section{Usage}
\label{sec:usage}

In addition to the standard command line options accepted by every
application of the TCE toolset [[document to define]], the CLI-based
Simulator accepts the command line options listed below.  All command line
options are prepended by two hyphens: `\verb|--|'.  For each option, the
second column indicates the type of argument (if any) of the option:
\emph{i} (integer number), \emph{k} (keyword), \emph{s} (string), \emph{f}
(fractional number), \emph{b} (no option, that is, boolean).  A Boolean
option disables the flag it controls if prepended by `\verb|no-|'.  Unless
otherwise stated, other option types cannot be prepended by the `\verb|no-|'
prefix.

\begin{center}
\begin{longtable}[htb]{@{}p{.20\textwidth}@{}p{.05\textwidth}%
                     @{}p{.75\textwidth}}

\multicolumn{3}{c}{Program environment control} \\
\verb|arguments|      & \emph{s} & The string is passed unmodified to the
                                   simulated application, which will treat
                                   it as its command line.  This option may
                                   be repeated, the strings are concatenated
                                   in the same order they appear on the
                                   Simulator command line.\\
\verb|SPR-location|   & \emph{i} & Address of the location where the initial
                                   value of SPR is stored. Ignored if the
                                   SPR initial value is hard-coded in the
                                   program.\\
\verb|SPR-init|       & \emph{i} & Initial value of the stack pointer to be
                                   stored in SPR. Ignored if the SPR initial
                                   value is hard-coded in the program.\\
\verb|SPR-true-top|   & \emph{b} & If enabled, the given initial value of
                                   SPR is the real top of the stack, and any
                                   initialisation data is stored below.  If
                                   disabled, the initial value of SPR is the
                                   bottom of the stack, and the real top of
                                   the stack is given by adding the size of
                                   the initialisation data area it.\\
\verb|init|           & \emph{s} & Run a Simulation control language script
                                   before starting simulation.  The name of
                                   the script file is given by \emph{s}.\\
\multicolumn{3}{c}{Trace control} \\
\verb|trace-run|      & \emph{k} & What to store in the control flow trace:
                                   \emph{syscall} calls to system,
                                   \emph{call} procedure entry and exit
                                   points, \emph{block} entry and exit
                                   points, \emph{instruction} the address of
                                   every instruction executed, \emph{stall}
                                   elapsed cycles in which the processor is
                                   locked.  This option may be repeated
                                   several times to accumulate data.\\
\verb|trace-bus|      & \emph{b} & Trace the signals carried on the
                                   transport network.\\
\verb|trace-memory|   & \emph{k} & Trace the memory activity: \emph{access}
                                   load and store address, \emph{data} data
                                   words sent to or loaded from memory,
                                   \emph{width} of the data words.\\
\verb|trace-freq|     & \emph{i} & Sampling rate: number of real-time cycles
                                   between two consecutive entries in the
                                   simulation trace.  By default, every
                                   cycle is traced (sampling rate 1).\\
\verb|trace-start|    & \emph{i} & Real-time start cycle of simulation
                                   tracing.  By default, simulation starts
                                   at cycle 0.\\
\verb|trace-stop|     & \emph{i} & Real-time cycle of simulation at which
                                   tracing stops.  By default, tracing stops
                                   when the simulation ends.\\

\multicolumn{3}{c}{Profile control} \\
\verb|profile|        & \emph{k} & Method of storing profiling data:
                                   \emph{none} disable profiling,
                                   \emph{accumulate} with any existing
                                   profile data set, \emph{append} a new
                                   data section with this profile data set,
                                   \emph{replace} any existing profile data
                                   sections.\\

\multicolumn{3}{c}{Simulation control} \\
\verb|no-debug-mode|     & \emph{b} & Do not start Simulator in debug mode.  By
                                   default, the simulation is started in debug mode.\\
\verb|hazards|        & \emph{k} & Control what happens when a hardware
                                   resource conflict (hazard) is detected:
                                   \emph{none} ignore and continue
                                   simulation, \emph{warning} print a
                                   warning to standard error stream,
                                   \emph{error} abort simulation.\\x
\verb|snapshot-at|    & \emph{i} & Machine state snapshot.  Real-time cycle
                                   at which a snapshot is taken.  This
                                   option may be repeated.\\
\verb|snapshot-every| & \emph{i} & Take a snapshot of the machine state at
                                   intervals of \emph{i} cycles.\\
\verb|start|          & \emph{i} & Start simulation at a given cycle
                                   \emph{i}.  Works only if a machine state
                                   snapshot at cycle \emph{i} has been taken
                                   in a previous simulation run, and the
                                   resulting file is in the current
                                   directory.  By default, simulation is
                                   started at cycle 0 without loading a
                                   snapshot file.\\
\verb|proc|           & \emph{s} & Include (exclude if option is negated)
                                   all procedures by name \emph{s} from
                                   collected data (execution trace,
                                   simulation statistics).  If prepended by
                                   a source file name followed by a colon
                                   `:', then only the procedure in the given
                                   file is affected.  This option may be
                                   repeated to include (or exclude) multiple
                                   procedures.  By default, all procedures
                                   are included.\\
\verb|proc-libs|      & \emph{b} & Include (exclude if option is negated)
                                   all procedures that belong to libraries
                                   linked to the target program.\\
\end{longtable}
\end{center}


\subsection{Command Line Option Default Values}
\label{ssec:usage-defaults}

If no default value is listed, it means that the option has no default value
(for example, an empty set/list of values) or that has an internal value
that cannot be specified via command line option (for example, the default
stop-tracing-at-end-of-simulation).

\begin{center}
\begin{tabular}[htb]{@{}p{.25\textwidth}@{}p{.75\textwidth}}

\verb|SPR-location|   & ignored (assumed hard-coded in program)\\
\verb|SPR-init|       & ignored (assumed hard-coded in program)\\
\verb|SPR-true-top|   & true\\
\verb|trace-bus|      & false\\
\verb|trace-freq|     & 1\\
\verb|trace-start|    & 0\\
\verb|profile|        & none\\
\verb|debug-mode|     & false\\
\verb|proc-libs|      & true\\

\end{tabular}
\end{center}

\section{Simulator Control Language}
\label{sec:debug}

The Simulator control language allows to:
\begin{enumerate}
\item %
  Start program simulation, setup the environment that affects the program
  behaviour.
\item %
  Stop and resume program simulation on specified conditions.
\item %
  Examine the state of the target processor and the simulation process.
\item %
  Modify the state of the target processor so that it is possible to correct
  program errors during simulation.
\end{enumerate}

\begin{quote}
  NOTE: The simulator control language syntax is strongly based on GNU
  Source-Level Debugger (GDB) commands.  In practice, it is almost a
  scaled-down and simplified version of the GDB command set.
\end{quote}

When the Simulator is run in debug mode, it automatically reads and executes
the initialisation command file `.ttasim-init' if found in the user home
directory.  After the initialisation command sequence is completed, the
Simulator processes the command line options, and then reads the
initialisation command file with the same name in current working directory.

After it has processed the initalisation files and the command line options,
the Simulator is ready to accept new commands, and prompts the user for
input.  The prompt line contains the string `(ttasim) ' (notice the space at
the end of the string).

This section describes all the Simulator commands that can be entered when
the Simulator runs in debug mode.  The Simulator displays a new line with
the prompt string only when is ready to accept new commands (the simulation
is not running).  The running simulation can be interrupted at any time by
the key combination CTRL-c.  The simulator stops simulation and prompts the
user for new commands as if it had been stopped by a breakpoint.

The Simulator control language is based on the Toolset Control
Language~\cite{Tcl-orig}.  It extends the predefined set of Tcl commands
with a set of commands that allow to perform the functions listed above.  In
addition to predefined commands, all basic properties of Tcl (expression
evaluation, parameter substitution rules, operators, and so on) are
supported.  For an extensive description of these properties, the reader is
referred to any good book on Tcl, for example~\cite{Tcl-practical}.


\subsection{Simulation and Program Environment Variables}
\label{ssec:debug-set}

The Simulator control language stores important information that affects
simulation in a number of \emph{simulation environment variables}.  It is
possible to set, modify and inspect these variables.

\begin{description}
\item[setting \emph{variable} \emph{value}] %
  Sets a new value of environment variable \emph{variable}.
\item[setting {[\emph{variable}]}] %
  Prints the current value contained by environment variable
  \emph{variable}.
\end{description}

These variables affect the simulator and must not be confused with the
variables that represent the environment of the simulated program.  The
\emph{program environment} is a set of variable definitions that are
accessible to the simulated program and are used to pass information to a
process from its caller.

Normally, the shell environment from which the Simulator application was
invoked is used for the simulated program, but it is possible to modify or
completely replace this environment by means of command \emph{set
environment}, described in Section~\ref{ssec:debug-running}.

The simulation environment variables recognised by Simulator will be
described in the following sections.

\subsection{Control of How the Simulation Runs}
\label{ssec:debug-running}

The commands described in this section allow to control the simulation process.

Before simulation can start, a program must be loaded into the Simulator.
If no program is loaded, the command \emph{run} causes the following
message:
\begin{quote}
  \begin{verbatim}
  Simulation not initialized.
  \end{verbatim}
\end{quote}

Whenever the simulation stops, the Simulator prints the TTA instruction
where the program stopped:
\begin{quote}\tt
  \parm{addr} <\parm{symbol-offset}>: \parm{instruction}
\end{quote}
%
where \emph{addr} is the absolute address of the instruction,
\emph{symbol-offset} is the symbolic address (symbol + offset), which shows
the offset from the nearest procedure entry point, and \emph{instruction} is
the TTA instruction.

\begin{description}
\item[run {[\emph{arguments}]}] %
  Starts simulation of the program currently loaded into the Simulator.  The
  program can be loaded by \emph{file} command (see
  Section~\ref{ssec:debug-files}) or may be given directly as argument, on
  the command line.  Simulation runs until either a breakpoint is
  encountered or the program terminates.

  If specified, the arguments are passed on to the program to be simulated.
  If no argument is given, the arguments (if any) of the most recent
  \emph{run} command executed will be used.  It is also possible to set the
  arguments in the environment variable \emph{args}.  A \emph{run} command
  without arguments will take its arguments from \emph{args}.

  To control the environment of the program, (see
  Section~\ref{ssec:debug-set}), it is possible to use the \emph{set
  environment} command, described below.

\item[setargs {[\emph{arguments}\ldots]}]%
  Sets the arguments that are passed on to the program being simulated.  The
  arguments are treated as pathname patterns and undergo pathname expansion.
  The pattern format and the expansion comply with POSIX.2
  standard.\footnote
  %
  {Unix \emph{glob} library function complies with this standard and is
  usually mimicked by shells.}
  %
  The arguments are then further processed.  Variable substitution takes
  place as defined by the underlying interpreted language.

\item[setenvironment \emph{varname} {[=]} \emph{value}]%
  Sets environment variable \emph{varname} with value \emph{value}, to be
  passed to the TTA program.  If parameter \emph{value} is omitted, the
  variable is set to a null value.  Spaces between `=' and the two
  parameters are removed.

\item[unsetenvironment \emph{varname}]%
  Remove variable \emph{varname} from the environment to be passed to the
  TTA program.

\item[resume {[\emph{count}]}] %
  Resume simulation of the program until the simulation is finished or a
  breakpoint is reached.  The \emph{count} argument gives the
  number of times the \emph{continue} command is repeated, that is, the
  number of times breakpoints should be ignored.

\item[step {[\emph{count}]}] %
  (optional) Continues to the next source line, stepping into a new stack
  frame if a function call is reached.  The \emph{count} argument gives the
  number of lines to skip.

\item[stepi {[\emph{count}]}] %
  Advances simulation to the next machine instructions, stepping into the
  first instruction a new procedure if a function call is simulated.  The
  \emph{count} argument gives the number of machine instruction to simulate.

\item[next {[\emph{count}]}] %
  (optional) Continues to the next source line in the current stack frame.
  The \emph{count} argument gives the number of lines to skip.

\item[nexti {[\emph{count}]}] %
  Advances simulation to the next machine instructions in current procedure.
  If the instruction contains a function call, simulation proceeds until
  control returns from it, to the instruction past the function call.  The
  \emph{count} argument gives the number of machine instruction to simulate.

\item[finish] %
  Continues simulation until the function of the selected stack frame
  returns (see Section~\ref{ssec:debug-stack} for details on stack frames).

\item[until {[\emph{arg}]}] %
  Continue running until the program location specified by \emph{arg} is
  reached.  Any valid argument that applies to command \emph{break} (see
  Section~\ref{ssec:debug-breakpoints}) is also a valid argument for
  \emph{until}.  If the argument is omitted, the implied program location%
  \note{implicit program location = next source line: optional feature}
  %
  is the next source line.  In practice, this command is useful when
  simulation control is inside a loop and the given location is outside it:
  simulation will continue for as many iterations as required in order to
  exit the loop (and reach the designated program location).

\item[kill] %
  Terminate the simulation.  The program being simulated remains loaded and
  the simulation can be restarted from the beginning by means of command
  \emph{run}.  The Simulator will normally prompt the user for confirmation
  before terminating the simulation.

\item[quit] %
  This command is used to terminate simulation and exit the Simulator.
\end{description}

\paragraph{Redirecting program's input and output.}
By default, the program being simulated reads its input and sends its output
on the same terminal used to invoke the Simulator.  This is the same
terminal that the Simulator uses to interact with the user by means of the
simulation control commands. It is also possible, however, to redirect the
input and the output.

It is possible to redirect the output of the program being simulated to a
file by appending the redirection operator `>' followed by a file name at
the end of the \emph{run} command.

Alternatively, it is possible to use command \emph{tty} to redirect both
input and output of the simulated program to another terminal (see
Section~\ref{ssec:debug-files} for details).

\subsection{Examining the Stack}
\label{ssec:debug-stack}

The stack of a program is divided up into contiguous blocks of memory called
\emph{frames}.  A frame is activated upon entering a function and is
destroyed (that is, its storage is made available for new frames) when the
Simulator returns from the function.  Each time the simulator performs a
function call, it generates information to track the state of the stack.

Several commands of the Simulator control language apply to the stack frame
currently selected. Normally, the selected frame is the top frame, which
belongs to the innermost function in the call chain.  Frames are identified
by a unique \emph{frame number}.  This number is zero for the top frame (the
frame of the innermost function call activation), and grows while proceeding
to outer frames.

A stack frame contains arguments of the called function, saved register
values and local variables.  \note{TTA-specific part}
%
The stack pointer register (SPR) always points to the first word past the
local variable area of the stack.  If current function calls other functions
and passes any arguments on the stack, the word pointed at by SPR is the
first word of the outgoing arguments area; otherwise, it is the first
unallocated word past the stack top.  Since the stack grows downwards,
incoming arguments and local variables on the stack frame have always
positive offsets when accessed from the current function, whereas outgoing
arguments have negative offsets.  See [[architecture specs]] for a detailed
description of the frame organisation.

The main properties of a stack frame are printed on two lines with the
following format:
\begin{quote}\tt
  \#\parm{num}: \parm{function} (\parm{arg}=\parm{value}, \ldots)
  at \parm{file}:\parm{lnum}\\
  \parm{lnum}\hspace{8ex}\parm{sourceline}\\
\end{quote}
%
where \emph{num} is the frame number, \emph{function} is the name of the
function the frame belongs to, \emph{file} is the source file where the
function is defined, and \emph{lnum} is the number of the source line in the
frame the simulation control is at.  The list of function arguments (if any)
and their values are also displayed.  (optional) The second line shows the
text of the source line \emph{sourceline}; if the selected frame is not the
innermost, this line should contain a function call.

Detailed frame information is displayed in several lines of text, with the
following format:
\begin{quote}\tt
  Stack level \parm{fnum}, frame at \parm{addr}:\\
  spr = \parm{addrsp} in \parm{function} (\parm{file}:\parm{line});
          saved spr \parm{par-addrsp}\\
  called by frame at \parm{par-addr}\\
  source language \parm{language}.\\
  Arglist at \parm{arg-base}, \parm{name1}=\parm{argaddr1}, \ldots\\
  Locals at \parm{local-base}\\
  Saved registers:\\
  \ \parm{regname1} at \parm{regaddr1}, \ldots\\
\end{quote}
%
where \emph{fnum} is the frame number, \emph{addr} is the start (highest)
address of the frame, \emph{addrsp} and \emph{par-addrsp} are, respectively,
the current and the previous contents of SPR,%
\note{clarify: contents of SPR before or after it is set up by current
function?}
%
\emph{par-addr} is the start address of the parent frame.  The addresses
\emph{arg-base} and \emph{local-base} are, respectively, the start address
of the frame dedicated to incoming function arguments and local variables.
In TTA architecture, \emph{arg-base} is always equal to \emph{addr}.  If no
incoming argument is passed on the stack then also \emph{local-base} is
equal to \emph{addr}.  The argument list base address is followed by a list
of all incoming arguments (name and stack address).  The list of saved
register shows the register name and the stack address where the registers
are saved.

Local variables and incoming arguments of selected frame are displayed on
one line each, with the following formats:
\begin{quote}\tt
  \parm{name} = \parm{addr} \parm{value}\\
\end{quote}
%
where \emph{name} is the name of the argument or the local variable,
\emph{addr}, if applicable, is the address where the data is stored, and
\emph{value} is the value of the variable.

The following commands allow to select and examine the stack frames during
simulation.

\begin{description}
\item[frame \emph{arg}] %
  Select a stack frame.  The argument \emph{arg} may be either the address
  of the frame or the stack frame number.  This command prints the main
  properties of the selected frame.

\item[select-frame \emph{arg}] %
  Select a stack frame.  This is a silent version of the \emph{frame}
  command which does not print the properties of the selected frame (see
  above).

\item[up {[\emph{num}]}] %
  Move \emph{num} frames up the stack.  If \emph{num} is a positive number,
  the Simulator advances toward the outermost frame, to higher frame
  numbers.  If omitted, \emph{num} defaults to one.  This command prints the
  main properties of the selected frame.

\item[up-silently {[\emph{num}]}] %
  Variant of command \emph{up} which does not display the new frame.

\item[down {[\emph{num}]}] %
  Move \emph{num} frames down the stack.  If \emph{num} is a positive
  number, the Simulator advances toward the innermost frame, to lower frame
  numbers.  If omitted, \emph{num} defaults to one.  This command prints the
  main properties of the selected frame.

\item[down-silently {[\emph{num}]}] %
  Variant of command \emph{down} which does not display the new frame.

\item[backtrace {[[-]\emph{num}]}] %
  Prints a backtrace of \emph{num} innermost (outermost if prepended by
  minus sign) frames.  If the argument is omitted, prints all stack frames.
  Each frame is printed in a separate line.  Only the first line of main
  properties (shown above) is printed.

\item[info args] %
  Prints the arguments of the selected frame, each on a separate line.

\item[info frame {[\emph{addr}]}] %
  Prints detailed description of the frame at address \emph{addr}.  The
  frame currently selected remains unchanged by this command.  A valid
  \emph{addr} argument for command \emph{frame} is also a valid argument of
  command \emph{info frame}.  If \emph{addr} is omitted, the command applies
  to the selected frame. 

\item[info locals] %
  Prints the local variables of the selected frame, each on a separate line.
  These are all variables (declared either static or automatic) accessible
  at the point of execution of the selected frame.
\end{description}

\subsection{Examining Program Code and Data}
\label{ssec:debug-data}

The Simulator allows to examine the program being simulated and the data it
uses.  Usually, a variable or expression is displayed on one line, with the
following format:
\begin{quote}\tt
  \verb|$|\parm{num} = \parm{addr} \parm{value}\\%$
\end{quote}
%
where \emph{num} is the identification number assigned to the expression
specified with the printing command, \emph{addr} is its absolute address (in
hexadecimal base) and \emph{value} is the value of the variable, displayed
with the appropriate data type.  If the expression displayed is a variable
that lives in a register, then the address \emph{addr} is replaced by the
register name, \emph{regname}:
\begin{quote}\tt
  \verb|$|\parm{num} = \parm{regname} \parm{value}\\%$
\end{quote}

The address \emph{addr} display can be disabled with command \emph{set print
address}; the register name \emph{regname} can be disabled with command
\emph{set print register}.  See Section~\ref{ssec:debug-misc} for details.

Data structures are displayed in a format that resembles a C \emph{struct}
declaration, as the following example shows:
\begin{verbatim}
$1 = {next = 0x0, foo = {bar = 5, bag = 77}, \
baz = 0x54 "Umma Gamma"}
\end{verbatim}%$

It is often useful to print the value of an expression frequently, even
every time simulation stops. The Simulator maintains a list of expression
whose value is printed automatically.  A display has the following format:
\begin{quote}\tt
  \parm{num}: \parm{expr} = \parm{value}
\end{quote}
%
where \emph{num} is a unique number that identifies this display,
\emph{expr} is the expression as entered with the \emph{display} command
(see below) and being displayed, and \emph{value} is the value of the
expression.

The program status is displayed with one of the following lines:
\begin{quote}\tt
  Program not loaded.\\
  Loaded program not being simulated.\\
  Program stopped at \parm{address}.\\
\end{quote}
If the program is being simulated, then an additional line summarises how
the program simulation got to the given address. One of the following
messages is possible:
\begin{quote}\tt
  Stopped at breakpoint \parm{num}.\\
  Stopped after being stepped.\\
  Stopped at a breakpoint that has since been deleted.\\
\end{quote}
The last message occurs when the Simulator has stopped at a temporary
breakpoint created by the user (see command \emph{tbp},
Section~\ref{ssec:debug-breakpoints}) or implicitly created by the Simulator
as side-effect of certain commands (such as \emph{finish}, also described in
Section~\ref{ssec:debug-breakpoints}).

Source line  information is displayed in the following format:
\begin{quote}\tt
  Line \parm{lnum} of "\parm{file}" starts at pc \parm{saddr}
      and ends at \parm{eaddr}.
\end{quote}
  %
  where \emph{saddr} and \emph{eaddr} are absolute addresses displayed in
  hexadecimal base.

\begin{description}
\item[print {[/\emph{fmt}] [\emph{expr}]}] %
  Prints the current value of expression \emph{expr}.  See below for a
  description of the output formats that can be specified by parameter
  \emph{fmt}.  Whenever a \emph{print} command is executed, the value
  displayed is assigned a unique identification number and stored in the
  value history log, described in Section~\ref{ssec:debug-history}.

  If \emph{expr} is omitted, the last value, taken from the value history,
  is printed again.  This feature is useful when the same value must be
  inspected in an alternative output format.

\item[printf \emph{string}, \emph{expr}] %
  Prints the values of the expression \emph{expr} with the format specified
  by string \emph{string}.  The expression may consist of a sequence of
  several expressions separated by commas.  The expression values are
  printed exactly as if the command were replaced by a call to C function
  \emph{printf} with the same arguments.

\item[echo \emph{text}] %
  Prints \emph{text}.  Nonprinting characters can be entered using C escape
  sequences; for example, `$\backslash$n' prints a newline.  In addition to
  the standard C escape sequences, a backslash followed by a space stands
  for a space. This is useful for displaying a string with spaces at the
  beginning or the end, since leading and trailing spaces are otherwise
  trimmed from all arguments.

  The values printed by command \emph{echo} are not entered in the value
  history (see Section~\ref{ssec:debug-history}).


\item[output {[/\emph{fmt}]} \emph{expr}] %
  Prints the value of the expression \emph{exp}, without the preceding
  `\$\emph{num} ='.  The value is not entered in the value history.  

  See below for a description of the output formats that can be specified by
  parameter \emph{fmt}

\item[x {[/\emph{nfu}][\emph{addr}]}] %
  This low-level command prints the data in memory starting at specified
  addresses \emph{addr}.  The optional parameters \emph{n}, \emph{f} and
  \emph{u} specify how much memory to display and how to format it.
  \begin{description}
  \item[\emph{n}] %
    Repeat count: how many data words (counting by units \emph{u}) to
    display.  If omitted, it defaults to 1.
  \item[\emph{f}] %
    Display format is one of the output (specified below), `s'
    (null-terminated string) or `i' (machine instruction).  If omitted, it
    defaults to `x' (hexadecimal).
  \item[\emph{u}] %
    Unit size: `b' (MAU, a byte in byte-addressed memories), `h' (double
    MAU), `w' (quadruple word, a `word' in byte-addressed 32-bit
    architectures), `g' (giant words, 8 MAU's).  The unit size is ignored
    for formats `s' and `i'.
  \end{description}

  If \emph{addr} is omitted, then the first address past the last address
  displayed by the previous \emph{x} command is implied.  If the value of
  \emph{n}, \emph{f}, or \emph{u} is not specified, the value given in the
  most recent \emph{x} command is maintained.

  The values printed by command \emph{x} are not entered in the value
  history (see Section~\ref{ssec:debug-history}).

\item[display {[/\emph{fmt}]} {[\emph{addr}|\emph{expr}]}] %
  Adds a memory address \emph{addr} or an expression \emph{expr} to the list
  of expressions that are automatically displayed every time the simulation
  stops.  If given an expression \emph{expr}, the \emph{display} command
  accepts an optional output format parameter, described below.  If given an
  address, the \emph{display} command accepts the same optional format
  parameter accepted by command \emph{x}.

  If only the format argument is given, then the most recent address or
  expression is added again to the display list (with the new print format).
  If only the address or expression is given, the format given in the most
  recent \emph{display} command is assumed.  If no argument at all is given,
  the Simulator prints the whole automatic display list (as if the program
  just stopped).

\item[undisplay {[num]}] %
\setlength{\itemsep}{0cm}
\setlength{\parskip}{0cm}
\item[delete display {[num]}] %
  Removes the display with number \emph{num} from the automatic display
  list.
\setlength{\itemsep}{\stditemsep}
\setlength{\parskip}{\stdparskip}

\item[enable {[num]}] %
  Enables the display with number \emph{num} (if previously disabled).  The
  Simulator automatically re-enables a display expression when its scope in
  the program is reentered.

\item[disable {[num]}] %
  Disables the display with number \emph{num}.  When disabled, a
  display remains in the automatic display list, but is not printed.  The
  Simulator automatically disables a display expression when the scope (the
  stack frame) where it is defined is not the currently active frame.

\item[info display] %
  Lists the expressions previously set up to display automatically, but
  without showing their values.  Disabled expression are also shown.

\item[disassemble {[\emph{addr1} [\emph{addr2}]]}] %
  Prints a range of memory addresses as machine instructions.  When two
  arguments \emph{addr1}, \emph{addr2} are given, \emph{addr1} specifies the
  first address of the range to display, and \emph{addr2} specifies the last
  address (not displayed).  If only one argument, \emph{addr1}, is given,
  then the function that contains \emph{addr1} is disassembled.  If no
  argument is given, the default memory range is the function surrounding
  the program counter of the selected frame.

\item[list {[\emph{line}|\emph{function}]}] %
  (optional) Prints a number of lines centred around source line
  \emph{line} or the beginning of function \emph{function}.

  By default, 10 lines are printed.  This can be changed by the command
  \emph{set listsize}.

\item[info line {[\emph{linespec}]}] %
  Prints the first and the last addresses of the source line
  \emph{linespec}.  The argument \emph{linespec} can be one of the
  line specifications described below.
\end{description}

\paragraph{Expressions.}
Many commands accept expressions as arguments.  Any valid C expression%
\footnote{The implementation language is Tcl, whose expression syntax is
compatible with C syntax.},%
with exception of type casts and function calls, is accepted as valid
expression.  Variable names identify variables in the program source code
and are valid expression terms.  A variable name can be qualified by the
name of the source file where it is defined:
\begin{quote}\tt
  \parm{file}::\parm{variable}
\end{quote}
This means that it is possible to access a variable whose scope is
a single source file even if the current execution point is not in that
file.  Also local static variables can be accessed anywhere, in this case,
the variable name is qualified with the function name:
\begin{quote}\tt
  \parm{function}::\parm{variable}
\end{quote}

\paragraph{Addresses.}
Memory addresses are a special type of expression term.  Since the target
architecture supports multiple, independent addressing spaces, the
sequential integer number of a memory location is not sufficient, in
general, to identify the location.  To be complete, an address must also
specify the address space it refers to.  The address space can be referred
to in several ways:
\begin{enumerate}
\item %
  By specifying a unique address space identification number.
\item %
  By specifying the name of the address space.
\end{enumerate}
All address spaces that form the memory system of the target architecture
are assigned a unique integer number.  The number zero is always assigned to
the memory that contains the instructions of the program (instruction
memory).  The address space is a suffix of the address and is separated from
the address location number by a colon `:':
\begin{quote}\tt
  \parm{address-space}:\parm{location}
\end{quote}
To facilitate the use of commands with address arguments, if the address
space is omitted, the last address space that was specified is implied.  The
address space can be omitted when it is obvious from the context (that is,
from the command in which it is used).  For example, the arguments specified
with command \emph{disassemble} belong to the instruction memory.

\paragraph{Output formats.}
Certain commands take an optional \emph{format} argument that specifies the
format of the values being printed by the command.  The format argument
always starts with a slash `/' and is followed by a format letter.  Unless
otherwise noted in the command description, the format letter supported are:
\begin{description}
\item[x] %
  Integer in base 16 (hexadecimal).
\item[d] %
  Signed integer in decimal.
\item[u] %
  Unsigned integer in decimal.
\item[o] %
  Integer in base 8 (octal).
\item[t] % 
  Integer in base 2 (binary).  The letter `t' stands for "two".
\item[a] %
  Address, both absolute (in hexadecimal) and as an offset from the nearest
  preceding symbol.  The latter is useful to discover in what function an
  unknown address is located.  The offset expression has the following
  format: \texttt{<\parm{function}+\parm{offset}>}. 
\item[c] %
  Integer printed as a character constant in the ASCII character set.
\item[f] %
  Floating point number printed using typical floating point syntax. 
\end{description}

\paragraph{Line specifications.}
Commands that operate on source code lines take one or more line
specification arguments.  A line specification argument may have several
formats:
\begin{description}
\item[\emph{number}]%
  Specifies the line number in the current source file.
\item[*\emph{address}]%
  Specifies the source line that contains address \emph{address}.
\item[+\emph{offset}]%
  Specifies the line \emph{offset} lines past the last line printed.
\item[-\emph{offset}]%
  Specifies the line \emph{offset} lines before the last line printed.
\item[\emph{file}:\emph{number}]%
  Specifies the line number and the source file name.
\item[\emph{function}]%
  Specifies the first source line of the function body of function
  \emph{function} in current source file.
\item[\emph{file}:\emph{function}]%
  Specifies the first source line of the function body of function
  \emph{function} in source file \emph{file}.
\end{description}

\subsection{Control Where and When to Stop Program Simulation}
\label{ssec:debug-breakpoints}

A breakpoint stops the simulation whenever the Simulator reaches a certain
point in the program. It is possible to add a condition to a breakpoint, to
control when the Simulator must stop with increased precision. There are two
kinds of breakpoints: \emph{breakpoints} (proper) and \emph{watchpoints}. A
watchpoint is a special breakpoint that stops simulation as soon as the
value of an expression changes.

Properties of breakpoints are displayed with a line in the following format:
\begin{quote}
 \parm{num}: \parm{description}
\end{quote}
%
where \emph{num} is a unique number that identifies this breakpoint or
watchpoint and \emph{description} describes the properties of the
breakpoint. The properties include: whether the breakpoint must be deleted
or disabled after it is reached; whether the breakpoint is currently
disabled; the program address of the breakpoint, in case of a program
breakpoint; the expression that, when modified by the program, causes the
Simulator to stop, in case of a watchpoint.

\begin{description}
\item[bp {[\emph{file}:]\emph{function}}] %
  Sets a breakpoint at the entry of any function whose name is
  \emph{function}.  The optional \emph{file} prefix allows to restrict the
  selection to the function of a given source file.

\item[bp +\emph{offset}] %
  (optional) Sets a breakpoint \emph{offset} lines forward from the position
  at which execution is stopped in selected frame (see
  Section~\ref{ssec:debug-stack} for details on stack frames).

\item[bp -\emph{offset}] %
  (optional) Sets a breakpoint \emph{offset} lines back from the position at
  which execution is stopped in selected frame (see
  Section~\ref{ssec:debug-stack} for details on stack frames).

\item[bp {[\emph{file}:]\emph{line}}] %
  (optional) Sets a breakpoint at line \emph{line} in the current source
  file.  Simulation will stop the program just before any instruction that
  belongs to given line is executed.
\item[bp \emph{address}] %
  Sets a breakpoint at address \emph{address}. Argument can also be a code
  label such as global procedure name (e.g. 'main').
\item[bp] %
  Sets a breakpoint at the next instruction to be executed in selected stack
  frame.  If selected frame is not the top frame, control returns from all
  called functions and all inner frames are destroyed (see
  Section~\ref{ssec:debug-stack} for details on stack frames).
\item[bp \emph{args} if] %
  Sets a conditional breakpoint.  The arguments \emph{args} are the same as
  for unconditional breakpoints.  After entering this command, Simulator
  prompts for the condition expression. Condition is evaluated each time 
  the breakpoint is reached, and the simulation only when the condition 
  evaluates as true.
\item[tbp \emph{args}] %
  Sets a temporary breakpoint, which is automatically deleted after the
  first time it stops the simulation.  The arguments \emph{args} are the
  same as for the \emph{bp} command.  Conditional temporary breakpoints
  are also possible (see command \emph{condition} below).
\item[watch \emph{expr}] %
  Sets a watchpoint for the expression \emph{expr}.  The Simulator will stop
  when the value of given expression is modified by the program.
  Conditional watchpoints are also possible (see command \emph{condition}
  below).

\item[condition {[\emph{num}] [\emph{expr}]}] %
  Specifies a condition under which breakpoint \emph{num} stops simulation.
  The Simulator evaluates the expression \emph{expr} whenever the breakpoint
  is reached, and stops simulation only if the expression evaluates as true
  (nonzero).  The Simulator checks \emph{expr} for syntactic correctness as
  the expression is entered.

  When \emph{condition} is given without expression argument, it removes any
  condition attached to the breakpoint, which becomes an ordinary
  unconditional breakpoint.
\item[ignore {[\emph{num}] [\emph{count}]}] %
  Sets the number of times the breakpoint \emph{num} must be ignored when
  reached.  A \emph{count} value zero means that the breakpoint will stop
  simulation next time it is reached.

\item[command {[\emph{num}] [\emph{command-list}]}] %
  Specifies a list of commands for breakpoint \emph{num}.  The list of
  commands starts with next line, spans multiple lines and is terminated by
  a line containing only the command \emph{end}.  The commands in the list
  will be executed each time the breakpoint is reached and, if conditional,
  its condition evaluates as true.

  If the first command in \emph{command-list} is \emph{silent}, then the
  usual message about stopping at a breakpoint is not printed. If the
  command list contains only \emph{end}, then any preexisting command list
  is removed.  If no breakpoint number \emph{num} is specified, then the
  command is applied to the last breakpoint that was set.

  It is possible to include a command such as \emph{continue} or \emph{next}
  (see Section~\ref{ssec:debug-running}) inside \emph{command-list} to
  resume simulation.  In that case, all remaining commands in the list are
  ignored; simulation resumes as if the simulation had not stopped at the
  breakpoint.  This allows to patch corrupted processor state caused by
  program errors and continue simulation.

\item[enablebp {[delete|once] [\emph{num} \ldots]}] %
  Enables the breakpoint specified by \emph{num}.  If \emph{once} flag is
  specified, the breakpoint will be automatically disabled after it is
  reached once.  If \emph{delete} flag is specified, the breakpoint will be
  automatically deleted after it is reached once.
\item[disablebp {[\emph{num} \ldots]}] %
  Disables the breakpoint specified by \emph{num}.  A disabled breakpoint
  has no effect, but all its options (ignore-counts, conditions and
  commands) are remembered in case the breakpoint is enabled again.
\item[deletebp {[\emph{num} \ldots]}] %
  Deletes the breakpoint specified by \emph{num}.  If no arguments are
  given, deletes all breakpoints currently set, asking first for
  confirmation.
\item[clear {[[\emph{filename}:]\emph{function}|\emph{line}]}]%
  Deletes any breakpoints set at the entry of function \emph{function} or at
  line \emph{line}.  If the filename prefix is specified, deletes only
  breakpoints that belong to \emph{filename} source file.  If no arguments
  are given, deletes any breakpoints at the next instruction in the selected
  stack frame.  This is a convenient way to delete the breakpoint at which
  simulation just stopped.
\item[info breakpoints {[\emph{num}]}] %
\setlength{\itemsep}{0cm}
\setlength{\parskip}{0cm}
\item[info watchpoints {[\emph{num}]}] %
  Prints a table of all breakpoints and watchpoints.  Each breakpoint is
  printed in a separate line.  The two commands are synonymous.
\setlength{\itemsep}{\stditemsep}
\setlength{\parskip}{\stdparskip}
\end{description}

\subsection{Specifying Files and Directories}
\label{ssec:debug-files}

The Simulator needs to know the file name of the program to
simulate/debug and, usually, the Architecture Definition File (ADF)
that describes the architecture of the target processor on which the
program is going to run.  The name of the ADF is not necessary when:
\begin{enumerate}
\item %
  The name of the file is specified in the program file, or is embedded in
  the program file.
\item %
  The program is completely unscheduled, and therefore the target processor
  information is unnecessary (see~\cite{TPEF-specs} for details).
\end{enumerate}

\begin{description}
\item[prog {[\emph{filename}]}]%
  Load the program to be simulated from file \emph{filename}.  If no
  directory is specified with \emph{set directory}, the Simulator will
  search in the current directory.

  If no argument is specified, the Simulator discards any information it has
  on the program.

\item[mach {[\emph{filename}]}]%
  Load the machine to be simulated from file \emph{filename}.  If no
  directory is specified with \emph{set directory}, the Simulator will
  search in the current directory.

  In case a parallel program is tried to be simulated without machine,
  an error message is printed and simulation is terminated immediately.
  In some cases the machine file can be stored in the TPEF file.

\item[conf {[\emph{filename}]}]%
  Load the processor configuration to be simulated from file \emph{filename}.
  If no directory is specified with \emph{set directory}, the Simulator will
  search in the current directory.

  Simulator expects to find the simulated machine from the processor 
  configuration file. Other settings are ignored. This can be used as
  replacement for the \emph{mach} command.

\item[tty {[\emph{terminal}|\emph{filename}]}]%
  Redirects input and output of the program being simulated to the given
  terminal or file.  The input entered by the user is not affected by this
  command.

\item[cd \emph{directory}]%
  Sets the working directory to \emph{directory}.  By default, the working
  directory is the directory from which the Simulator was invoked.

\item[pwd]%
  Prints the current working directory.
\end{description}

\subsection{Examining State of Target Processor and Simulation}
\label{ssec:debug-state}

The current contents of any programmer visible state, which includes any
programmable register, bus, or the last data word read from or written to a
port, can be displayed. The value is displayed in base 10 to allow using it
easily in Tcl expressions or conditions. This makes it possible, for
example, to set a conditional breakpoint which stops simulation only if the
value of some register is greater than some constant.

The segments of a bus are displayed in one line, with the following format:
\begin{quote}\tt
  \parm{busname}: \parm{segment1} \ldots{} \parm{segmentN}
\end{quote}
where \emph{busname} is the name of the bus, and \emph{segment1} to
\emph{segmentN} are the names of the \emph{N} segments that form the bus.

The current contents of a transport bus segment are displayed in one line
with the following format:
\begin{quote}\tt
  \parm{busname} \parm{segmentname} \parm{hexvalue} \parm{value}
\end{quote}
where \emph{busname} is the name of the transport bus, \emph{segmentname} is
the name of one of the segments of the bus, \emph{hexvalue} is the current
value transported by the bus segment expressed in base 16, and \emph{value}
is the same value expressed in base 10.  The string `(undef)' is printed if
the bus segment is not used in current simulation cycle.

\begin{description}
\item[info proc cycles] %
  Displays the total execution cycle count and the total stall cycles count.

\item[info proc mapping] %
  Displays the address spaces and the address ranges occupied by the
  program: address space, start and end address occupied, size.

\item[info proc stats] %
  In case of parallel simulation, displays current processor utilization
  statistics. In case 'rf\_tracking' setting is enabled and running parallel
  simulation, also lists the detailed register file access information. In
  sequential simulation, displays only operation utilization statistics.

\item[info regfiles] %
  Prints the name of all the register files of the target processor.

\item[info registers \emph{regfile} {[\emph{regname}]}] %
  Prints the value of register \emph{regname} in register file
  \emph{regfile}, where \emph{regfile}
  is the name of a register file of the target processor, and \emph{regname}
  is the name of a register that belongs to the specified register file.

  If \emph{regname} is omitted, the value of all registers of the specified
  register file is displayed.

\item[info funits] %
  Prints the name of all function units of the target processor.

\item[info iunits] %
  Prints the name of all immediate units of the target processor.

\item[info immediates \emph{iunit} {[\emph{regname}]}] %
  Prints the value of immediate register \emph{regname} in immediate unit
  \emph{iunit}, where \emph{iunit} is the name of an immediate unit of 
  the target processor, and \emph{regname} is the name of a register 
  that belongs to the specified unit.

  If \emph{regname} is omitted, the value of all registers of the specified
  immediate unit is displayed.

\item[info ports \emph{unit} {[\emph{portname}]}] %
  Prints the last data word read from or written to port \emph{portname} of
  unit \emph{unit}, where \emph{unit} may be any function unit, register
  file or immediate unit of the target processor.  The value of the data
  word is relative to the selected stack frame.

  If \emph{portname} is omitted, the last value on every port of the
  specified unit is displayed.

\item[info busses {[\emph{busname}]}] %
  Displays the name of all bus segments of transport bus \emph{busname}.  If
  the argument is omitted, displays the name of the segments of all busses
  of the target processor.%
  \note{DISCUSS: display also values?}

\item[info segments \emph{bus} {[\emph{segmentname}]}]] %
  Prints the value currently transported by bus segment \emph{segmentname}
  of the transport bus \emph{busname}.

  If no segment name is given, the Simulator displays the contents of all
  segments of transport bus \emph{bus}.

\item[info program] %
  Displays information about the status of the program: whether it is
  loaded or running, why it stopped.
\end{description}

\subsection{Miscellaneous Support Commands and Features}
\label{ssec:debug-misc}

The following commands are facilities for finer control on the behaviour of
the simulation control language.

\begin{description}
\item[help \emph{command}] %
  Prints a help message briefly describing command \emph{command}.

\item[set print address \emph{on}|\emph{off}] %
  Whenever the Simulator displays the contents of a memory address, such as
  the location of stack traces, structure values, pointer values,
  breakpoints, it can automatically print also the address examined.  By
  default, the addresses are printed.  The \emph{print address} variable
  allows to enable (\emph{on}) or disable (\emph{off}) automatic address
  display.
 
\item[show print address] %
  Shows the current setting of \emph{print address} variable.

\item[set print register \emph{on}|\emph{off}] %
  Whenever the Simulator displays the contents of a variable that is
  allocated in a GPR, it can automatically print also the name of the
  register.  By default, the register names are not printed.  The
  \emph{print register} variable allows to enable (\emph{on}) or disable
  (\emph{off}) automatic register name display.
 

\item[show print address] %
  Shows the current setting of \emph{print register} variable.

\item[set print elements \emph{number-of-elements}] %
  Specifies the maximum number of elements that the Simulator prints when an
  array is displayed.

\item[show print elements] %
  Maximum number of elements of an array that are displayed.

\item[set profile \emph{handling-mode}] %
  Specifies how the profile information is handled.  Valid values for argument
  \emph{handling-mode} are keywords from a restricted set: `none',
  `accumulate', `append', `replace'.  See the corresponding command line
  option `profile' for details.

\item[set profile] %
  Shows the current setting of profile handling mode.
\end{description}

\paragraph{Input mode for command sequences.}
Two commands, \emph{command} (see Section~\ref{ssec:debug-breakpoints}) and
\emph{define} (see Section~\ref{ssec:debug-macros}) accept a sequence of
commands spanning multiple input lines as argument.  To enter these special
arguments, the normal command input mode is suspended and the \emph{command
sequence input mode} is enabled.  In this input mode, all the lines of text
entered are stored without being executed.  This input mode is terminated by
the command \emph{end}, which is only useful in command sequence input mode,
and is ignored in normal command input mode.

The command sequence input mode is signalled by the prompt `>' (note no
space after the prompt).


\subsection{Command and Value History Logs}
\label{ssec:debug-history}

All commands given during a simulation/debugging session are saved in a
\emph{command history log}.  This forms a complete log of the session, and
can be stored or reloaded at any moment.  By loading and running a complete
session log, it is possible to resume the same state in which the session
was saved.  Since running a session log implies running again the simulation
as specified by the saved command, the time taken to resume the session
state may be very long.  (Compare this with resuming the machine state,
which is significantly faster, see Section~\ref{sec:control}).

It is possible to run a sequence of commands stored in a command file at any
time during simulation in debug mode using the \emph{source} command.  The
lines in a command file are executed sequentially and are not printed as
they are executed.  An error in any command terminates execution of the
command file.  Commands that usually ask for confirmation or print messages
that describe what they are doing proceed silently and without confirmation.

The values displayed by commands that print expressions or memory addresses,
unless otherwise stated, are saved in a separate \emph{value history} log.
Each value in this log is assigned a unique number \emph{num} and can be
referred to in subsequent commands or expressions as follows:
`\$\emph{num}'.


\begin{description}
\item[setting history\_filename \emph{fname}]%
  Sets the name of the Simulator command history file to \emph{fname}.  When
  command history logging is enabled, the Simulator saves the command
  history of current session in this file.  If the file already exists, the
  new commands are appended at the end of the file.  The name of the history
  file defaults to `.ttasim-history'.

\item[setting history\_save \emph{on}|\emph{off}]%
  Starts or stops recording the command history in the history file.  By
  default, recording is disabled.

\item[setting history\_size \emph{size}]%
  Sets the maximum number of commands that are stored in the history file.
  By default, up to 50 commands are saved.

\item[commands [\emph{num}]]%
  Displays the last \emph{num} commands in the command history log. If
  the argument is omitted, the \emph{num} value defaults to 10.

\item[source \emph{filename}]%
  Executes the command file \emph{filename}.

\item[show values {[+]|[\emph{num}]}]%
  Without any arguments, prints the last 10 values in value history.  If
  \emph{num} is given, prints the 10 values in the history log that are
  centred at value \emph{num}.  If `+' is given, prints 10 history values
  after the last vale that was printed.  An entry of the value history is
  show in a separate line and has the following format: `\$\emph{num} =
  \emph{value}'. If not enough values (or no values at all) exist, the
  \emph{show value} command will display a list with fewer history values
  (or nothing at all).
\end{description}

\chapter{DATABASES}

%% Remove this whole chapter (not needed) for functional specifications of
%% applications that do not define relevant data structures and that use
%% only data from library modules.

The most important bases of data used by the Simulator are managed
externally, in the TCE domain library.  These bases of data are: the Machine
object model (ADF), specified in~\cite{ADF-specs}, the Program object model
(TPEF), specified in~\cite{TPEF-specs}.  In addition, the Simulator depends
on a model of the target processor state and the contents of the memories
accessed by the target processor.  Nevertheless, the Simulator produces and
uses its own databases, which are described in this chapter.

The data bases described in the following sections are stored in a binary
format SQL databases, or, in cases where efficiency while writing is
required, in pure ASCII format.

% PEKKA:
% This chapter describes the data that is processed and used by the module
% or application in broad but still complete way.  More detailed
% descriptions like database tables or XML field names should be in the
% design document.

\section{Contents of Information}

% (PJ:data)

% PEKKA:
%
% This section describes the data types defined and used by the module and
% their relations.  The description is at a high-level, for example using
% UML class diagrams that reflect the real world data, not the classes of
% the target program.
The Simulator generates and manages a number of data bases for communicating
the results of simulation to users and client applications:
\begin{enumerate}
\item %
  Simulation traces.
\item %
  Execution profile data.
\item %
  Machine state snapshot.
\end{enumerate}

The data described in this chapter does not imply the structure or the
logical format in which data is stored.  Some structure is hinted at by
grouping related data fields together, but this is merely done to enhance
clarity of specification.  A possible actual implementation as relational
database is out of the scope of this document.

The data bases described in this chapter consist of data fields of different
types.  A field is identified by a name followed by a data type indicator,
which may be one of the following: integer \ftyp{i}, string of characters
\ftyp{s}, boolean flag \ftyp{b}, fractional number \ftyp{f}.  Optional
fields are surrounded by brackets `[ ]'.

\paragraph{Data format version management}
Every data base produced or read by the Simulator must contain two items of
information pertaining data format version:
\begin{itemize}
\item Current version number.
\item Oldest version number guaranteed to be compatible.
\end{itemize}
The oldest compatible version number is an aid for backward compatibility of
any application that processes the data bases produced by Simulator.  If
such an application supports a data base version equal or greater than
current version number, it is guaranteed that it will work with the current
version.  Both version numbers are stored as a couple of integer numbers:
version-major, version-minor. Second-level minor version numbers are not
supported; it is implied that a minor-minor version numbers must not impact
compatibility.


\subsection{Execution Trace}
\label{ssec:exec-trace}

A trace contains a subset of the following items of information for a given
cycle of the simulation:
\begin{enumerate}
\item Nesting level in function call.
\item Address of the instruction.
\item Enter and exit of program control data structures.
\item Data and control bits carried on each transport bus segment.
\item Memory access activity: address, direction (load or store), size of
  the data word.
\end{enumerate}

The execution trace contains also version information.  A trace processing
application can read any trace for which the oldest compatible version
number is equal or smaller than its own.  The version number of the first
complete data base format is 1.0.  Version numbers smaller than that, used
during development, will not be considered for backward compatibility.

Each traced cycle is tagged with the execution cycle count (the ``virtual
time'' seen by the target architecture) and the real cycle count (which
takes into account also the stall cycles in which the target processor is
locked).  Each traced cycle is described by a variable number of fields
(called entries).  Some entries are optional and can be enabled or disabled.

The trace data base consists of the following fields:
\begin{description}
\item[execution cycle count] \ftyp{i} %
  Current cycle count, counting only architecturally-visible cycles, in
  which the target processor executes and instruction.
\item[real cycle count] \ftyp{i} %
  Current cycle count, counting also cycles in which the target processor is
  globally locked.
\item[nesting level] \ftyp{i} %
  Function call nesting level.  Whenever a function is entered, this number
  is increased by one; after a function returns, this number is decreased by
  one.  The initial nesting level is zero.
\item[instruction address] [\ftyp{i}] %
  Note: the address space index is not necessary, since only one address
  space can contain the program code.
\item[system-call] [\ftyp{s}] %
  Name of the invoked system call.
\item[procedure entry point] %
  The following entries are optional, and are stored only for cycles in
  which a new procedure is entered.  Notice that this may mean zero to
  \emph{L} cycles after the procedure call, depending on how many call delay
  slots have been successfully filled with instructions from the called
  procedure.
  \begin{description}
  \item[name] \ftyp{s} %
    Procedure name.
  \item[address] \ftyp{i} %
    Procedure entry address (effective, not nominal).
  \end{description}
\item[procedure exit point] %
  The following entries are optional, and are stored only for cycles in
  which a procedure returns.
  \begin{description}
  \item[name] \ftyp{s} %
    Procedure name.
  \end{description}
\item[basic block entry point] %
  The following entries are optional, and are only stored for cycles in
  which the first instruction of a block is executed.
  \begin{description}
  \item[index] \ftyp{i} %
    Basic block index. The index is a sequential number that uniquely
    identifies the block \emph{within its procedure}.%
    \note{assign indices sequentially in memory layout order?}
  \item[address] \ftyp{i} %
    Basic block entry address.
  \end{description}

\item[bus activity] %
  The following entries are optional, and are stored for each bus that is
  programmed in current cycle:
  \begin{description}
    \item[index] \ftyp{i} %
      Bus index. The index is a sequential number that uniquely identifies
      the bus.  Indices are assigned starting from 0 and follow the
      declaration order in the ADF.
    \item[data] \ftyp{s} %
      The data written on the bus.  The interpretation of the data (number
      of bits, whether it is integer of floating-point) depends on the bus
      characteristics and the program context.
    \item[squash] [\ftyp{b}] %
      True if current signal on the bus is not carried to destination (that
      is, if the transport has been \emph{squashed} by a guard expression
      that evaluated to \emph{false}).
  \end{description}

\item[memory activity] %
  The following entries are optional.  One entry of this type is stored for
  each memory access occurring in current cycle.  Instruction memory reads
  due to instruction fetching is not traced.
  \begin{description}
    \item[index]  \ftyp{i} %
      Memory index. The index is a sequential number that uniquely
      identifies the address space (memory).  Indices are assigned starting
      from 0 and follow the declaration order in the ADF.
    \item[address] \ftyp{i} %
      Memory address being accessed.
    \item[access type] \ftyp{b} %
      Either load or store.
    \item[size] [\ftyp{i}] %
      Size (in MAU's) of the data word being read or written.
    \item[data] [\ftyp{s}] %
      The data read from or written to the memory.  The interpretation of
      the data (number of bits, whether it is integer of floating-point)
      depends on the program context.
  \end{description}
\end{description}

\subsection{Bus Trace}
\label{ssec:bus-trace}

It is possible to make simulator save a pure ASCII format bus trace file.
Bus trace lists contents of each bus in each simulated clock cycle. This
file is useful in verification and debugging. Bus trace can be enabled in
simulator control language with `setting bus\_trace 1'. Bus trace is stored
in a file called \textit{[programname].bustrace}.

\subsection{Profile Data}
\label{ssec:sim-profile}

Profile data is stored in binary format in a TPEF section.
See~\cite{TPEF-specs} for the specifications of TPEF and the profile
section.

\begin{figure}[tb]
  \centerline{\psfig{figure=eps/CallsAndProfile.eps,width=6cm}}
  \caption{Example of calls to the same procedure with different entry point
    offsets, assuming 2 (or more) call delay slots.}%
  \label{fig:call-profile}
\end{figure}

Profile data consists of execution counts.  Each count takes a fixed-sized%
\note{DISCUSS: fixed-sized double fp counts or integer words of nonfixed
  size? counts may easily exceed a 32-bit word capacity!}
%
data word.  In general, the program profile is stored at level of basic
blocks: for every block one execution count is stored.  Some basic blocks,
however, may require several counts, because they may be entered at
different instruction offsets.  This is the case, for example, of procedure
entry blocks (see Figure~\ref{fig:call-profile}).  Depending on scheduling
conditions, one or more instructions of the called procedure may be
duplicated and imported into the call delay slots of the call site.  Thus, a
procedure called 1000 times may be entered, for example, 500 times from the
first instructions (offset 0), 350 times with offset 1, 150 times with offset
2.

The first part of the profile section contains \emph{n} execution counts for
each procedure, where $n-1$ is the number of delay slots of call operations
of the target architecture.  The counts are ordered in ascending order of
procedure address.

The second part of the profile section contains one execution count for each
basic block of the program, starting from the lowest program  address.  The
counts are ordered in ascending order of block address.  The execution count
of procedure entry blocks contains the sum of the execution counts at all
the offsets of the block.

The number of execution counts stored in a profile section depends on the
number of procedures and basic blocks, neither of which is stored explicitly
in the section; those numbers are implied by the program control flow.
Similarly, the number of execution counts per procedure entry \emph{n}
depends on the target architecture and is not explicitly stored in the
profile section.

The current version number and the oldest compatible version number of the
profile data section are stored in the section header.

\subsection{Machine State Snapshot}
\label{ssec:sim-snapshot}

The precise information contents and data format of the machine state
snapshot depends on the model of the target processor state, and are out of
the scope of this document.  The specifications of the target processor
state model are given in~\cite{SimulatorDesign}.

The machine state snapshots contain sufficient information to completely
recreate the target processor state at the time the program simulation was
stopped and the snapshot file was created.

\paragraph{Auxiliary information}
In addition to information strictly related to the processor, the machine
state snapshot contains auxiliary information to enable the Simulator to
completely
recreate its internal state.%
\note{DISCUSS}
%
This information includes the cycle count at the time the snapshot was taken
and stack frame state.

The machine snapshot file contains also numerical codes to indicate whether
the machine snapshot data has been altered since it was created by the
Simulator.

The machine snapshot file contains also the modification time stamps the TTA
Application file and the Machine Definition File.  These time stamps are
compared with the time stamps of the files; if the time stamps mismatch, the
the Simulator issues a warning, and nevertheless attempts to read the file.%
\footnote{The motivation for a simple warning instead of a fatal failure is
that a large space of modifications to a given TTA Application file and a
given ADF do not make the state snapshot invalid.}

The machine snapshot also contains current version number and oldest
compatible version number.  A Simulator can read back a snapshot file with
current version number greater than its own, as long as the oldest
compatible version number of the file is smaller or equal.  In this case,
the Simulator ignores any data fields that it does not recognise.  The
oldest compatible version number guarantees that the new, unknown data
fields can be safely ignored. It is responsibility of the designer of the
data format modifications to determine the oldest compatible version.

\section{Intensity of Use}

The simulator is a one-user product and is expected to be used intensively.

When driven by another application, use intensity is even greater: the
simulator could be invoked thousands of times in a single run batch (for
example, during an exploration of the target architecture space).

\section{Files and Configuration}
\label{ssec:config-files}

The Simulator accesses the following data files:
\begin{enumerate}
\item %
  TTA application, a binary TPEF file (see~\cite{TPEF-specs} for TPEF
  specifications).  The input TPEF file is also used to store the execution
  profile obtained from simulation.  Execution profiles are stored in
  dedicates auxiliary TPEF sections.
\item %
  Definition of the target processor architecture, a textual file in ADF
  format (see~\cite{ADF-specs} for ADF specifications).
\item %
  Execution trace, a database.  The detailed contents of information of
  execution traces, outlined in Section~\ref{ssec:exec-trace}, is specified
  in~\cite{SimulatorModDesign}.
\item %
  Snapshot of the target processor state at a given cycle, a textual%
  \note{PENDING} file in [[TO DO]] format (see~\cite{MSM-design} for [[TO
  DO]] specifications).
\item %
  Log of a simulation control language work session, a Tcl script with
  extended commands of the simulator control language commands, specified in
  Section~\ref{sec:debug}.
\end{enumerate}

In addition to above listed input and/or output files, the GUI-based
Simulator can read a configuration file which controls initialisation of the
simulation environment and the configuration of the graphical interface.

The simulation setup includes parameters that control stack initialisation,
simulation tracing (verbosity, type of information traced), type and amount
of statistics that are collected.

The simulation setup parameters contained in the configuration
file must mirror exactly the command line options provided by the CLI-based
Simulator (see Chapter~\ref{ch:product-features}).



\chapter{OTHER PRODUCT REQUIREMENTS}

\section{Performance Requirements}

The slowdown factor of the Simulator (interpreted) should not exceed a
factor 1000; the goal to aim for is a slowdown factor of 100.

Response time to user activity in case of GUI-based Simulation must be
within the limits that match normal human interaction speed (within few
hundreds milliseconds).

For a compiled Simulator (see Section~\ref{sec:further-ideas}) the slowdown
factor achievable should be around 10.



\chapter{DOCUMENTATION}

\section{User Manual}

%% Remove this section (not needed) for functional specifications of
%% modules or other parts of a larger system that don't have user manual.
The Simulator is fully documented by a users' manual. 

\section{Online Help}

%% Remove this section if not applicable. Only applications and, as general
%% description, larger subsystems of a project have online help.
The GUI-based Simulator front-end has a built-in online help hypertext
documents that can be browsed by invoking an external HTML browser.

The CLI-based Simulator provides an on-screen summary of the command line
options it accepts and their default values (if applicable).

\section{Design Document}

The Design Document describes the software architecture of the Simulator
from a developer's perspective.  The design choices taken may be motivated
when several nonobvious alternatives are possible.

\section{Installation and Configuration Guide}

%% Remove this section (not needed) when the functional specifications refer
%% to a module or application of a larger project.  May apply to some
%% complex subsystems of a larger project when installation.
The installation of the Simulator is an integral part of the TCE system, and
is documented in the TCE Installation and Configuration
guide~\cite{ProjectPlan}.

\section{System Reference and Developer's Guide}

%  Usually, application programming interface (API) hypertext reference
%  generated by Doxygen. Sometimes also a separate manual (or part of
%  another manual).
System reference of the Simulator consists of an introductionary document
(based on the design notes written during development) and a complete
documentation of the application programming interface (API) is provided in
the form of automatically generated hypertext and is integrated
within the TCE API documentation.%
\note{DISCUSS: better separate API docs of each application?}

%  Usually (always?) must refer to a companion document of these functional
%  specifications, less formal but more extensive in depth, about
%  nonfunctional specifications, what we call "design notes".



\chapter{KNOWN PROBLEMS AND RISKS}

\chapter{REJECTED OR ABANDONED IDEAS}

% PEKKA:

% The abandoned requirements should be listed here with the reasoning and
% date the requirement was dumped. This is for future reference.
%
%11.11.2003 HTTP connection to external server to fetch something. 
%           This is simply not sensible and should be forgotten.

\begin{description}
\item[04.08.2005 --- Canned Command Sequences.]%
  They are not providing any extra benefit over normal Tcl functions, as far
  as I see it. --P.J.
\end{description}

\chapter{IDEAS FOR FURTHER DEVELOPMENT}
\label{sec:further-ideas}

% Ideas that are not part of the specifications yet but that might be
% required in future are listed here. This is to "leave room" in design for
% later extensions and to have ideas written down somewhere.  This list
% should contain the date of addition, the idea described briefly and the
% inventor of the idea.

\begin{description}
\item[10.12.2003]%
  Compiled simulator for faster simulation.  A compiled simulator minimizes
  the time spent on repetitive tasks such as the simulation of instruction
  fetch and decode.  A compiled simulator hard-codes in a single executable
  application the target TTA processor and the target application.

  A sophisticated compiled simulator could be compiled only partially,
  during simulation (just in time compilation).
\item[07.01.2004]%
  Support abbreviations of simulation control language commands in the same
  way as GDB does.  At least, support one-character aliases of most used
  commands.
\item[07.01.2004]%
  Support user-defined aliases.
\item[08.01.2004]%
  Allow function calls and other expressions that may have side-effects in
  all commands (such as \emph{print}) that accept an expression as argument.
\item[08.01.2004]%
  Command \emph{shell} to run shell commands using the shell given by SHELL
  environment variable or, by default, `/bin/sh'.
\item[08.01.2004]%
  Command completion and symbol name completion with TAB key.  The latter
  may be not so easy to implement.
\item[12.01.2004]%
  Commands to navigate the source code: \emph{list}, \emph{search},
  \emph{forward-search}, \emph{reverse-search}.  The command \emph{list}
  displays a fixed number of source lines centred at the current program
  position.  The search commands takes a regular expression as argument and
  and shows the first source line that matches the expression.
\item[12.01.2004]%
  The \emph{source path} list contains the paths searched by the Simulator
  when looking for the source text of the program currently loaded.  The
  TPEF program file contains either a reference to the source files path or
  the whole source code as embedded symbolic information, so setting the
  source path list is not normally needed.  This capability is useful when
  the source files are moved from their original position.
\item[12.01.2004]%
  Support for ``artificial array'': given a variable \emph{v} of type
  ``pointer to \emph{t}'', the expression `\emph{v}@\emph{num}' displays
  \emph{num} elements of type \emph{t} starting with the first element
  located at \emph{v}.
\item[14.01.2004]%
  Extend live range statistics to include live ranges that are (partly)
  spilled to memory.  This extension may require auxiliary information
  (annotations?) in the TPEF files.  Extend also with register spilling
  information: ratio of spilled vs. non-spilled GPR's.  This may also require
  auxiliary information.
\item[15.01.2004]%
  Extend run time memory usage statistics with: maximum amount of free-store
  used, average block size of each memory allocation.
\item[15.01.2004]%
  Extend simulation statistics about processor utilisation with resource
  sharing ratios: socket sharing and, for architectures with
  \emph{multicasts}, bus sharing.  These statistics may be computed off-line
  with the data from the simulation trace.
\item[16.01.2004]%
  Extend \emph{bp} command (and other commands that can have a function
  name as argument) with support for overloaded functions.  This extension
  is useful if C++ source language is supported.
\item[16.01.2004]%
  Support C++ name demangling.
\item[16.01.2004]%
  Pretty-print data structures with indentation and line breaks to increase
  readability (\emph{set print pretty} command).
\item[16.01.2004]%
  Capability of altering program execution and processor state.  This
  feature is tightly related to evaluation of expression with side effects.
  An equivalent to command \verb|print myvar=5| which sets the program
  variable `myvar' without printing out it (new) value would be the command
  \emph{set variable}: \verb|set variable myvar=5|.  Altering values is not
  limited to variables defined in the source code: also processor storage,
  such as GPR's, function unit registers, immediate registers, can be set by
  similar \emph{set} commands.  For example, by setting the program counter
  register \$pc, it is possible to have to simulation jump to an arbitrary
  address.
\item[16.01.2004]%
  Support for advanced canned sequences of commands.  To make user-defined
  commands (that is, canned sequences of commands), it is necessary to
  provide basic conditional execution and control flow commands \emph{if}
  and \emph{while}.  In addition, provide facilities to define and retrieve
  help documentation of user-defined commands.
\item[28.01.2004]%
  Capability of altering the memory. This is an extension of the capability
  of altering the processor state to memory addresses.  The syntax for
  altering memory is to be decided.  One possibility, if source data types
  are fully supported, is the syntax:
  \begin{quote}\tt
    \verb|{|\parm{type}\verb|}| \parm{addr} = \parm{value}
  \end{quote}
  where \emph{type} is the type of the object stored at address \emph{addr},
  irrespective of what kind of data is supposed be located at \emph{addr}.
\item[06.04.2004]%
  Possibility to initialise the stack with data loaded from a separate file
  `stack.data' before the processor starts to run the program.  The data
  file could have the encoding and alignment rules specified in TPEF
  specification document~\cite{TPEF-specs}, Section~16.  Alternatively, the
  data file could mimic exactly the format of a machine snapshot file memory
  area.  At least on personal computer operating systems, the environment
  area contains a sequence of NULL-terminated strings, therefore encoding
  and alignment presents no difficulty.
 
\end{description}



\chapter{PENDING ISSUES}

% Pending issues concerning these specifications, a sort of TODO list.
% This chapter should be empty when the final product is ready.

\section{Memory Stack}
\label{ch:pending:stack}

(\emph{to be moved to document about architecture conventions when
available}). In case the ADF defines multiple address spaces that can be
used to store data, how to select which address space is used for the stack?
The Simulator has no choice, because the A.S. of the stack is implicitly
determined by the memory address and the unit that performs the load/store.
Here are possible options:
\begin{itemize}
\item Additional parameter in ADF.
\item Command line option of scheduler/program transformation tool.
\item Use of some default rule (for example, first address space declared).
\end{itemize}

The first options is not convincing, since the position of the stack is an
architectural convention, not an intrinsic feature.

\section{Simulation Statistics}
\label{ch:pending:stats}

Should the Simulator allow a fine-grained control on the simulation
statistical data that is stored in a file?  There are several levels of
control, from coarser (0, already supported) to finer (4):
\begin{enumerate}
\item %
  Whole file: enable/disable generation of file with statistical data.
\item %
  Individual sections of file: performance, utilisation, program statistics.
\item %
  Entry type: bus segment utilisation, data transport counts, upper bounds,
  etc.
\item %
  Entries (for example, per processor component): bus ``X'' utilisation
  count, squashed predicated data transports, parallelism upper bound, etc.
\end{enumerate}
%
For some of the data entries collected, disabling computation of counts
during simulation may reduce the simulation time.  The main disadvantage of
finer levels of control is that they complicate the simulator interface
(lots of command line options).  No matter how complex such interface, it is
unlikely that any filtering and navigation wanted by users is supported.

The alternative assumed at the moment is that all data is stored
unconditionally and that a helper application (possibly with graphical
interface) allows to navigate and filter the statistical data in full
freedom.

\section{Function Unit Utilisation}
\label{ch:pending:fu-util}

Problem: how to compute function unit utilisation? Depending whether a unit
is (fully) pipelined or not, different formulae give different results.
There are three possibilities:
\begin{enumerate}
\item %
  Simple trigger count ratio.
\item %
  Pipeline stage utilisation ratio.
\item %
  Throughput ratio.
\end{enumerate}

The first is given by the simple formula:
\begin{equation}
  \label{eq:trig-ratio}
  \frac{N_{ops}}{N_{cycles}}
\end{equation}
where $N_{ops}$ is the total trigger count for a given unit and $N_{cycle}$
is the (execution) cycle count of the entire application.

The second is given by:
\begin{equation}
  \label{eq:pipe-ratio}
  \frac{N_{ops}\cdot L}{N_{cycles}\cdot P}
\end{equation}
where $L$ is the number of cycles the operation takes to complete and $P$ is
the number of pipeline stages that are used by the operation.  This formula
can be extended to the case in which a function unit implements several
operations with differing values of $L$ and $P$; in that case, the results
obtained for each operation with (\ref{eq:pipe-ratio}) are simply added
up.  The problem of this formula is that the upper bound ratio, for certain
operations, may be lower than 1, and requires then some analysis to be
interpreted.

The third is given by:
\begin{equation}
  \label{eq:fu-throughput}
  \frac{N_{ops}}{\frac{N_{cycles}}{L'}}
\end{equation}
where $L'$ is the number of cycles, following a trigger, after which the
function unit may be triggered again.  Also this formula can be easily
extended to the case in which a function unit implements several operations.
The problem with this formula is that, in general, the value of $L'$ depends
on which operations are triggered in first and second position and is
difficult to compute precisely.

In summary, equation~(\ref{eq:trig-ratio}) is not giving much information,
equation~(\ref{eq:pipe-ratio}) is of difficult interpretation for units with
complex pipelines, and equation~(\ref{eq:fu-throughput}) is difficult to
compute.

\section{Non-GPR Live Ranges}
\label{ch:pending:bypass-lr}

It could be useful to provide another set of live range statistics that
takes into account live ranges of bypassed data (as opposed to the
conventional live ranges, which are allocated to GPR's).

In GPR live range analysis, definitions of a GPR are easily recognised as
the start points of a live range allocated to that GPR.  Each definition
corresponds to a data transport that writes that GPR.  Non-GPR live range
analysis is slightly more complicated, because the definition is never
explicit.  It must be computed based on two factors:
\begin{itemize}
\item the cycle in which the operation is triggered,
\item the cycle in which the result is ready.
\end{itemize}
The second item requires knowledge of the operation latencies; simple
dataflow analysis of the program is not sufficient.

\section{Memory Implementation and Processor Architecture}
\label{ch:pending:mem-latency}

The details of memory implementation and the memory characteristics are not
part of the processor architecture.  The number of stall cycles due to, for
example, bank conflicts on a memory system in which the number of requests
on a given bank exceeds the number of available memory ports is not an
architectural property: the stall cycles are ``invisible'' to the program.

Yet, it may be crucial, for effective operation of the instruction scheduler
and the code optimisation passes, to know what is the maximum number of
memory accesses that can be placed simultaneously and what is the expected
latency in case of a cache miss.

Since this information is not architectural, it should not be stored in
ADF~\cite{ADF-specs}.  The TTA programs for a processor configuration
specified by a ADF can work perfectly well with radically different
implementations of the same address space described in the ADF.

Where to put this information?


% ------------------------------------------------------------------------

%  References are generated with BibTeX from a bibtex file.
\bibliographystyle{alpha}
\cleardoublepage
%% Equivalent to a chapter in the table of contents
\addcontentsline{toc}{chapter}{BIBLIOGRAPHY}
\bibliography{Bibliography}


\end{document}

