\documentclass[a4paper,twoside]{tce}
\usepackage{pslatex}

\RequirePackage{algorithm}
\RequirePackage{algorithmic}

\begin{document}
\author{Tommi Rantanen}
\title{TTA Processor Estimator}
\ver{0.4.2}
\firstday{10.11.2004}
\lastday{27.04.2006}
% id number in S- sequence
\docnum{037}
% draft/complete/committed
\state{draft}

\maketitle


\chapter*{Document History}

\begin{HistoryTable}

0.1    & 10.11.2004 & T. Rantanen   &
First draft. Initial version of the design.\\
0.2    & 13.01.2005 & J. Mantyneva  &
Added first draft of CostDB schema.\\
0.2.1  & 13.01.2005 & J. Mantyneva  &
Removed the Schema. The same information is found in the requirement. \\

0.3 & 22.09.2005 & P. J��skel�inen &
Updated to HDB and the new estimator design (all algorithms delegated to
plug-ins). Removed empty sections. \\

0.4 & 23.09.2005 & P. J��skel�inen &
Added main design principles. Added plug-in interface of FU cost estimator,
described others. \\

0.4.1 & 29.09.2005 & P. J��skel�inen &
Removed example and code from plug-in description. To be added when
stable.\\

0.4.2 & 27.04.2006 & A. Cilio &
Minor format revision. Spelling corrections. \\

\end{HistoryTable}


\tableofcontents



\chapter{INTRODUCTION}

\section{Purpose and Scope}

This document describes the design of the TTA Processor Estimator
together with the communication with the external modules. The
document is intended for the people implementing and maintaining the
Estimator. However, it does not cover all the details required for the
implementation but it is rather a reference of the design decisions
and principles as well as for the issues that are good to know during
the maintenance.

\section{Definitions}

\begin{description}
\item[Hardware Database (HDB)] %
  Database that includes access point to external cost estimation plug-ins
  and the data the plug-ins need to estimate the costs of TTA processor
  building blocks for different architectural properties.

\item[Cost Model] %
  The way which is used to estimate the costs of the target processor.

\item[Estimator] %
  The target architecture estimator of the TTA templated architecture.

\item[Match Type] %
  The type of the cost database query applied to a characteristic, i.e., a
  field of a database entry. Alternatives are: exact match, superset, subset
  and interpolation.

\item[Simulation Trace Database] %
  Database containing statistics from the simulation run of the target
  application for a given processor.
\end{description}

\section{Acronyms and Abbreviations}

\begin{table}[htb]
\begin{center}
\begin{tabular}{p{0.15\textwidth}p{0.85\textwidth}}
FU    & Function Unit. \\
GCU   & Global Control Unit. \\
GUI   & Graphical User Interface. \\
IU    & Immediate Unit. \\
ADF   & Architecture Definition File. \\
RF    & Register File. \\
TCE   & TTA Codesign Environment. \\
TPEF  & TTA Program Exchange Format. \\
TTA   & Transport Triggered Architecture. \\
\end{tabular}
\end{center}
\end{table}



\chapter{DESIGN OVERVIEW}

\section{``Philosophy of Design''}

One of the flexibility points required from TCE is easy experimentation of
different algorithms for estimating processor costs. This requirement lead
to such design of Estimator that most of the actual algorithmic work is
allowed to be redefined very easily. In addition, all functionality
involving selection of implementations for different machine parts were
decided to be implemented in another library called Implementation Selector.
These two reasons made the design of the actual cost Estimator of TCE is
very simple.

All complex work of actually estimating the costs of individual parts of the
estimated processor is delegated to external cost estimation plug-in modules.
User of Estimator is free to define new estimation plug-in modules in case he
wants to use a new kind of cost estimation algorithm for estimating (parts
of) the costs of the target machine. Estimator performs no intelligent
implementation selection, but expects that the user of Estimator knows (at
least partially) the implementations that are used. In case implementation
information is missing for some of the machine parts, Estimator asks
individual function unit plug-in modules whether they are able to estimate
the costs of the given parts.

Due to the plug-in system, the actual Estimator is merely a top-level
"driver" which loads correct estimation plug-ins, estimates costs of
individual machine parts using the loaded plug-ins, and simply sums the
costs retrieved from the estimation functions to get the total costs for the
given machine.

Estimation functionality can also be invoked for individual parts of the
machine. For example, costs of a single FU can be queried. This
functionality is used by the Implementation Selector while performing its
exploration of processor part implementations.

\section{External Modules/ Application Environment}

Estimator uses services of at least following modules.

\subsection{Base Modules}

\begin{itemize}

\item \textbf{Hardware Database}

  Hardware Database is used as an access point to cost estimation plug-in
  modules. Cost estimation plug-in modules use HDB to retrieve data used in
  the estimation algorithm. Such data can be anything, for example, purely
  measurement data, or costs associated to an operation in an FU.

\item \textbf{Machine Object Model}

  MOM is used to load the given architecture definition file and traverse
  trough the estimated machine parts.

\item \textbf{Implementation Definition File}

  Used to handle the IDF given as input. Estimator asks from this module to
  give an implementation for a machine part, if defined in the file.

\item \textbf{Simulation Trace Database}

  Simulation trace database is used by the estimation plug-ins to get
  simulation data for the program with which the consumed processor energy
  is estimated. TraceDB provides information of the executed instruction in
  each clock cycle and also the values contained in each bus in each clock
  cycle. State transitions consume energy, therefore this kind of
  information is needed in some algorithms.

\item \textbf{Program Object Model}

  For calculating consumed energy by running a program in the target
  processor, estimator algorithms use POM to get access to instruction data.
  Paired with simulation trace, estimation algorithm can compute how many
  times a port was written, for example.

\end{itemize}

\subsection{Toolkit Modules}

\begin{itemize}

\item \textbf{PluginTools}

Used to load and access cost estimation plug-ins.

\end{itemize}

\section{Architecture of Databases}

Estimator depends on IDF, ADF, TPEF, and HDB databases. They are described
in their own documents. In addition, the external cost estimation plug-in
modules can be considered databases, after all, they store functions used to
estimate costs.

This section describes the user-visible plug-in interfaces of the cost
estimation plug-ins.

\subsection{FU Cost Estimation Plug-in}

FU cost estimation plug-in is used to estimate area, energy, and speed of
individual function units. Each plug-in may be able to estimate only one
kind (architecture) of function unit, or arbitrary number of function unit
architectures.

The plug-in interface consists of methods to query the costs associated to a
function unit. The functions return false in case they are not able to
calculate the specific cost for the given function unit architecture
(optionally accompanied with an implementation id). Using this mechanism,
the estimator can query all FU estimation plug-ins in case they can estimate
costs for a FU architecture in case no implementation id is given.

\subsection{RF Cost Estimation Plug-in}

RF cost estimation plug-in (currently) looks exactly like FU cost estimation
plug-in. The interface is identical expect where it stands FU in FU plug-in,
there stands RF in RF plug-in.

\subsection{IC\&decoder Cost Estimation Plug-in}

The plug-in interface is contained in the same plug-in interface which is
used to generate the HDL for the IC\&decoder. The interface is documented
in~\cite{ProGeDesign}.

\subsection{Decompressor Cost Estimation Plug-in}

Decompressor cost estimation interface is documented in~\cite{PIGDesign}.
The cost estimation functionality is part of the plug-in that applies the
compression to a program bit image and generates the HDL decompressor block.

% \section{Module Communication}
% TODO: some sequential diagram of two estimation cases:
% - fully known implementation blocks
% - some blocks do not have implementation (ask from estimation plugins
%   to estimate)

%% Remove this section if the design document describes just one module (or
%% an application consisting of a single module).

% This section describes the communication between modules of the
% application or subsystem documented in this design document.



\chapter{Module Design}

The design of Estimator consists of a frontend class called
\emph{CostEstimator}.

\section{CostEstimator}

\emph{CostEstimator} class is the main class of the cost estimation system.
It loads the cost estimation plug-ins and calculates total cost estimate.
Additionally, it provides interface for partial cost estimation purposes.
Costs can be queried for individual FUs and RFs, and IC/Decoder, and
decompressor.

The class provides the same methods provided by individual plug-in modules.
These methods are used to find an appropriate plug-in module for estimating
the cost for the given machine part and estimate costs using it. In
addition, there are methods for estimating costs of the whole machine.

\end{document}

%%% Local Variables: 
%%% mode: latex
%%% TeX-master: t
%%% End: 
