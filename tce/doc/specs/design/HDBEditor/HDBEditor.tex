\documentclass[twoside]{tce}

\usepackage{pslatex}

\makeindex

\begin{document}
\author{Veli-Pekka J��skel�inen}
\title{HDB Editor}
\ver{0.0.1}
\firstday{8.5.2006}
\lastday{8.5.2006}
\docnum{(undefined)}
\state{draft}
\maketitle

\chapter*{Version History}

\begin{table}[htb]
\begin{center}
\begin{tabular}{|p{0.10\textwidth}|p{0.15\textwidth}
                |p{0.15\textwidth}|p{0.46\textwidth}|}
\hline
\textbf{Version} &\textbf{Date} &\textbf{Author} &
\textbf{Description}\\
\hline
0.1 & 8.5.2006 & Veli-Pekka J��skel�inen & First version.\\
\hline
\end{tabular}
\end{center}
\end{table}

\chapter{INTRODUCTION}

\section{Purpose and Scope}

This document describes the software architecture and the design of the HDB
Editor (Hardware database editor). HDBEditor is a graphical tool for displaying
and modifying information in TCE hardware databases.

\section{Acronyms and Abbreviations}

\begin{table}[htb]
\begin{center}
\begin{tabular}{p{0.15\textwidth}p{0.85\textwidth}}
GUI & Graphical User Interface.\\
HDB & Hardware Database\\
MVC & Model View Controller\\
TCE & TTA Codesign Environment\\
\end{tabular}
\end{center}
\end{table}

\chapter{ARCHITECTURE DESIGN}

\section{Philosophy of Solution}

Philosophy behind the solution is based on the classical MVC design pattern, 
in which system is divided to three entities (model, view, controller). Model 
is the data contents of the system. View is a graphical representation of the 
data. Controller controls all events, and updates the model and view according 
to events.

In HDB Editor, the hardware database files accessed by the HDBManager
interface of the HDB library act as the model. The HDBManager interface
wraps the HDB functionality behind it's interface, and HDB Editor doesn't
have to care about the inner workings of the HDB library or the mechanism
how the data is actually stored. HDBs are accesed only by one application
at a time, so there's no need to track HDBs for outside changes.

View of the HDB Editor consist of the main window and a nubmer of dialogs for
creating and modifying the elements in HDBs. All window and dialog classes
are derived from wxWidgets base classes, and follow the usual wxWidgets
framework.  More information about wxWidgets can be found in \cite{wxWindows}.

Control of the HDBEditor consists of the CommandRegistry classes of the
wxToolkit library. HDBEditor commands are stored in a CommandRegistry
instance, which maps menu items to command objects, derived from the
GUICommand class.


\section{Classes}

Class diagram containing the core classes of HDBEditor is shown in
Figure~\ref{fig:class_diagram}. Most of the dialog and command classes are left
out. \emph{AddFUArchitectureCmd} and \emph{AddFUArchitectureDialog} and
\emph{AddFUEntryCmd}  are examples of command and dialog classes.

\begin{figure}[htb]
\centerline{\psfig{figure=eps/ClassDiagram.eps,width=0.9\textwidth}}
\caption{Core classes of the HDBEditor.}
\label{fig:class_diagram}
\end{figure}


\subsection{HDBEditor}

HDBEditor is the main class of the application, and it represents the
application itself. HDBEditor class is responsible for initializing
the GUI by creating the main frame instance.

\subsection{HDBEditorMainFrame}

HDBEditorMainFrame is the top level parent window of the GUI. It's responsible
for managing the GUI by creating and executing commands from the command
registry when a menu item is selected. HDBEditorMainFrame also creates and
owns the CommandRegistry instance and the HDBBrowser instance, which is
displayed in the main frame.

\subsection{HDBBrowserWindow}
HDBBrowserWindow is a window for browsing hardware database contents.
An instance of the HDBBrowserWindow class is set as the contents of the
HDBEditorMainFrame when the frame is initialzied. HDBBrowserWindow
contains a HDBBrowserInfoPanel instance which displays details of the
selected HDB element in a html-window.

\subsection{HDBBrowserInfoPanel}
HDBBrowserInfoPanel is a html-widget which utilizes HDBToHtml class to create
html-format descriptions of the objects in a HDB.

\subsection{HDBToHtml}
HDBToHtml is a helper class which creates html-format descriptions of
the objects stored in a HDB. This class used by passing a constant
reference to an HDB element and a reference to an output stream to the
static member functions of this class. Detailed description of the
element is then written as html-code to the output stream.

% ------------------------------------------------------------------------

%% Remove this part if there are no references.  Usually there will be at
%% least a reference to the functional specifications of the same module.

%  References are generated with BibTeX from a bibtex file.
\bibliographystyle{alpha}
\cleardoublepage
%% Equivalent to a chapter in the table of contents
\addcontentsline{toc}{chapter}{BIBLIOGRAPHY}
\bibliography{Bibliography}

\end{document}
