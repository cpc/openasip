\documentclass[twoside]{tce}

\usepackage{pslatex}

\makeindex

\begin{document}
\author{Jussi Nyk�nen}
\title{OSEd}
\ver{0.2.1}
\firstday{8.7.2004}
\lastday{12.10.2004}
\docnum{(undefined)}
\state{draft}
\maketitle

\chapter*{Version History}

\begin{table}[htb]
\begin{center}
\begin{tabular}{|p{0.10\textwidth}|p{0.15\textwidth}
                |p{0.15\textwidth}|p{0.46\textwidth}|}
\hline
\textbf{Version} &\textbf{Date} &\textbf{Author} &
\textbf{Description}\\
\hline
0.1 & 8.7.2004 & J. Nyk�nen & 
First version.\\
0.2 & 5.8.2004 & J. Nyk�nen &
Enhanced class diagram. Added a lot more information about implementation.\\
0.2.1 & 9.8.2004 & J. Nyk�nen &
Added information about info view.\\
0.3   & 12.10.2004 & J. Nyk�nen &
Added information about Listener pattern usage.\\
\hline
\end{tabular}
\end{center}
\end{table}

\chapter{INTRODUCTION}

\section{Purpose and Scope}

This document describes the software architecture and the design of the OSEd 
(Operation Set Editor).

\section{Acronyms and Abbreviations}

\begin{table}[htb]
\begin{center}
\begin{tabular}{p{0.15\textwidth}p{0.85\textwidth}}
OSAL & Operation Set Abstraction Layer.\\
GUI & Graphical User Interface.\\
CLI & Command Line Interface\\
MVC & Model View Controller\\
\end{tabular}
\end{center}
\end{table}

\chapter{ARCHITECTURE DESIGN}

\section{Philosophy of Solution}

Philosophy behind the solution is based on the classical MVC design pattern, 
in which system is divided to three entities (model, view, controller). Model 
is the data contents of the system. View is a physical representation of the 
data. Controller controls all events, and updates the model and view according 
to events. In OSEd, model is OperationContainer. OperationContainer contains all
necessary information: search paths, operation modules, and operations. View
is a splitted window, which shows operation data base as tree-like structure in
one window, and additional information on the second. Controller role is played
by two parts: main window controls menu actions, and a tree view controls
actions that take place in its components.

OSEd and OSAL CLI tools need some common functionality (building operation
behaviors, simulating operations). This functionality is separated to two
modules that can be used by both OSEd and OSAL CLI applications. These modules
are \emph{OperationBuilder} and \emph{OperationSimulator}.

Graphic library used to implement OSEd is wxWidgets. More information
about wxWidgets can be found in \cite{wxWindows}.

\chapter{CLASSES}

Class diagram containing the most important classes of OSEd is shown in 
Figure~\ref{fig:class_diagram}. Most of the dialog and command classes are left
out. \emph{OSEdSimulateCmd} and \emph{SimulateDialog} are examples of command
and dialog classes.

\begin{figure}[htb]
\centerline{\psfig{figure=eps/ClassDiagram.eps,width=0.9\textwidth}}
\caption{Main part of OSEd class diagram.}
\label{fig:class_diagram}
\end{figure}

\section{OSEd}

OSEd is the main class of the application. It is inherited from wxApp.
OSEd initializes the application, ie. it creates the main window and reads
the option file.

\section{OSEdMainFrame}

Class that controls the main window of the application. Initializes the 
split window: the left side is a tree view (OSEdTreeView), and right side
is a list of static information (OSEdInfoView). Initializes also the menu bar
and adds all the commands in command registry.

OSEdMainFrame handles all events from the menu.

\section{OSEdTreeView} 

Shows a tree-like structure of search paths, operation modules, and operations.
For that purpose it uses wxTreeCtrl. OSEdTreeView controls the list of static 
information (OSEdInfoView). The contents of static information list depends 
on what is selected on the tree view. OSEdTreeView fetches the necessary 
information about search paths, modules, and operations from OperationContainer.

OSEdTreeView handles two kinds of events:

1. An item is selected. Depending on the item, OSEdInfoView is asked to show
a certain view.\\
2. Right mouse button is clicked over an item. Depending on the item, a different
drop down menu is shown. The handling of menu events is delegated to 
OSEdMainFrame.\\

\section{OSEdInfoView}

Shows static information about search paths, modules, and operations. Uses
wxListCtrl. OSEdInfoView offers the same kind of functionality than OSEdTreeView.
User can select an item by clicking RMB, which brings up a drop down menu.

\section{CommandRegistry}

Holds all possible commands for OSEd. Same class is used also in ProDe 
(Processor Designer).

\section{Common Classes with OSAL CLI}

OSAL CLI applications and OSEd shares same functionality: operation modules
must be build, and operation behavior must be simulated. This functionality
is separated to two classes, OperationBuilder and OperationSimulator. These
classes contains necessary functions to build modules and to simulate
operations.

\section{OSEdOptions and OSEdOptionsSerializer}

OSEdOptions is a very simple class, since the only option is OSEd currently is
the text editor used to modify operation behavior source files. Options are
given to OSEd as XML file. OSEdOptionsSerializer is used to read the XML file.

\section{OSEdTextGenerator}

OSEdTextGenerator is used to offer all the textual information shown to the user:
dialog names, error messages etc. OSEdTextGenerator is inherited from
TextGenerator. More information about TextGenerator can be found in 
\cite{ToolkitDesign}.

\section{OperationContainer}

OperationContainer offers all the functionality needed from OSAL. OperationContainer
uses three classes: OperationIndex, OperationSerializer and PluginTools.
OperationIndex is used to get information of modules and operations, as well as
to add new modules and operations. OperationSerializer is used for reading the
operation property files. PluginTools is used to dynamically load the behavior
of operations. More information about PluginTools can be found in 
\cite{ToolkitDesign}. OperationContainer contains also necessary memory information
for the simulation of the commands. \emph{OperationContext} used by all operations
is also in OperationContainer (maybe it is no longer a good name?). 
Thus, OperationContainer serves all the information needed for editing, adding,
and simulation of operations.

\chapter{Usage of listener pattern}

Most of the dialogs in OSEd are modal, but a few dialogs are not modal. An example
of such dialog is dialog for viewing and editing memory. Dialogs that are not modal
needs somehow to know when occurs events that has an effect on them. This problem
is solved by using Listener pattern. \emph{OSEdInformer} object located in mainframe
serves as Observer, which then informs all the registered listeners that an event
occurred. In the example case memory dialog is the listener. More information about
listener pattern can be found in \cite{DesignPatterns}.

\chapter{Handling Commands}

When an action takes place, CommandRegistry is asked for a corresponding 
command. Command executes itself and, if necessary, directly modifies 
operations or modules in OperationContainer.

% ------------------------------------------------------------------------

%% Remove this part if there are no references.  Usually there will be at
%% least a reference to the functional specifications of the same module.

%  References are generated with BibTeX from a bibtex file.
\bibliographystyle{alpha}
\cleardoublepage
%% Equivalent to a chapter in the table of contents
\addcontentsline{toc}{chapter}{BIBLIOGRAPHY}
\bibliography{Bibliography}

\end{document}
