\documentclass[twoside]{tce}
\usepackage{pslatex}

\begin{document}
\author{Ari Mets�halme, Andrea Cilio}
\title{The Graph Interfaces}
\ver{0.4}
\firstday{12.11.2004}
\lastday{24.01.2006}
% id number in S- sequence
\docnum{023}
% draft/complete/committed
\state{draft}

\maketitle


\chapter*{Document History}

\begin{HistoryTable}

 0.1    & 12.11.2004 & A. Mets�halme &
 Initial draft.\\
 0.2    & 16.11.2004 & A. Mets�halme &
 Added (incomplete) interfaces, made fixes throughout the document. Removed
 empty sections.\\
 0.3	& 10.12.2004 & A. Mets�halme &
 Rewrote Chapter 3.\\
 0.3.1	& 10.12.2004 & A. Cilio      &
 Applied standard formatting, added 2.2, bibliography.\\
 0.4	& 24.01.2006 & A. Cilio      &
 Complete revision. Changed title and scope of document.\\

\end{HistoryTable}

\tableofcontents



\chapter{INTRODUCTION}

\section{Purpose and Scope}

This document describes the suggested interfaces for graph-based toolset
modules or applications. The interfaces described here are not a software
framework proper. The status of these interfaces is that of guidelines for
people designing graph-based modules and their own specialised graph
structures, such as, for example, a PDG~\cite{Ferrante87} for the Scheduler
Framework~\cite{SchedulerArchitecture,SchedulerDesign}. A subset of the
interfaces described here, however, could be implemented as framework proper
infrastructure.

\section{Definitions}

\begin{description}
\item[Graph Interfaces] Guidelines for defining interfaces of graph-based
  structures throught the TCE toolset.
\item[Scheduler]
  The instruction scheduler of the TCE toolset.
\end{description}


\section{Acronyms and Abbreviations}

\begin{table}[htb]
\begin{center}
\begin{tabular}{p{0.15\textwidth}p{0.85\textwidth}}
PDG & Program Dependence Graph \\
TCE & TTA Codesign Environment \\
\end{tabular}
\end{center}
\end{table}



\chapter{FRAMEWORK OVERVIEW}

\section{Philosophy of Design}

The interfaces described here are intended to be generally applicable to any
type of graphs. The primary purpose of these guidelines is to encourage
consistent interfaces in different graph classes of the toolset. 

It also suggessts a minimum set of services that should be available in any
graph, for example addition and deletion of nodes and edges, tests for
connectivity, search algorithms. These services depend only on the
topological structure of the graph, not on the data that may reside in its
nodes and edges.

\section{External Modules and Application Environment}

This section describes the relation of the graph interfaces with the rest of
the system.


\subsection{External Libraries}
\label{ssec:external-libraries}

The interfaces described in this document have been defined taking into
account the \emph{boost-graph} library~\cite{BoostGraph} as a likely base
for the implementations. In case other sensible alternatives for an
operation could have been chosen, the alternative described here is the one
that is more efficient to implement with \emph{boost-graph}.

\subsection{External File Data Formats}
\label{ssec:data-files}

If the graph implementation provides a way to output (serialise) its
information contents, the recommended output format should be
\emph{DOT}~\cite{Graphviz}, which can be processed and rendered into several
graphical formats by a number of publicly available applications.


\subsection{Dependencies to Other Modules}
\label{ssec:other-modules}

The implementations of the graph interfaces may depend on domain or
application-specific structures, as needed. The basic interfaces should not
imply any dependency, not make any assumption about the specialised
implementations. It is very likely, however, that any implementation will
depend on certain components of the toolkit library~\cite{ToolkitDesign}.


\section{Architecture of Databases}

The graph interfaces do not define or require any basic data structure.
However, it is recommended that the output format be \emph{DOT} in order to
ensure interoperability with external applications.

As an alternative to DOT, an XML-based format taking advantage of the base
serialisation infrastructure of the toolkit could be defined, as needed.
Such alternative would minimise the work necessary to serialise node and
edge properties.

In any case, an external representation of the graph structures should allow
for storage and retrieval of node (vertex) and edge properties.



\chapter{MODULE DESIGN}
\label{cha:module_design}

%% This chapter is repeated for each module in case this design document
%% describes a multi-module larger module (or application).

\section{Overview}

% Overview of the module design in concrete terms.

%% Remove this section (redundant) if the design document describes a single
%% module (or an application consisting of a single module).

The root interface of graph data structures is \emph{Graph}. Any
\emph{Graph} implementation is responsible for nodes and edges it contains.
Nodes are defined as implementations of the interface \emph{GraphNode},
while edges implement the \emph{GraphEdge} interface.

If a graph is
directed, it is represented by a specialised \emph{DiGraph} object. A
\emph{UniGraph} instance represents an undirected graph. Furthermore, a
graph that can contain both directed and undirected edges is represented by
a \emph{MixedGraph} object.

For each specialised graph type (a data dependence graph for example),
the required special node and edge classes are inherited from the
\emph{GraphNode} and \emph{GraphEdge} base classes. Most importantly,
the graph class itself is specialised from the appropriate base class
(from \emph{DiGraph} in case of a data dependence graph).

A \emph{Graph} object has full control over the nodes and the edges
and their interrelations. It offers an interface for performing all
actions needed to modify a graph, that is, interface for creating,
adding, deleting, connecting and disconnecting nodes and edges. The
graph can be altered only through the \emph{Graph} interface.

It also offers services for obtaining various graph-related
information: are node A and node B connected, what is the connecting
edge etc. If a service is applicable only for a certain graph type,
that service is not included in the base class interface. Instead, it
will be offered only by the appropriate specialised class representing
that graph type. For example, if a service is only sensible with a
directed graph, only the \emph{DiGraph} class or its subclasses should
offer that service. These services are described in detail in
Section~\ref{sec:interfaces}.

Figure~[TODO]~%\ref{fig:class_diagram}
depicts the class structure of the framework.

%\begin{figure}[htb]
%\centerline{\psfig{figure=eps/class_diagram.eps,width=0.60\textwidth}}
%\caption{The class diagram of the Graph Interfaces}
%\label{fig:class_diagram}
%\end{figure}

\section{Interfaces}
\label{sec:interfaces}

% High level description of the interface of this module.  Low level details
% are possible, but should be restricted to simple classes (like, for
% example, NodeDescriptor in mdf module).  Low level details of large
% interfaces, if present at all, should be stored in a separate UML
% specification file (for example, Rational Rose MDL).

\subsection{Graph}
\label{ssec:graph}

The \emph{Graph} interface stands for a graph and allows to access and
manipulate it as needed and allowed by the specific implementation.

A \emph{Graph} manages \emph{GraphNodes} and \emph{GraphEdges} and their
relations. Graph nodes and edges are added, removed or accessed through the
\emph{Graph} interface. This means that any request for adding or removing a
node or an edge requires and explicit request to \emph{Graph} interface.

\emph{Graph} describes services applicable to any graph, such as traversing
and searches. These services are not dependent of the types of nodes and
edges in the graph. The services described here exclusively depend on the
topological properties of the nodes and edges, and are type-independent. Any
type-related service or bookkeeping activity should be implemented in
separate operations, specific to a graph implementation (or family thereof).
Creation of node and edge instances is also left as a responsibility for the
concrete graph subclasses: the base interface defines only how to connect
and register those instances.

Basic opterations:

\begin{description}

\item[Constructor]%
  The constructor creates an empty graph.

\item[Destructor]%
  The destructor deletes all nodes and edges in the graph.

\item[edgeCount(node : GraphNode\&) : int]%
  Return the number of edges connected to the given node regardless of
  the edge type.

\item[edgeCount(void) : int]%
  Return the total number of edges in the graph.

\item[edge(index : int, node : GraphNode\&) : GraphEdge\&]%
  Return
%
  \note{CHECK--pending \ref{ch:pending:edge-navigation}}
%
  the edge connected to a node in the graph at the given index. Applies to
  all edges, regardless of the type.

\item[edge(index : int) : GraphEdge\&]%
  Return
%
  \note{CHECK - computationally expensive?}
%
  The total the edge of the graph identified by given index.

\item[nodeCount() : int]%
  Return the total number of nodes in the graph.

\item[node(index : int) : GraphNode\&]%
  Returns the node of the graph at the given index. Applies to all nodes,
  regardless of the type.

\item[headNode(edge : GraphEdge\&) : GraphNode\&]%
  Return the first node connected to \emph{edge}. If the edge is directed,
  this should be interpreted as the ``head'' node.

  Throw \emph{InstanceNotFound} if \emph{edge} does not belong to this
  graph.

\item[tailNode(edge : GraphEdge\&) : GraphNode\&]%
  Returns the second node connected to \emph{edge}. If the edge is directed,
  this should be interpreted as the ``tail'' node.

  Throw \emph{InstanceNotFound} if \emph{edge} does not belong to this
  graph.

\item[removeNode(node : GraphNode\&) : void]%
  Delete all edges connected to the given node and then remove it from the
  graph. After this call, the node is not part of the graph.

\item[removeEdge(edge : GraphEdge\&) : void]%
  Disconnect the tail and head nodes of the given edge and remove it from
  the graph. After this call, the edge is not part of the graph.

\item[disconnectNodes(node1 : GraphNode\&, node2 : GraphNode\&) : void]%
  Disconnect two nodes. Remove from the graph and delete all edges between
  them.

\end{description}

The following operations could be public (if construction of independent
nodes and edges is permitted to any client) or protected (if nodes and edges
are, by construction, part of a graph).

\begin{description}

\item[addNode(node : GraphNode\&) : void]%
  Add the given node to the graph. After this call, the node is part of the
  graph.

\item[connectNodes(node1 : GraphNode\&, node2 : GraphNode\&, edge:
  GraphEdge\&) : void]%
  Connect two nodes with the given edge. After this call, the edge is part
  of the graph.

\end{description}

*** TO DO ***

This is implementation issues

\begin{verbatim}
template<GraphNode, GraphEdge>
class GraphBase {
    // common graph interface
};
\end{verbatim}

Typical use for a complete reimplementation of the graph interface:

\begin{verbatim}
class MyGraph : public GraphBase<MyNode, MyEdge> {
    // only needed methods of interface implemented
    // new specialised methods if necessary
};
\end{verbatim}

Note:
\begin{enumerate}
\item %
  Probably no client will ever use directly the (awkward?) base class
  interface \verb|GraphBase<ConcreteNode,ConcreteEdge>| because it is
  obscure, unnecessary (too generic: if a client needs to work at that
  level, then it is probably a toolkit component and could be integrated in
  graph class implementation?).
\item %
  Since the class is a template, the base interface does not force the
  concrete graphs to implement every method.
\end{enumerate}

It is questionable if there is any need for a base templated class.


Another typical use of the graph framework is to derive the specialised
graph class from a concrete but templated implementation of the base graph
interface. For example, the toolset defines a templated implementation based
on the boost graph library (BGL), \emph{BoostGraph}. To use it, a class must
simply inherit and instantiate that template:

\begin{verbatim}
class MyGraph : public BoostGraph<MyNode, MyEdge> {
    // only new specialised methods (if necessary)
};
\end{verbatim}

\emph{BoostGraph} itself is a complete implementation of the graph
interface:

\begin{verbatim}
template<GraphNode,GraphEdge>
class BoostGraph : public GraphBase<GraphNode,GraphEdge> {
    // all methods of interface implemented
    // no specialised public methods
};
\end{verbatim}


\subsection{DiGraph}
\label{ssec:di_graph}

A \emph{DiGraph} instance represents a directed graph. The edges in an
undirected graph are instances of the \emph{DirectedGraphEdge} class.

\begin{description}

\item[DiGraph()]
	The constructor creates an empty graph.

\item[$\sim$DiGraph()]
	The destructor deletes all nodes and edges in the graph.

\item[incomingEdgeCount(node : GraphNode\&) : int]
	Returns the number of incoming edges connected to the given node.

\item[incomingEdge(index : int, node : GraphNode\&) : DirectedGraphEdge\&]
	Returns the incoming edge connected to a node in the given index.

\item[outgoingEdgeCount(node : GraphNode\&) : int]
	Returns the number of outgoing edges connected to the given node.

\item[outgoingEdge(index : int, node : GraphNode\&) : DirectedGraphEdge\&]
	Returns the outgoing edge connected to a node in the given index.

\end{description}

\subsection{MixedGraph}
\label{ssec:mixed_graph}

A \emph{MixedGraph} instance is a graph that may contain both directed
and undirected edges.

\begin{description}

\item[MixedGraph()]
	The constructor creates an empty graph.

\item[$\sim$MixedGraph()]
	The destructor deletes all nodes and edges in the graph.

\item[incomingEdgeCount(node : GraphNode\&) : int]
	Returns the number of incoming edges connected to the given node.

\item[incomingEdge(index : int, node : GraphNode\&) : DirectedGraphEdge\&]
	Returns the incoming edge connected to a node in the given index.

\item[outgoingEdgeCount(node : GraphNode\&) : int]
	Returns the number of outgoing edges connected to the given node.

\item[outgoingEdge(index : int, node : GraphNode\&) : DirectedGraphEdge\&]
	Returns the outgoing edge connected to a node in the given index.

\item[undirectedEdgeCount(node : GraphNode\&) : int]
	Returns the number of undirected edges connected to a node in the
	given index. If no node parameter is given, the index refers to the
	graph itself.

\item[undirectedEdge(index : int, node : GraphNode\&) : UndirectedGraphEdge\&]
	Returns the undirected edge connected to a node in the given index.

\end{description}

\subsection{GraphNode}
\label{ssec:graph_node}

A \emph{GraphNode} instance represents a single node in a graph. It is
not responsible of, nor knows the edges connected to it. The
\emph{GraphNode} class is an abstract base class for concrete node types.

Note that concrete instances of \emph{GraphNode} subclasses should not
be created outside of the \emph{Graph}. The interface of the
appropriate \emph{Graph} subclass offers the required methods for
creation of node objects for that graph type.

\begin{description}

\item[GraphNode()]
	The constructor creates a node with no data or connections.

\item[$\sim$GraphNode()]
	The destructor.

\end{description}

\subsection{GraphEdge}
\label{ssec:graph_edge}

An abstract base class representing a single edge in a graph. It is
always associated with two \emph{GraphNode} objects.

Note that concrete instances of \emph{GraphEdge} subclasses should not
be created independent of a \emph{Graph} instance. The interface of
the appropriate \emph{Graph} subclass offers the required methods for
creation of edge objects for that graph type.

\begin{description}

\item[GraphEdge(head : GraphNode\&, tail : GraphNode\&)]
	The constructor creates an edge and connects it to two nodes
	in the graph.

\item[$\sim$GraphEdge()]
	The destructor.
\end{description}

\subsection{DirectedGraphEdge}
\label{ssec:digraph_edge}

Represents a directed edge between two nodes.

\subsection{UndirectedGraphEdge}
\label{ssec:unigraph_edge}

Represents an undirected edge between two nodes.

\section{Error Handling}

% How are errors handled? What is considered internal and what is considered
% external source of errors? What is handled by throwing exceptions and what
% by assertions? What are the 

Certain graph operations may result in error conditions.
%
All errors due to topological properties of the graph (for example,
requesting the outgoing edges of an independent node to a graph) should be
signalled by a consistent and restricted set of exceptions.

The following exceptions from the TCE toolkit are suggested:
\begin{enumerate}
\item \emph{InstanceNotFound}:%
  whenever the operation requested to a graph involves a component that is
  supposed to be in a graph is in fact not present (independent or in
  another graph).

  Example: \emph{Graph::tailNode(edge)} on an edge that does not belong to
  the graph.

\item \emph{ObjectAlreadExists}:%
  whenever the operation requested tries to add a component that is already
  part of the graph.

  Example: \emph{Graph::addNode(node)} on a node that already belongs to the
  graph. Notice that this situation is impossible by design if nodes are
  added automatically by their constructors.
\end{enumerate}



\chapter{REJECTED ALTERNATIVES}

\begin{description}
\item[24.01.2006] --- A graph software framework proper.

  A framework proper, that is, an actual base abstract class and a set of
  services and base classes to be implemented by all instances of concrete
  graphs in the toolset, has been rejected in favour of a nonbinding set of
  interface guidelines.

  Motivations:
  \begin{enumerate}
  \item %
    Most of the generally useful operations (topological sorts, dominance
    computation) cannot be implemented (with reasonable efficency) in the
    base classes without assuming an underlying implementation (for example,
    the Boost graph library. An implementation, however, can be tuned for
    and is affected by the type of graph (mutable or constant, undirected,
    directed, with or without parallel edges).
  \item %
    Most of the generally useful operations may be needed only by a
    restricted subset of the specialised graphs. All the work of defining
    base interfaces might actually be restrictive and annoying extra methods
    for most implementations, which just have to disable them in order to
    keep their interface lean.
  \end{enumerate}

\item[27.01.2006] --- Interface for graph nodes and graph edges.

  Methods to walk inside the graph are defined in the base interface of
  nodes and edges. For example, edge's \emph{target} would return the target
  node, and node's \emph{outEdges} would return the outgoing edges connected
  to the node. This brings about several problems.

  It is not possible for the node and edge subclasses of a concrete graph to
  implement these methods independently. To access the non-public parts of
  the graph, the subclasses must be declared friends (it it not enough that
  their base node and edge classes are). This is problematic, since
  subclassing of node, edge and graph types is open-ended.

  So, the methods in node and edges must be implemented generically in the
  base (templated) graph class. For this to work, either the methods in
  graph base class are public or the node and edge base classes must be
  declared \emph{friend}.

  In the former case, the node and edge methods are mere shortcuts to the
  equivalent graph methods, resulting in redundancy. In the latter case,
  only the base node and edge classes have access to the graph non-public
  parts (\emph{friend}-ness is not inherited).

  Another drawback of this solution is that every node and edge object must
  hold a backpointer to the owning graph, and that every node and edge type
  must be derived from the base classes, thus making the graph framework
  less generic.

  Even the mere empty interfaces for nodes and edges might be problematic:
  any method that returns a (reference to a) node or edge must necessarily
  return the base classes. However, concrete graph subclasses will contain
  specialised nodes and edges. Their clients will probably have to access
  the specialised API of these nodes and edges most of the time. So, every
  graph class should define wrappers that dynamically cast the returned
  nodes and edges to the concrete subclasses. However, this is not possible
  due to C++ binding rules, which ignore the return type. As a result, using
  base node and edge nodes would require ubiquitous \verb|dynamic_cast| in
  client code or a further wrapper graph class, thus making the graph
  subclass not readily and fully usable.

\end{description}


%\chapter{IDEAS FOR FURTHER DEVELOPMENT}

% Ideas that are not part of the design yet but that might be added in the
% future are listed here. This is to have ideas written down somewhere.
% This list should contain the date of addition, the idea described briefly
% and the inventor of the idea.

% Example:
%
% 18.11.2003 . . . -P.J��skel�inen



\chapter{PENDING ISSUES}

\section{Navigation of Edges}
\label{ch:pending:edge-navigation}

The currently suggested interface for navigating though edges (outgoing,
incoming) of a node, or parallel edges connecting a node, is potentially
extremely inefficient, because it requires to redo, internally, the same
search at each request.

A more efficient alternative would be a system based on edge iterators.
However, this must be carefully designed to prevent ``leaks'' of
implementation details (for example, boost templated types) into general,
public interfaces.


%\chapter{MAINTENANCE}

% This chapter treats the problem of extending the design with new
% capabilities that fit in a predefined framework.

% For example, in case of the TPEF module, this chapter could describe the
% steps that must be taken in order to add a new Section subclass and/or a
% SectionReader subclass.

% Remove this chapter no obvious and standardised way to extend the design
% with new capabilities exists.


% ------------------------------------------------------------------------

%  References are generated with BibTeX from a bibtex file.
\bibliographystyle{alpha}
\cleardoublepage
% Equivalent to a chapter in the table of contents
\addcontentsline{toc}{chapter}{BIBLIOGRAPHY}
\bibliography{Bibliography}



\end{document}

%%% Local Variables: 
%%% mode: latex
%%% mode: auto-fill
%%% TeX-master: t
%%% End: 
