\documentclass[a4paper,twoside]{tce}
\usepackage{pslatex}

\begin{document}
\author{Pekka J��skel�inen}
\title{TCE Toolkit Library}
\ver{0.12}
\firstday{05.04.2004}
\lastday{23.01.2006}
% id number in S- sequence
\docnum{029}
% draft/complete/committed
\state{draft}

\maketitle


\chapter*{Document History}

\begin{HistoryTable}

 0.1    & 05.04.2004 & P. J��skel�inen &
 Initial version. PluginTools design.\\

 0.1.1  & 13.04.2004 & P. J��skel�inen &
 PluginTools: change to search path list. Minor cleanups.\\

 0.2    & 16.04.2004 & P. J��skel�inen &
 Chapter about \emph{TextGenerator}.\\

 0.2.1  & 21.04.2004 & J. Nyk�nen      &
 Chapter about \emph{CmdLineOption} and \emph{CmdLineOptions}.\\

 0.2.2  & 22.04.2004 & A. Cilio        &
 General review. Restructured 2.4, 4. Corrected several minor errors in
 command line option framework.\\

 0.2.3  & 29.04.2004 & A. Cilio        &
 Minor corrections in command line options and PluginTools (renamed
 method).\\

 0.2.4  & 19.05.2004 & P. J��skel�inen &
 PluginTools now uses RTLD\_LOCAL instead of RTLD\_GLOBAL.\\

 0.3    & 19.05.2004 & A. Oksman       &
 Added chapters ``XMLSerializer'' and ``IDEAS FOR FUTURE DEVELOPMENT''.\\

 0.3.1  & 20.05.2004 & A. Oksman       &
 Added a bit more information on the implementation of a new serializer.\\

 0.3.2  & 26.05.2004 & J. Nyk�nen      &
 Updated the chapter about \emph{TextGenerator}, now implemented with boost
 library.\\

 0.4    & 12.01.2005 & P. J��skel�inen &
 Moved \emph{EventHandler} from Simulator design.\\

 0.4.1  & 28.09.2005 & A. Cilio        &
 Assigned document number. Minor corrections.\\

 0.5    & 28.09.2005 & A. Cilio        &
 Updated obsolete parts, 2, 5. Corrected PluginTools.\\

 0.5.1  & 30.09.2005 & A. Cilio        &
 Corrections in PluginTools.\\

 0.6    & 05.10.2005 & A. Cilio        &
 Corrections to PluginTools visible behaviour.\\

 0.7    & 13.10.2005 & A. Cilio        &
 Added stub chapter ``Application Services''.\\

 0.8    & 13.10.2005 & A. Cilio        &
 Added chapter ``Text Logger Server''.\\

 0.9    & 21.10.2005 & A. Cilio        &
 Reorganised and cleaned up option system.\\

 0.10   & 30.10.2005 & A. Cilio        &
 Revised text generator (old design deprecated). Complete redesign of
 logger.\\

 0.11   & 24.11.2005 & A. Cilio        &
 Added support for late symbol resolution to plug-in tools.\\

 0.12 & 23.01.2006 & A. Cilio &
 Complete redesign of option system by J. M�ntyneva.\\
\end{HistoryTable}


\tableofcontents



\chapter{INTRODUCTION}

\section{Purpose and Scope}

This document describes the design of non-trivial modules in the TCE Toolkit
library.  Purpose of this document is to provide enough detail about the
library to make implementation of the modules straightforward.  Also, this
document may be used as a reference manual for developers of code that uses
the services provided by the toolkit.

\section{Definitions}

%\begin{description}
%\item[term]%
%  term definition
%\end{description}

\section{Acronyms and Abbreviations}

\begin{table}[htb]
\begin{center}
\begin{tabular}{p{0.15\textwidth}p{0.85\textwidth}}
 API  & Application Programming Interface\\
 DOM  & Document Object Model\\
 ADF  & Architecture Description File\\
 SAX  & Simple API for XML\\
 TCE  & TTA Codesign Environment\\
 TTA  & Transport Triggered Achitecture\\
 UI   & User Interface\\
 XML  & eXtensible Markup Language\\
\end{tabular}
\end{center}
\end{table}



\chapter{PluginTools}

The module PluginTools provides a convenient API for accessing dynamic
modules, which are often used for providing runtime loaded, ``plug-in''
functionality to programs. PluginTools acts as a ``portability layer'' to
system dependent function calls that handle dynamic libraries. PluginTools
is also the name of the only class provided by this module, which is not
static and must be instantiated at least once by clients that want to gain
access to its services.

\section{Main Design Ideas}

Using the PluginTools module should be as easy as possible. The module
provides templated convenience methods to handle the casting of imported
symbols to correct types and also a ``registry'' of opened plugin modules to
make it possible to request new symbols from already opened plugin modules.
In addition, the module provides methods to add search paths for plugin
modules. Search paths are used when loading a plugin module without an
absolute path.
%
Using the search path mechanism, clients of PluginTools can load a plugin
module without knowing its absolute path and referring to it using only the
module name.

The search path list is only used when registering dynamic modules.  Symbols
are only searched in modules that are in the registry.  This avoids
confusion and nasty errors when there are multiple modules with the same
file name.

\subsection{Symbol Resolution and Binding}

By default, all symbol references inside a plug-in module are resolved as
soon as the module is opened, and the symbol space searched for the
resolution does not include any symbols that will be available in modules to
be linked later.

An alternative behaviour is: symbol resolution occurs later (lazy
resolution), usually either when the symbols that contains unresolved
references is imported or when the code containing the unresolved references
is run, and takes into account all dynamic modules (global resolution) that
have been linked so far.

When
%
\note{DISCUSS: construction time or run-time method?}
%
clients need to link several dynamic modules in any order and the modules
can contain references to each other's symbols, lazy and global resolution
may be enabled at construction time.
%
The default behaviour is preferred because it makes error detection easier
and errors more predictable. Lazy resolution can cause missing symbol errors
later, when the symbol imported from the module is used.

\section{Interface}

PluginTools provides a public interface to register and unregister dynamic
modules, load symbols from registered modules and manage the path list used
to search modules.

\begin{description}
\item[PluginTools(modeLazyGlobal : bool)] %
  Contructor. Create an instance of plug-in loader with the given resolution
  and binding mode (local and immediate versus global and lazy). If the mode
  flag is not specified, the default is local and immediate.
\end{description}

\paragraph{addSearchPath(searchPath : string)}
Add a new search path to PluginTools search path list. This search path
list is gone through from the first added path to the last when a plugin
module without absolute path is registered. When the given path is relative,
the search path is expanded to an absolute path starting with the current
working directory.

Exception is thrown if the path does not exist.

\paragraph{removeSearchPath(searchPath : string)}
Remove a search path from PluginTools path list.

Trying to remove a path that does not exist in the path list is not
considered an error.

\paragraph{clearSearchPaths()}
Delete all search paths from the path list. Should never fail.

\paragraph{registerModule(module : string)}
Register a plugin module into the PluginTools registry. If the module name
\emph{module} does not have an absolute path, then the module is searched in
the list of search paths. When found, the module is opened and inserted into
a map where the key is the absolute path of the module file, and the value
is the handle to the opened dynamic module, which is the \emph{void*}
pointer returned by dlopen() call.

In case dynamic module file is not found or dlopen() fails (for example,
because of unresolved symbols), an exception is thrown. In case dlopen()
fails, an error string is retrieved from dlerror() and copied to the
exception's error string. Multiple registration attempts of the same module
are just ignored and not considered an error condition.

\paragraph{unregisterModule(module : string)}
Unregister a plugin module from the PluginTools registry. If the module
name \emph{module} does not have an absolute path, but a relative path, then
the registry is scanned for any absolute path that contains the module name
in the given relative path. If only one absolute path is found, it's
unregistered from the registry. The search path list is not used in this
method to look for the module: only the keys in the registry map are
scanned.

Unregistration means removing the entry from the registry (map) and 
calling dlclose() with the dynamic module handle in that entry.

In case dlclose() fails for some reason, an exception is thrown with
dlerror() contents in it. Exception is thrown also if no absolute path is
given as \emph{module} and either several modules with different absolute
paths or no module at all are found in the registry.

It is not possible to use symbols imported from a module after it is
unregistered!

\paragraph{unregisterAllModules()}
Unregister all modules in the registry. See previous method for definition
of unregistration. This is also called in the destructor.

\paragraph{importSymbol(symbolName : string, target : TypeName*, module :
string)}
\paragraph{importSymbol(symbolName : string, target : TypeName*)}
Templated methods to import a symbol. 

The first version of this method requires explicit definition of the module
the symbol should be loaded from, while the latter uses the registry to look
up the symbol.

The first module found in the registry providing the symbol with 
given \emph{symbolName} is chosen in the latter case. Note that
the order of modules in the registry is not the same as the order in which
they were registered. Latter method is practically a convenience method for
cases where there are only a few registered modules and user is sure that
right one is always picked for the symbol.

In the explicit \emph{module} version, an exception is thrown 
in case no absolute path is given as \emph{module} and multiple 
registrations for that name with different absolute paths are found in the 
map of registered modules. If the module is not found in registry, it's first 
registered using the registerModule() (thus its exceptions may leak from
this function too), and then symbol is returned.

In case dlsym() call fails for some reason, exception is thrown with dlerror()
contents in it.

\verb|TypeName| is a type and is the template parameter.  These templated
methods are convenience methods that call a protected method which does the
actual dlsym() call.  These methods cast the \verb|void*| from dlsym() to
the \verb|TypeName*| type.

\section{Implementation}

\subsection{External Library}
\label{ssec:extern-lib}

The current implementation of the module depends on \emph{dlopen},
\emph{dlsym}, \emph{dlerror} and \emph{dlclose} functions declared in
`dlfcn.h' header. These functions are defined in a standard library that
should be availabe in Solaris and Linux operating systems.

By default, when opening a dynamic module with \emph{dlopen}, flags
`RTLD\_LOCAL' and `RTLD\_NOW' are used. The first flag prevents symbols
imported from the dynamic module to be used by dynamic modules loaded after
it. The second flag indicates that any symbols that are referred to by a
symbol (a function) of the dynamic module must be resolved as the plugin
module is opened. On the contrary, the opposite flag, `RTLD\_LAZY', resolves
symbols later. How late depends on the implementation of the run-time
linkage mechanism.

\subsection{Templated importSymbol() vs. the Real importSymbol()}

The templated importSymbol() methods consist only of the
\verb|reinterpret_cast| operation on the pointer to the linked symbol.  The
real symbol-loading functionality resides in a non-templated member
function.  This is to hide the ugly casting syntax from clients.

Most of the symbol-loading functionality should reside in a normal function
to avoid code bloat because of the nature of the template mechanism.

\subsection{Multiple Instances of PluginTools and dlclose()}

It is possible for dlclose() to ``fail'' in case there are multiple
instances of PluginTools class in use at the same time and multiple
instances have opened the same plugin module. dlclose() does reference
counting, which means that dlclose() call might not actually close the
module but only decrement its reference counter on every call.  When
reference counter reaches zero the module is finally closed. This
functionality should be realized and handled without producing an error.  A
unit test should be written to test this feature.

\subsection{dlsym() and Symbols with NULL Value}

From the manual pages of \emph{dlopen}:
\begin{quote}
  Since the value of the symbol could actually be NULL (so that a NULL
  return from dlsym() need not indicate an error), the correct way to test
  for an error is to call dlerror() to clear any old error conditions, then
  call dlsym(), and then call dlerror() again, saving its return value into
  a variable, and check whether this saved value is not NULL.
\end{quote}

\subsection{Portable Implementation of Search Paths}

By default, the library function \emph{dlopen} of Solaris and Linux uses
LD\_LIBRARY\_PATH to search for dynamic libraries. This variable, however,
cannot be exploited by PluginTools, because it is not generally portable.
%
For this reason, PluginTools should look up only the custom search paths
defined by the client when a relative path is given.
%
The absolute path is detected when the string representing the path starts
with a directory separator character (in Unix, `/').

\section{Error Handling}

All errors caused by a non-existing file, invalid path or error in dl* calls 
etc. are signalled with exceptions.



\chapter{TextGenerator}

This chapter describes a facility to generate the user interface (UI) texts
displayed in applications. This kind of system is motivated with the need to
easily locate UI texts within the source code.

\begin{quote}\it
  \textbf{Warning} --- The original text generation toolkit facility is
  deprecated and is going to be phased out. Do not use the old API for new
  applications.
\end{quote}

\section{Main Design Ideas}

TextGenerator provides a database of text template strings which are
requested using text ids. Text template strings are processed with
a boost::format function.  This makes it possible to translate texts with
variable parts to texts displayed to user. 

An example of a string template:

\begin{verbatim}
Your query "%s" returned %d result(s).
\end{verbatim}

This string template is formatted to a string by providing the values
to the `\%s' and `\%d' parameters of the template to produce a result
of something like:

\begin{verbatim}
Your query "SELECT * FROM trace_data" returned 493 result(s).
\end{verbatim}

Base class \emph{TextGenerator} includes the basic database and formatting
functionality, and some general text templates. Instead of using
TextGenerator directly, each application defines a subclass of TextGenerator
with application-specific capabilities. For example, the Simulator could
define SimulatorTextGenerator, and augments the database with application
specific text messages. Application-specific TextGenerator subclasses may
also provide extensions to the template formatting function for convenience.
For example, a subclass could provide a format string `\%M' to print the
contents of a Move class in human readable format.

\section{Interface}

\emph{TextGenerator} provides a method for converting a given input string
into a template string that can be formatted with a predefined number of
input parameters.

\begin{description}
\item[text(s : const string\&) : boost::format] %
  Convert the given string template into a form ready to be instantiated
  with input parameters, as shown below.
\end{description}

\emph{Example. Using \emph{TextGenerator}.}
\begin{verbatim}
  const string& TXT_HELLO_USER = "Hello, %s!";
  boost::format fmt = generator.text(Text::TXT_HELLO_USER);
  fmt % currentUser.firstName();
  std::cout << fmt.str() << endl;
\end{verbatim}
given a current user with name ``George'', will produce the message ``Hello,
George!''.

\begin{quote}\it
  \textbf{Warning} --- This part of the interface is deprecated and left for
  maintenance of legacy code.
\end{quote}

\emph{TextGenerator} provides functions for adding text templates to
database and for returning boost::format object which contains such a text
template.

\paragraph{void addText(int textId, const std::string\& templateString)}
Adds a text template to the database and associates it to a integer
identification code \emph{textId}.  Text codes are enumerated values defined
in namespace \emph{Texts}.

\paragraph{boost::format text(int textId) throw (KeyNotFound)}
Retrieves the text template with code \emph{textId} from database and
creates a boost::format object that contains the text template.

\section{Using TextGenerator}
\label{sec:textgen-use}

When a subclass of TextGenerator adds its own text codes to the namespace in
its constructor, it must make sure that no existing text codes are
conflicting with the new codes (that is, have the same numeric value).  To
this purpose, TextGenerator provides a special code \verb|LAST__| that marks
the first numeric code that is not used by any text message of the base
class.

Example:
\begin{verbatim}
// TextGenerator.cc:
namespace Text {
  enum {
    TXT_HELLO_WORLD = 1;
    ...
    LAST__;   
  }; 
};

// SimulatorTexts.cc:
namespace Text {
  enum {
     TXT_WELCOME = LAST__;
     ...  
  }; 
};
\end{verbatim}



\chapter{Options Framework}

This chapter describes the framework for implementing application-wide setup
parameters and command line option parsing.

\section{Overview}

The options framework provides a uniform set of interfaces and services to
manage the setup values (parameters) that control the way standalone
applications work.
%
In addition, the framework provides support for a specialised form of setup
parameter, the command line option, including option parsing parsing for all
applications of the project that require it.

Other features of the framework are:
\begin{enumerate}
\item %
  Default options.
\item %
  Built-in support for option prefixes.
\item %
  Built-in support for brief and long option names.
\end{enumerate}

\subsection{Motivation}
\label{ssec:motivation}

The options framework allows all applications of the toolset to reuse a
common set of services (increased productivity) and provides a consistent
and uniform interface (syntax rules for parsing options, default options,
automatic self-documentation of each single option).

\subsection{Object Architecture}

At the root of the framework lies the \emph{Options} class, a container of
parameters. Each parameter has a name and a value, which is modeled by the
\emph{OptionValue} interface. There are many types of option values
(Boolean, real numbers, integer numbers, strings) and more can be added by
re-implementing the interface.

The \emph{CmdLineParser} class implements a parser of strings. Parsing
results in a set of options and their value, which can be stored in an
\emph{Options} instance. Internally, parsing is parcelled into
option-specific mini-parsers, which are implementations of
\emph{CmdLineOptionParser} interface. Each implementation class is specific
to a type of option values.

The \emph{CmdLineOptions} is a specialisation of the parser destined to
command lines coming from interactive shells. It has the additional
functionality of handling default command line switches, which are present
in all applications of the toolset.

\subsection{Command Line Options}
\label{ssec:cmdline-options}

Option parsing adheres to a number of conventions.
\begin{enumerate}
\item %
  All options have a (long) name and are introduced by a double hyphen,
  `\verb|--|'. The option parameter, when present, is given either like
  this: `\verb|--foo=2|' or like this: `\verb|--foo 2|'.
\item %
  Options may also have a short, one-character alias. When they are referred
  to by their short name, options are introduced by a single hyphen,
  `\verb|-|'. The option parameter, when present, is given like this:
  `\verb|-f 2|'. The blank between the short option name and the parameter
  may be omitted. `\verb|-f=2|' is illegal.
\item %
  Options may have at most one parameter (or argument). If an argument
  consists of several words, the blank characters must be escaped using
  double quotes (`\verb|"|') or backslashes (`\verb|\|').
\item %
  A command line may contain one or more arguments (free arguments).
\item %
  Free arguments must follow options.
\item %
  An isolated `\verb|--|' in the command line marks the end of the options.
  The rest in the line will be treated as command line arguments.
\item %
  The the relative order of options in the command line is free.
\item %
  Switches (Boolean options) do not take parameters. Several switches may be
  grouped if they are expressed with their short name. For example,
  ``\verb|-abc|'' is equivalent to ``\verb|-a -b -c|''. It is \emph{not}
  allowed to group together switches and an option with parameter.
\end{enumerate}

\subsection{Default Options for Command Lines}
\label{ssec:default-opts}

The options framework provides two default (that is, automatically defined)
options, destined to applications run from a command line:

\begin{enumerate}
\item %
  \verb|--help| or \verb|-h| %
  Display the help page of the application: a brief description of the
  application and an automatically-built list of options accepted and their
  usage.
\item %
  \verb|--version|%
  Display the version number of the application.
%
  \note{EXTENSION: version number of the toolset}
\end{enumerate}

\subsection{Option Prefixes: Negation}
\label{ssec:prefixes-negation}

Boolean flag options can be specified in two forms: plain and negated.
Negated form is `\verb|--no-|' followed by the option name. Notice that
negated form applies only to the long name. Negation is supported only
boolean options.
%
\note{EXTENSION: support negation for any option}

\section{Interface}

\subsection{Options}
\label{ssec:Options-if}

The \emph{Options} base class is a generic container of option values
(concrete implementations of \emph{OptionValue} class). Each option is
identified by a unique name within the container. The interface of
\emph{Options} allows for:
\begin{enumerate}
\item %
  Adding (or setting again) a new option value to a given option.
\item %
  Accessing values of a given option.
\item %
  Counting the number of items of options that can have multiple items
  (lists of values).
\end{enumerate}

\subsection{OptionValue}

\emph{OptionValue} is an abstract interface for representing any possible
value (argument) of a setup parameter or option. An option value does not
bear any link with the parameter or option it defines a value for.

There are as many implementations of the \emph{OptionValue} interface as
data types for option values. The following types are defined: Boolean,
integer, real (double-precision floating-point), string.

The interface of \emph{OptionValue} provides setter and accessor methods for
every supported concrete type.

\begin{quote}
  \textbf{Warning.} This design does not allow seamless system
  extensibility. Any new concrete option value type requires interface
  changes to the toolkit.
\end{quote}

While every concrete option value type implies a domain of valid values
\emph{per se}, any further semantic restrictions of option value types that
depend on specific option types must be dealt with within the parsers (Sections~\ref{ssec:CmdLineParser}, \ref{ssec:CmdLineOptionParser}).

\subsection{CmdLineParser}
\label{ssec:CmdLineParser}

Class \emph{CmdLineParser} provides functionality to parse strings of
options (in the style of shell command line options).
%
A legal string consists of a variety of options and arguments. Command line
options have a mandatory long name, like `foo', and an optional short name,
like `f'.

\begin{description}
\item[CmdLineParser(string description)]%
  Constructor. The description of the application is given as parameter.
  Initialize the legal prefixes of options. Currently the only legal prefix
  is ``no-''.

\item[parse(options : vector<string>)]%
  Parse the string vector \emph{options}. The first element of the vector is
  treated as the name of the command and is stored.

  Throw \emph{IllegalCommandLine} exception if the command line is malformed
  (causes parsing errors). Throw \emph{ParserStopRequest} exception if one
  of the parsed options requests termination of execution.

\item[parse(argv : (char*){[ ]}, argc : int)]%
  Parse the command line \emph{argv}. The first element of vector
  \emph{argv} is treated as the name of the command and is stored.

  Throw \emph{IllegalCommandLine} exception if the command line is malformed
  (causes parsing errors). Throw \emph{ParserStopRequest} exception if one
  of the parsed options requests termination of execution.

\item[numberOfArguments() : int]%
  Return the number of independent arguments in command line.

\item[argument(int index) : string]%
  Return the independent argument at position \emph{index}, where 1 is the
  position of the first argument.

  Abort if \emph{index} is less than 1 or greater than the number of
  arguments.
%
  \note{TO DO: change to exception}
\item[storeOptions(options : Options\&)]%
  Store the parsed option values into an object that implements
  \emph{Options} interface.


\item[addOption(option : CmdLineOption*)]%
  Add a new command line option to command line option data base. This
  method is protected; only concrete options classes can use it to create
  the set of recognised options.
\end{description}

\subsection{CmdLineOptions}

Class \emph{CmdLineOptions} extends the command line parser
(Section~\ref{ssec:CmdLineParser}) with functionality specific to
interactive command-line shells: default, system-wide command line options
and output to a stream.

\begin{quote}
  \textbf{Warning.} This class is mainly meant for backward compatibility.
  It should not be used for new applications.
\end{quote}

\begin{description}
\item[printHelp()]%
  Print the help text that describes the application and its command line
  options.

\item[printVersion()]%
  Print the version of the program. Derived classes must implement this
  method.
\end{description}

\subsection{CmdLineOptionParser}
\label{ssec:CmdLineOptionParser}

\emph{CmdLineOptionParser} is an abstract base class for all different types
of option parsers. Each concrete implementation implements a mini-parser of
a specific type of option values. In addition to Boolean values, which have
no argument, the framework defines implementations for options that accept
arguments of the following types: integer, string, real, and integer list.

\begin{description}
\item[CmdLineOptionParser(name : string, description : string desc, alias :
  string)]%
  Constructor. Create a new command line option and set its name,
  description, and optional alias (short name). If no alias string is given,
  then the option can only be referred to by its (long) name.

\item[longName() : string]%
  Return the (long) name of the option.

\item[shortName() : string]%
  Returns the alias short name of the option. Return an empty string if the
  option does not have an alias.
%
  \note{DISCUSS: safer to throw?}

\item[description() : string]%
  Return the description of an option.

\item[parseValue(arguments : string, prefix : string) : bool]%
  Parse the option name. It this method is invoked repeatedly for the same
  option, two things can occur: if the option type is a scalar, the new
  value overwrites the preexisting value; if the option type is an
  aggregate, then the new value is added.

  Throw \emph{IllegalCommandLine} exception if the arguments or the prefix
  of the option are not valid and cause a parsing error.

\item[integer(index : int) : int]%
  Return the integer value of the option. Can only be used for options which
  expect an integer parameter. If the option parameter is an integer list,
  the list elements can be accessed by the optional \emph{index} parameter.

  Abort if \emph{index} is less than 1 or greater than the number of
  arguments.
%
  \note{TO DO: change to exception}

\item[String() : string]%
  Return the string value of the option. Can be only used for options which
  expect a string parameter.
%
  \note{TO DO: rename to ``stringValue''}

\item[real() : double]%
  Return the floating point value of option. Can be only used for options
  which expect a floating point number as parameter.

\item[isFlagOn() : bool]%
  Returns true if the option is set. Can only be used with Boolean options.

\item[isFlagOff() : bool]%
  Returns true if the option is not set. Can only be used with Boolean
  options.

\item[listSize() : int]%
  Return the number of elements in the (list) argument of the option. Can
  only be used for which expect a comma-separated list of elements as
  parameter.
\end{description}

\section{Usage of Command Line Option Framework}

The options framework can parse command lines options either by adding new
\emph{CmdLineOptionParser} concrete objects to a \emph{CmdLineParser}. The
resulting option values must be stored in an instance of \emph{Options}
class or of an application-specific subclass.

Alternatively, it is possible to subclass the legacy \emph{CmdLineOptions}.
In this case, the application-specific options should be added inside the
constructor, during initialisation. The subclass can be a complete wrapper
around the option set, providing specific methods for accessing the value of
each option. Using the application-specific command line class includes
creation of options, parsing of command line and obtaining the values of
different options and arguments.

\subsection{Usage Example}

Example of usage of the option system with the base command line parser and
an application-specific option container (subclass of \emph{Options}).

\begin{verbatim}
CmdLineParser cmdLineParser;

// Set the command line "grammar" to the parser. Here it is done manually,
// but it is possible to derive a specialised CmdLineOptions parser that
// will add the supported parameters automatically in its constructor.

cmdLineParser.addOption(new StringCmdLineOptionParser(
    ADF_PARAM_NAME, "The ADF file of the target architecture", "t");

// parse command line and store options

SchedulerOptions options;
cmdLineParser.parseCommandLine(cmdLine); // cmdLine may be a string vector
cmdLineParser.storeOptions(options);

// get the value of an option via base Options interface
cout << options.optionValue(ADF_PARAM_NAME).stringValue();

// get the value of an option via a specialised accessor
cout << options.targetADFFileName();
\end{verbatim}

\subsection{Alternative Usage Example}

Eaxample of usage with application-specific subclassing of the legacy class
\emph{CmdLineOptions}.

\begin{verbatim}
// FooCmdLineOptions.cc:
FooCmdLineOptions::FooCmdLineOptions() : 
    CmdLineOptions("Usage: foo [OPTION]...") {

    BoolCmdLineOptionParser* binMode =
        new BoolCommandLineOptionParser(
            "binmode", "sets the output mode to binary", "b");
    addOption(binMode);

    StringCmdLineOptionParser* fooName =
        new StringCmdLineOptionParser("name", "name of the foo", "n");
    addOption(fooName);

    IntegerCmdLineOptionParser* fooSize =
        new StringCmdLineOptionParser("size", "size of the foo", "s");
    addOption(fooSize);
}

bool FooCmdLineOptions::binMode() {
    return findOption("binmode")->isFlagOn();
}

string FooCmdLineOptions::fooName() {
    return findOption("name")->String();
}

int FooCmdLineOptions::fooSize() {
    return findOption("size")->integer();
}

// main.cc
int main(int argc, char* argv[]) {
    FooCmdLineOptions options;

    if(!options.parseCommandLine(argv, argc)) {
        return EXIT_FAILURE;
    }

    int arguments = options.numberOfArguments();
    string input1 = "";
    string input2 = "";
    if (arguments == 0 || arguments > 2) {
        options.printHelp();
        return EXIT_FAILURE;
    } else if (arguments == 1) {
        input1 = options.argument(1);
    } else {
        input1 = options.argument(1);
        input2 = options.argument(2);
    }

    string name = options.fooName();
    if (name == "pelle") {
        input1 += "pelle";
        input2 += "pelle";
    }

    if (options.binMode()) {
        cout << Conversion::toBinString(options.fooSize()) << endl;
    } else {
        cout << options.fooSize() << endl;
    }
}
\end{verbatim}



\chapter{XMLSerializer}

Many of the modules in TCE use some kind of data files, either to read
information or to store (temporary) information to a permanent support. Many
of the data files used in this project are in XML format~\cite{XMLSpec}. XML
is a way to represent data in plain text format using a tree-like structure
to markup the text in various ways.

Since all modules need their own kinds of data files they all need their own
ways to read and/or write the data. \emph{XMLSerializer} together with
\emph{ObjectState} objects provide a generic way to handle XML files.

\section{Main Design Ideas}

\emph{XMLSerializer} is a general XML reader-writer application for the TCE
project. It uses objects of the ObjectState class to represent the XML
element tree in the XML file. ObjectState is a class whose instances can be
used as nodes of a tree structure or as single data holders. An object of
the ObjectState class can have attributes and the actual payload text data.

\emph{XMLSerializer} uses Xerces C++ Parser~\cite{Xerces} to parse and write
the actual XML format. When reading an XML document Xerces creates a DOM
tree~\cite{DOM} that represents the XML element tree in the original file.
\emph{XMLSerializer} then converts the DOM tree to \emph{ObjectState} tree
that has the same structure as the original XML file, i.e. one element in
the file is one object in the \emph{ObjectState} tree. When writing an XML
file the \emph{XMLSerializer} uses Xerces to first create a DOM tree from a
\emph{ObjectState} tree and then to write the DOM tree to the file.

\section{External Library Dependencies}

\emph{XMLSerializer} uses the Xerces C++ Parser to read and parse as well as
to write the actual XML format.

\section{Interface}

\emph{XMLSerializer} provides a public interface to read and write XML
files. The interface includes the following functions.

\paragraph{writeState (model : Serializable\&, fileName : const string\&)}
Writes the given serializable model into the given XML file.

\paragraph{readState (fileName : const string\&, schemaFile : const
  string\&) : ObjectState*}
Reads the given XML file and creates an \emph{ObjectState} tree according to
it. The XML file is validated with the given Schema~\cite{Schema} file. If
the Schema file is not given the XML file is not validated. Returns the root
node of the created \emph{ObjectState} tree.

\paragraph{writeState(rootState : ObjectState*, fileName : const string\&)}
Writes the given \emph{ObjectState} tree into the given XML file.

\section{Implementation}

\emph{XMLSerializer} is the generic application when handling XML files. It
creates an \emph{ObjectState} tree that mirrors the element structure in the
XML file. When someone wants to create a more application specific XML file
handler (i.e. the needed \emph{ObjectState} tree is something else than the
one \emph{XMLSerializer} produces) one must use the \emph{XMLSerializer}
class as a base class and inherit a new serializer from it.

The inherited serializer's task is to restructure the \emph{ObjectState}
tree \emph{XMLSerializer} creates into a tree that is suitable for the
application at hand. The inherited serializer should still use the
\emph{XMLSerializer}'s interface to read and write XML files and define the
\emph{readState} and \emph{writeState} functions again overloading them. It
can also create its own (private) methods it needs to restructure and
otherwise handle the \emph{ObjectState} tree.

The objects created by the \emph{ObjectState} tree must implement the
\emph{Serializable} interface.

\subsection{Already Implemented Serializers}

The following serializers have already been implemented in TCE. Some hints
on the implementation of a new serializer may be found in their
documentation.

\subsubsection{ADFSerializer}

\emph{ADFSerializer} restructures the \emph{ObjectState} tree created by
\emph{XMLSerializer} from an ADF file~\cite{ADF-specs} so that mirrors the
internal machine object model~\cite{MOM-design}. After restructuring, one
node (object) in the modified \emph{ObjectState} tree represents one object
in the machine.

Naturally, \emph{ADFSerializer} also provides methods to create an
\emph{ObjectState} tree from the internal machine. The \emph{ObjectState}
tree representing the machine is constructed by the machine itself, but
\emph{ADFSerializer} restructures the tree so that it can be written to an
XML of the ADF format. The actual writing of the XML file is again done by
\emph{XMLSerializer}.

\subsubsection{ProDeOptionsSerializer}

\emph{ProDeOptionsSerializer} is used to read and write
ProDe~\cite{ProDeReq,ProDeDes} options from or to an XML file. It
restructures the \emph{ObjectState} tree between the \emph{XMLSerializer}
and the ProDe program.

\section{Error Handling}

All errors are signalled by exceptions.



\chapter{Event Handler}
\label{ch:event-handler}

\section{Overview}

Event Handler is an event handling system consisting of two main roles as
illustrated in Figure~\ref{fig:event_handler_module}.  \emph{Informer} is a
class that notifies \emph{Listener} instances when an event that an instance
registered for takes place. Events are notified to \emph{Informer} by a
client, and passes on the notification to interested \emph{Listener}
objects.

\begin{figure}[tb]
\centerline{\psfig{figure=eps/EventHandler.eps,width=0.8\textwidth}}
\caption{Event Handler module.}
\label{fig:event_handler_module}
\end{figure}

\section{Classes}

\subsection{Informer}

Informer is a class which keeps a registry of events connected
to listeners interested in them. Informing \emph{Informer} about new events
and (un)registering \emph{Listeners} for events is done using following
methods.

\paragraph{handleEvent(event : int)}
Looks up all \emph{Listeners} from the registry and calls
\verb|handleEvent()| for each of them.

\paragraph{registerListener(event: int, listener : Listener*)}
Registers a \emph{Listener} instance to listen for \emph{event} type of
events. Values of \emph{event} are from application specific enumeration
types. After registration, the \emph{Listener} starts to receive event
notifications when the specific \emph{event} occurs.

\paragraph{unregisterListener(event : int, listener : Listener*)}
Unregisters a \emph{Listener} instance from a specific \emph{event}. 
After unregistration, the \emph{Listener} stops receiving event notifications
for that specific event type.

\subsection{Listener}

The \emph{Listener} interface class provides the base interface implemented
by all event handlers. With this interface the \emph{Listener} is informed
that an event has taken place.  The listener checks whether it is its
responsibility to handle such event and, if that is the case, takes
appropriate action.

With the event notification to \emph{Listener} objects no additional
information about the event is sent. It is responsibility of the
\emph{Listener}, by checking the registered objects (its event context), to
determine further actions.

\paragraph{handleEvent()}
This method is called by \emph{Informer} when an event the \emph{Listener}
is interested in happens. It's then up to the listener to take appropriate
action. For example, some listeners may listen to clock cycle events and
write tracing data to database.



\chapter{Application Services}

\section{Overview}

Application services consists mainly of an \emph{Application} class, which
provides project-wide basic application services. These services include
assertion, debugging aid, running system (shell) commands, managing signal
handlers, and so on.

Application services include also a few macro definitions that are generally
useful thoughout the toolset code.



\chapter{Text Logger Server}

This chapter describes the module that provides logging to output stream.

\section{Overview}

The architecture of the logging system of any application of the project is
subdivided in two parts: the client side and the server side. While the
client side is completely application specific and must be designed and
implemented case-by-case, the server side is a service offered by the
toolkit. The responsibility of the logging server is also called ''logging
proper'', since it consists of sending data to an output stream.

The main class that encapsulate the logging machinery is \emph{TextLogger}.
This class provides few simple services:
\begin{enumerate}
\item %
  Define an output stream where messages are stored (possibly: a void
  stream, meaning that the log data is discarded).
\item %
  Provide simple text formatting capabilities, like indentation, automatic
  line breaking and automatic message numbering.
\end{enumerate}

\section{Interfaces}

\subsection{TextLogger}
\label{ssec:TextLogger-if}

Provides the basic interface for clients that need to log messages.

\begin{description}
\item[log(message : const string\&)]%
  Log a message to the output stream. If necessary, open or create the
  output stream. Internally, this method may format and augment the given
  message.

  Throw appropriate \emph{IOException} subclass when the message cannot be
  written to the output stream.

\item[setLineWidth(width : int)]%
  Set a new maximum line width. The logger will try to break all text lines
  of following messages to ensure they are not longer than \emph{width}. It
  the shortest sequence of characters delimited by white space is longer
  than \emph{width} characters, then the line will be broken at the end of
  the sequence.

  Throw \emph{OutOfRange} if the given width is smaller than 1.
\end{description}

\subsection{StreamTextLogger}

Implements logging servers that write to output streams.

\begin{description}
\item[TextLogger(output : const string\&)]%
  Constructor. Create a text logger and prepare to write to an output stream
  with given name. The constructor does not fail if the output stream cannot
  be opened or written to.

\item[TextLogger(output : std::ostream\&)]%
  Constructor. Create a text logger and prepare to write to given output
  stream. The constructor does not fail if the output stream cannot be
  opened or written to.

\item[$\sim$TextLogger()]%
  Destructor. Flush and close the output stream if open.

\end{description}

\section{Usage of the Logging Server}

Clients of the logging system will usually access only the basic interface
\emph{TextLogger}.

The following example shows a controller that creates a server logger and
sends to two separate clients (Scheduler pass modules, in this case) the
request to record the logger server.
\begin{verbatim}
  StreamTextLogger server(logFileName);
  clientA.setLogger(server);
  clientB.setLogger(server);
\end{verbatim}

Inside the clients, loggers should not be used directly. Instead, the client
should construct an object that implements the client-side of logging
activity (in particular, formatting the string and testing for the verbosity
level) and send all requests to this object (the logger client).

In the following example, the module that implements the register allocation
pass (client) constructs the logger client object automatically when the
server logger is registered:

\begin{verbatim}
  setLogger(TextLogger& server) {
      int logVerbosity = globalOptions_.verbosity();
      if (logVerbosity == 0) {
          // leave current empty logger client, logging disabled
      } else {
          delete clientLogger_;
          clientLogger_ = new ConcreteClientLogger(server, this, logVerbosity);
      }
  }
\end{verbatim}

The concrete logger client class \emph{ConcreteClientLogger} provides
methods to log context- and data-dependent messages. Inside these methods,
the message to log is composed and the logging request is forwarded to the
server:\footnote{
%
  Note that the example show just one possible implementation of the client
  logger. Many alternatives (some significantly more efficient) are
  possible.}

\begin{verbatim}
  logSpillRegister(Node& node, int cost) {
      if (verbosity_ < SPILL_NODE_FOUND_VERBOSITY) {
          return;
      }
      // "Spill candidate found: %1%. Spill cost heuristic: %2%.
      boost::format message = gen.text(Text::RA_SPILL_NODE_FOUND);
      message % node->destination() % cost;
      server_.log(message, id());
  }
\end{verbatim}

A run of this logging system could produce a line of log text like the
following:
\begin{verbatim}
  Spill candidate found: r.567. Spill cost heuristic: 234.
\end{verbatim}



\chapter{REJECTED ALTERNATIVES}

\begin{description}
\item[22.04.2004] --- Complex plugin module registry.

  The plugin module registry could be augmented with two features.  First,
  the order of registration of the modules is the same order in which
  modules are searched for symbols.  Second, it is not an error to search
  several modules with the same name, all registered in the registry.  A
  releated helper service for these features is the possibility to register
  with a single call all modules with a given name that are found in any of
  the paths in the search list.

  This feature has been rejected due to the increased complexity and the
  potential loss of generality.  It is relatively easy to implement above
  described features on top of \emph{PluginTools}.

\item[29.10.2005] --- Extra capabilities of logger server.
  \begin{enumerate}
  \item %
    Support for multiple streams.
  \item %
    Automatic support for a minimalistic log files structure, with headers
    defining sections and sub-sections.
  \end{enumerate}
  Multiple log streams can be generated in parallel by instantiate multiple
  loggers. Log file structure can be created (with more effort, but more
  flexibly) by clients themselves.
\end{description}



\chapter{IDEAS FOR FURTHER DEVELOPMENT}

\begin{description}
\item[19.05.2004] --- XMLSerializer: Using SAX instead of DOM with Xerces.

  DOM is quite a heavy representation of an XML file. It might be possible
  to use SAX instead of DOM with Xerces to read and write XML files.
%
  -- L.~Laasonen

\item[20.10.2005] --- TextGenerator: Support for multiple streams.

  Instead of mapping all log requests to the same output stream, the logger
  server could also provide support for multiple streams. Each client would
  be linked to a single stream. Several clients can share the same stream.
%
  -- A.~Cilio

\item[23.01.2006] --- Version number of the main libraries of the toolset.

  In addition to the application-specific version number, the version option
  of the options framework could display also the version number of the
  toolset base infrastructure (the domain and the toolkit library versions).
%
  -- A.~Cilio
\end{description}



% \chapter{PENDING ISSUES}

% Pending issues concerning this design, a sort of TODO list.
% This chapter should be empty when implementation of this module/subsystem
% is completed.


% ------------------------------------------------------------------------


%  References are generated with BibTeX from a bibtex file.
\bibliographystyle{alpha}
\cleardoublepage
%% Equivalent to a chapter in the table of contents
\addcontentsline{toc}{chapter}{BIBLIOGRAPHY}
\bibliography{Bibliography}



\end{document}

%%% Local Variables: 
%%% mode: latex
%%% TeX-master: t
%%% End: 
