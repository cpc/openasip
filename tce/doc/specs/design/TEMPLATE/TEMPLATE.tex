\documentclass[a4paper,twoside]{tce}
\usepackage{pslatex}

\begin{document}
\author{[author's name]}
\title{[name of designed module/subsystem]}
\ver{0.1}
\firstday{[creation date]}
\lastday{[last modified]}
% id number in S- sequence
\docnum{\#\#\#}
% draft/complete/committed
\state{draft}

\maketitle


\chapter*{Document History}

\begin{HistoryTable}

 0.1    & dd.mm.yyyy & [A. Name]       & [what was added]\\
 ver    & dd.mm.yyyy & [A. Name]       & [what was added]\\

\end{HistoryTable}


%\tableofcontents



\chapter{INTRODUCTION}

\section{Purpose and Scope}

%  Describe the purpose of the document - example:
%  This document describes ...

\section{Definitions}

\begin{description}
\item[term]%
  term definition
\end{description}

\section{Acronyms and Abbreviations}

\begin{longtable}{p{0.15\textwidth}p{0.85\textwidth}}
[acronym] & [description] \\
 \ldots &  \ldots \\
\end{longtable}



\chapter{SYSTEM/MODULE OVERVIEW/ARCHITECTURE}
%% TO DECIDE THE MOST APPROPRIATE NAME

\section{``Philosophy of Design''}

% Describe the principles behind this design and give an overview of
% it. Motivate briefly the design if necessary.

%% No description of the module purpose and motivation, that belongs to the
%% functional specification document.

\section{External Modules/ Application Environment}

% Overview of module from an external perspective:
% - dependencies with external modules-subsystems
% - IFs of external modules that are used
% - external (file) data formats used

% Do not put here the interfaces provided to other modules.

\section{Architecture of Databases}

% Describes the data types defined and used by the module and their
% relations.  The description is at type-accurate level. For example, the
% description may consist of UML class diagrams with the attributes and
% interfaces of classes of the target program. Another example, the database
% tables are defined precisely, at level of single fields.

\section{? Main Restrictions of Implementation}

%% This section is in design document of SW document course
%% I need to know precisely what it should contain to decide if it's useful
%% enough for us.

\section{Module Communication}

%% Remove this section if the design document describes just one module (or
%% an application consistingof a single module).

% This section describes the communication between modules of the
% application or subsystem documented in this design document.



\chapter{Module Design}

%% This chapter is repeated for each module in case this design document
%% describes a multi-module larger module (or application).

\section{Overview}

% Overview of the module design in concrete terms.

%% Remove this section (redundant) if the design document describes a single
%% module (or an application consisting of a single module).

\section{Interfaces}

% High level description of the interface of this module.  Low level details
% are possible, but should be restricted to simple classes (like, for
% example, NodeDescriptor in mdf module).  Low level details of large
% interfaces, if present at all, should be stored in a separate UML
% specification file (for example, Rational Rose MDL).

\section{Implementation}

% Describe here important low-level details. Only details about complex
% problems that require ``smart solution'' should be treated.  Do not
% describe every single detail of the implementation when that follows from
% the specification/design in a straightforward manner.

\section{Error Handling}

% How are errors handled? What is considered internal and what is considered
% external source of errors? What is handled by throwing exceptions and what
% by assertions? What are the 


% ADD NEW SECTIONS ON DESIGN HERE



\chapter{REJECTED ALTERNATIVES}

% Rejected alternatives for (parts of) the design should be listed here with
% the reasoning and date the alternative was dumped. For future reference.

%\begin{description}
%\item[18.11.2003 --- rejected idea] %
%  Description of rejected idea.
%\end{description}



\chapter{IDEAS FOR FURTHER DEVELOPMENT}

% Ideas that are not part of the design yet but that might be added in the
% future are listed here. This is to have ideas written down somewhere.
% This list should contain the date of addition, the idea described briefly
% and the inventor of the idea.

%\begin{description}
%\item[18.11.2003 --- rejected idea] %
%  Description of rejected idea.
%%
%  --- Author
%\end{description}



\chapter{PENDING ISSUES}

% Pending issues concerning this design, a sort of TODO list.
% This chapter should be empty when implementation of this module/subsystem
% is completed.



\chapter{MAINTENANCE}

% This chapter treats the problem of extending the design with new
% capabilities that fit in a predefined framework.

% For example, in case of the TPEF module, this chapter could describe the
% steps that must be taken in order to add a new Section subclass and/or a
% SectionReader subclass.

% Remove this chapter no obvious and standardised way to extend the design
% with new capabilities exists.


% ------------------------------------------------------------------------

%% Uncomment this part if there are references.  Usually there will be at
%% least a reference to the functional specifications of the same module.

%  References are generated with BibTeX from a bibtex file.
%\bibliographystyle{alpha}
%\cleardoublepage
%% Equivalent to a chapter in the table of contents
%\addcontentsline{toc}{chapter}{BIBLIOGRAPHY}
%\bibliography{Bibliography}


\end{document}

%%% Local Variables: 
%%% mode: latex
%%% TeX-master: t
%%% End: 
